\documentclass[11pt]{ctexart}
\usepackage[margin=2cm,a4paper]{geometry}
\usepackage{amsthm, amsfonts, amsmath, amssymb, mathrsfs, newclude, tikz-cd, tikz, ctex, mathtools, stmaryrd, datetime}


%\setmainfont{Caladea}

%% 也可以选用其它字库:
% \setCJKmainfont[%
%   ItalicFont=AR PL KaitiM GB,
%   BoldFont=Noto Sans CJK SC,
% ]{Noto Serif CJK SC}
% \setCJKsansfont{Noto Sans CJK SC}
% \renewcommand{\kaishu}{\CJKfontspec{AR PL KaitiM GB}}



\usepackage[colorlinks = true,
linkcolor = blue,
urlcolor  = blue,
citecolor = blue,
anchorcolor = blue]{hyperref}

% Include the x-color package for color support
\usepackage{xcolor}

% Define a new environment for red comments
\usepackage{verbatim} % Required for the comment environment
\usepackage{environ}

\usepackage{mdframed} % Include mdframed for creating framed environments

\definecolor{pinked}{RGB}{255,231,229} % Define a base color 
% Define a new environment with a background color
\newmdenv[
  backgroundcolor=pinked, % Set the desired background color
  linecolor=white, % Optional: Set the border line color
  linewidth=1pt, % Optional: Set the border line width
  roundcorner=5pt, % Optional: Set rounded corners
  nobreak=true % Optional: Prevent page breaks within the environment
]{pinked}

\theoremstyle{definition}
\newtheorem{qqq}{问题}[section]

\newcommand{\ExternalLink}{%
    \tikz[x=1.2ex, y=1.2ex, baseline=-0.05ex]{% 
        \begin{scope}[x=1ex, y=1ex]
            \clip (-0.1,-0.1) 
                --++ (-0, 1.2) 
                --++ (0.6, 0) 
                --++ (0, -0.6) 
                --++ (0.6, 0) 
                --++ (0, -1);
            \path[draw, 
                line width = 0.5, 
                rounded corners=0.5] 
                (0,0) rectangle (1,1);
        \end{scope}
        \path[draw, line width = 0.5] (0.5, 0.5) 
            -- (1, 1);
        \path[draw, line width = 0.5] (0.6, 1) 
            -- (1, 1) -- (1, 0.6);
        }
    }

\NewEnviron{aaa}{~\\
    \noindent {\textcolor{teal}{\textbf{解答}} \BODY }
}

\NewEnviron{llll}{
    \noindent {~\\$\ExternalLink$ 外部链接 $\,\,\,$ \color{blue}\url{\BODY} }
}

\renewcommand{\proofname}{证明}
\renewcommand\qedsymbol{${\boxed{\substack{\textit{完证}\\\textit{毕明}}}}$}


% Define a custom command for \kuing
\newcommand{\kuing}{\texorpdfstring{$\textstyle{\int_u^c k}=\texttt{kuing}$}{}}



% Change equation numbering to include the section number
\usepackage{cleveref}
\renewcommand{\theequation}{\thesection.\thesubsection.\arabic{equation}}
\numberwithin{equation}{section}


%我可以在这里定义一些简化的命令一图便利
\newcommand{\sspan}{\operatorname{span}}%定义span正体
\newcommand{\op}[1]{\operatorname{#1}}%简化函数表达
\newcommand{\cnm}[2]{\binom{#1}{#2} }%简化组合数命令
\newcommand{\set}[2]{\{ #1 \mid #2\}}%定义集合表示
\newcommand{\FF}{\mathbb{F}}%数域
\newcommand{\RR}{\mathbb{R}}
\newcommand{\CC}{\mathbb{C}}
\newcommand{\QQ}{\mathbb{Q}}


%%注意:  行内公式强制行间形式用"  \limits  ",反之用  "  \nolimits   ";
%%注意:  公式引用编号 :" \label{}  ";   公式显示编号" \tag{}  ":  引用公式"  \eqref{}    "


%截止到这里是我定义的

\usepackage{listings}
% Define listings style
\lstset{
  frame=tb,
  language=TeX,
  aboveskip=3mm,
  belowskip=3mm,
  showstringspaces=false,
  columns=flexible,
  basicstyle={\small\ttfamily},
  numbers=none,
  breaklines=true,
  breakatwhitespace=true,
  tabsize=3
}

\theoremstyle{definition}
\newtheorem*{definition}{定义}
\newtheorem*{proposition}{命题}
\newtheorem*{theorem}{定理}
\newtheorem*{notation}{记号}
\newtheorem*{example}{例子}
\newtheorem*{exercise}{习题}
\theoremstyle{remark}
\newtheorem*{remark}{备注}
\newtheorem*{lemma}{引理}
\newtheorem*{corollary}{推论}



\title{高等代数 (荣誉) I 作业模板}
\author{请输入姓名}

\setcounter{section}{-1}

\setcounter{page}{0}

\setlength\parindent{0pt}

\begin{document}

\maketitle

\section{说明}

可以将作业中遇到的问题标注在此. 如有, 请补充.

\tableofcontents

\newpage

%%%%%%%%%%%%%%%%%%%%%%%%%%%%%%%%%%%%%%%%%%%%%%
%%%%%%%%%%%%%%%%%%%%%%%%%%%%%%%%%%%%%%%%%%%%%%
%%%%%%%%%%%%%%%%%%%%%%%%%%%%%%%%%%%%%%%%%%%%%%
%%%%%%%%%%%%%%%%%%%%%%%%%%%%%%%%%%%%%%%%%%%%%%
%%%%%%%%%%%%%%%%%%%%%%%%%%%%%%%%%%%%%%%%%%%%%%
%%%%%%%%%%%%% 请从此处开始阅读 %%%%%%%%%%%%%%%%

\subsection{Exercise 1}
Characterise the ceigenspace corresponding to the ceigenvalue $0$.
\begin{aaa}
    $\bar{A}$ 的零空间
\end{aaa}
\subsection{Exercise 2}
Demonstrate that if $\lambda$ is a ceigenvalue, then so too is $e^{i \theta} \cdot \lambda$. Furthermore, show that for any ceigenvector $v$, there exists a ceigenvector associated with a ceigenvalue $\geq 0$.
\begin{aaa}
    考虑$A\bar{v}=\lambda v$, $\Rightarrow A(\bar{e^{-i\theta}}v)=\lambda e^{2i\theta}(e^{-i\theta}v)$
\end{aaa}
\subsection{Exercise 3}
Suppose $(v, \sigma)$ is a ceigenpair of $A$; then $(v, \sigma \overline \sigma)$ is an eigenpair of $A \overline A$. In particular,

- if $A \in \mathbb C^{d \times d}$ possesses $d$ ceigenpairs which are linearly independent over $\mathbb C$, then $A\overline A$ is diagonalisable with all eigenvalues non-negative.

Now consider the following **partial converse statement**:

- $A$ has no ceigenvectors if and only if $A \overline A$ admits no eigenvalues in $\mathbb R_{\geq 0}$;
- $A \in \mathbb C^{d \times d}$ has $d$ linearly independent ceigenpairs over $\mathbb C$ if and only if $A\overline A$ is diagonalisable with non-negative eigenvalues.
\begin{aaa}
    $A\bar{A}v=\sigma\bar{\sigma}v$, 
\end{aaa}
\subsection{Exercise 4}
Elucidate **Exercise 3** (particularly the warning) with the following example:
$$
A = \begin{pmatrix} 1 & i \\ 0 & 1 \end{pmatrix},\quad v = \begin{pmatrix}1 \\ i \end{pmatrix}.
$$
\begin{aaa}
    $A\bar{A}=\begin{pmatrix} 1 & i \\ 0 & 1 \end{pmatrix}\begin{pmatrix} 1 & -i \\ 0 & 1 \end{pmatrix} = \begin{pmatrix}
    1&0\\0&1
    \end{pmatrix}$
    因此$v$是$A\bar{A}$的特征向量,
    但是可以计算$v$不是$A$的c特征向量.
\end{aaa}
\subsection{Exercise 5}
This serves to complete **Exercise 2**. Suppose $\{(v_i , \lambda_i)\}_{i=1}^n$ are ceigenpairs of $A$ such that $|\lambda_i| \neq |\lambda_j|$; show that $\{v_i\}_{i=1}^n$ are linearly independent.
\begin{aaa}
    $\{(v_i , |\lambda_i|^2)\}_{i=1}^n$也是$A\bar{A}$的特征对, 由于$|\lambda_i|^2\neq |\lambda_j|^2$, 所以$\{v_i\}_{i=1}^n$线性无关.
\end{aaa}
\subsection{Exercise 6}
We aim to identify the analogue of the Jordan canonical form in the c-version, namely, $X =P J\overline P^{-1}$. We commence with a very special case:
\\- $P\overline {P} = I$ if and only if $P = Q\overline Q^{-1}$ for some invertible $Q$.

\begin{aaa}
    "$\Rightarrow$":  $P\bar{P}=I$, Let $Q=P+I$, then $ P\bar{Q}=P\bar{P}+P=P+I+Q$
\end{aaa}
\subsection{Exercise 7}
The most elementary Jordan matrix is the diagonal matrix. Show that $A$ is cdiagonalisable (i.e., $A = PD\overline P^{-1}$ for some $P$) if and only if $A\overline A$ is diagonalisable. In this scenario, $A\overline A$ has only non-negative eigenvalues.\
\begin{aaa}
    "$\leftarrow$": $A\bar{A}=P^{-1}DP$ and $D>0$, then we can find $P$ such that $A=P\sqrt{D}P^{-1}$\\
\end{aaa}
\subsection{Exercise 8}
(Nilpotent part). Show that there exists a matrix $P$ such that $PA\overline P^{-1} = \binom{N \ \ O}{O \ \ B}$, where $N$ comprises nilpotent Jordan blocks and $B$ is invertible.
\begin{aaa}
    把$A\bar{A}$进行$Jordan$
\end{aaa}
\subsection{Exercise 9}
To analyse the Jordan form of $A$, we begin by studying the Jordan form of $A \overline A$. Demonstrate the following:

1. $A \overline A$ has a characteristic polynomial with real coefficients;\\
2. non-real Jordan blocks occur in conjugate pairs;\\
3. if $\dim \ker (\lambda I - A\overline A) = 1$ and $\lambda \in \mathbb R$, then $\lambda \geq 0$;
\begin{aaa}
    1. Ex 2. $A\bar{A}$ 的特征值都是实数, 其特征多项式是实系数的.\\
    2. Ex 12. $A$ is csimilar to $\bar{A}$ and $\bar{A\bar{A}}=\bar{A}A$
    3.$v^*A\bar{A}v=\lambda v^*v$ , 左为正, 故$\lambda >0$
\end{aaa}
\subsection{Exercise 10}
\begin{aaa}
    对$A\bar{A}$正特征值部分直接取对应特征向量, 对复数特征值部分, 交替选取$z$和$\bar{z}$对应特征向量, 对负数特征值部分, 个数一定是偶数, 然后对对应 Jordan 块直和得到半-Jordan 块后再计算 .
\end{aaa}
\end{document}