\documentclass[11pt]{ctexart}
\usepackage[margin=2cm,a4paper]{geometry}
\usepackage{amsthm, amsfonts, amsmath, amssymb, mathrsfs, newclude, tikz-cd, tikz, ctex, mathtools, stmaryrd, datetime}


%\setmainfont{Caladea}

%% 也可以选用其它字库:
% \setCJKmainfont[%
%   ItalicFont=AR PL KaitiM GB,
%   BoldFont=Noto Sans CJK SC,
% ]{Noto Serif CJK SC}
% \setCJKsansfont{Noto Sans CJK SC}
% \renewcommand{\kaishu}{\CJKfontspec{AR PL KaitiM GB}}



\usepackage[colorlinks = true,
linkcolor = blue,
urlcolor  = blue,
citecolor = blue,
anchorcolor = blue]{hyperref}

% Include the x-color package for color support
\usepackage{xcolor}

% Define a new environment for red comments
\usepackage{verbatim} % Required for the comment environment
\usepackage{environ}

\usepackage{mdframed} % Include mdframed for creating framed environments

\definecolor{pinked}{RGB}{255,231,229} % Define a base color 
% Define a new environment with a background color
\newmdenv[
  backgroundcolor=pinked, % Set the desired background color
  linecolor=white, % Optional: Set the border line color
  linewidth=1pt, % Optional: Set the border line width
  roundcorner=5pt, % Optional: Set rounded corners
  nobreak=true % Optional: Prevent page breaks within the environment
]{pinked}

\theoremstyle{definition}
\newtheorem{qqq}{问题}[section]

\newcommand{\ExternalLink}{%
    \tikz[x=1.2ex, y=1.2ex, baseline=-0.05ex]{% 
        \begin{scope}[x=1ex, y=1ex]
            \clip (-0.1,-0.1) 
                --++ (-0, 1.2) 
                --++ (0.6, 0) 
                --++ (0, -0.6) 
                --++ (0.6, 0) 
                --++ (0, -1);
            \path[draw, 
                line width = 0.5, 
                rounded corners=0.5] 
                (0,0) rectangle (1,1);
        \end{scope}
        \path[draw, line width = 0.5] (0.5, 0.5) 
            -- (1, 1);
        \path[draw, line width = 0.5] (0.6, 1) 
            -- (1, 1) -- (1, 0.6);
        }
    }

\NewEnviron{aaa}{~\\
    \noindent {\textcolor{teal}{\textbf{解答}} \BODY }
}

\NewEnviron{llll}{
    \noindent {~\\$\ExternalLink$ 外部链接 $\,\,\,$ \color{blue}\url{\BODY} }
}

\renewcommand{\proofname}{证明}
\renewcommand\qedsymbol{${\boxed{\substack{\textit{完证}\\\textit{毕明}}}}$}


% Define a custom command for \kuing
\newcommand{\kuing}{\texorpdfstring{$\textstyle{\int_u^c k}=\texttt{kuing}$}{}}



% Change equation numbering to include the section number
\usepackage{cleveref}
\renewcommand{\theequation}{\thesection.\thesubsection.\arabic{equation}}
\numberwithin{equation}{section}


%我可以在这里定义一些简化的命令一图便利
\newcommand{\sspan}{\operatorname{span}}%定义span正体
\newcommand{\op}[1]{\operatorname{#1}}%简化函数表达
\newcommand{\cnm}[2]{\binom{#1}{#2} }%简化组合数命令
\newcommand{\set}[2]{\{ #1 \mid #2\}}%定义集合表示
\newcommand{\FF}{\mathbb{F}}%数域
\newcommand{\RR}{\mathbb{R}}
\newcommand{\CC}{\mathbb{C}}
\newcommand{\QQ}{\mathbb{Q}}


%%注意:  行内公式强制行间形式用"  \limits  ",反之用  "  \nolimits   ";
%%注意:  公式引用编号 :" \label{}  ";   公式显示编号" \tag{}  ":  引用公式"  \eqref{}    "


%截止到这里是我定义的

\usepackage{listings}
% Define listings style
\lstset{
  frame=tb,
  language=TeX,
  aboveskip=3mm,
  belowskip=3mm,
  showstringspaces=false,
  columns=flexible,
  basicstyle={\small\ttfamily},
  numbers=none,
  breaklines=true,
  breakatwhitespace=true,
  tabsize=3
}

\theoremstyle{definition}
\newtheorem*{definition}{定义}
\newtheorem*{proposition}{命题}
\newtheorem*{theorem}{定理}
\newtheorem*{notation}{记号}
\newtheorem*{example}{例子}
\newtheorem*{exercise}{习题}
\theoremstyle{remark}
\newtheorem*{remark}{备注}
\newtheorem*{lemma}{引理}
\newtheorem*{corollary}{推论}



\title{第四周作业}
\author{董仕强}

\setcounter{section}{-1}

\setcounter{page}{0}

\setlength\parindent{0pt}

\begin{document}

\maketitle

\section{说明}

可以将作业中遇到的问题标注在此. 如有, 请补充.

\tableofcontents

\newpage

%%%%%%%%%%%%%%%%%%%%%%%%%%%%%%%%%%%%%%%%%%%%%%
%%%%%%%%%%%%%%%%%%%%%%%%%%%%%%%%%%%%%%%%%%%%%%
%%%%%%%%%%%%%%%%%%%%%%%%%%%%%%%%%%%%%%%%%%%%%%
%%%%%%%%%%%%%%%%%%%%%%%%%%%%%%%%%%%%%%%%%%%%%%
%%%%%%%%%%%%%%%%%%%%%%%%%%%%%%%%%%%%%%%%%%%%%%
%%%%%%%%%%%%% 请从此处开始阅读 %%%%%%%%%%%%%%%

\section{Problem Set For March 2025}
\subsection{Exercise 1}
Let $V= \CC[x] $, and let $w= \frac{-1+\sqrt3\op{i}}{2}$ denote a cubic root of unite.
\begin{enumerate}
    \item Prove that each $V_k:=\{f\in \CC [x]\mid f(x)=w^k\cdot f(wx)\}$ is a linear subspace of $V$.
    \item Prove that $V=V_0\oplus V_1\oplus V_2$.
    \item Prove that the mapping \[L: V\to V, \qquad f(x)\mapsto w\cdot f(wx)-w^2f(w^2x)\]is a linear transformation.
    \item Compute $\op{im}(L)$ and $\ker(L)$.
    \item Say $(u,\lambda \in V\times \CC)$ is an eigenppair of $L$, provide $u\neq 0$ and $Lu=\lambda u$.Determine all eigenpairs of $L$.
\end{enumerate}
\begin{aaa}
    \begin{enumerate}
        \item $V_k\neq 0$.\\$\forall f,g \in V_k,f(x)+g(x)=w^k\cdot f(wx)+w^k\cdot g(wx)=w^k\cdot(f(wx)+g(wx))\in V_k$\\$\forall f\in V_k , \lambda \in \CC$, $\lambda f(x)=\lambda w^k \cdot f(wx)=w^k \cdot \lambda f(wx)\in V_k$.
        \item Suppose $f\in V_0\cap V_1$, then $f(x)=w^0\cdot f(wx)=w^1\cdot f(wx)$, which implies $f(x)=0$. \\$\forall f(x)=\sum_{i=0}^n a_ix^i$ , let $f_m(x)=\sum_{k=0}^{s_m}a_{3k+m}x^{3k+m},m=0,1,2$ $\Rightarrow$ $f_m\in V_m$.
        \item $L(af+bg)\\=w\cdot (af+bg)(wx)-w^2\cdot (af+bg)(w^2x)\\=a(w\cdot f(wx)-w^2\cdot f(w^2x))+b(w\cdot f(wx)-w^2\cdot f(w^2x))\\=aL(f)+bL(g)$.
        \item $\op{im} (L)$:\\$\forall f(x)=\sum_{i=0}^n a_ix^i$, $L(f)=w\cdot f(wx)-w^2\cdot f(w^2x)\\=w\cdot \sum_{i=0}^n a_i(wx)^i-w^2\cdot \sum_{i=0}^n a_i(w^2x)^i\\=\sum_{i=0}^{n}(w^{i+1}-w^{2i+2})a_ix^i\\=\sum_{k=0}^{s}(w-w^2)(a_{3k}x^{3k}-a_{3k+1}x^{3k+1})$\\so , $\op{im}(L)=V_0\oplus V_1$\\$\ker(L)$:\\$\forall f(x)=\sum_{i=0}^n a_ix^i$, $L(f)=0\\ \Rightarrow \sum_{i=0}^{n}(w^{i+1}-w^{2i+2})a_ix^i=0\\ \Rightarrow a_{3k}=a_{3k+1}=0$\\so, $\ker(L)=V_2$.
        \item $L(f(x))=wf(wx)-w^2f(w^2)x$,\quad $L(f(wx))=wf(w^2x)-w^2f(x)$,\quad $L(f(w^2x))=wf(x)-w^2f(wx)$\\$A=\begin{pmatrix}
            0 & w & -w^2\\
            -w^2 & 0 & w\\
            w & -w^2 & 0
        \end{pmatrix}$
    \end{enumerate}
\end{aaa}
\subsection{Exercise 2}
\begin{aaa}
    \begin{enumerate}
        \item $V=\FF[x]$ , $f: g(x)\mapsto x\cdot g(x)$
        \item $V=\FF[x]$ , $f: g(x)\mapsto g'(x)$
        \item ?
        \item $V=\FF[x]$ , $f: h(x)\mapsto xh(x)$ , $g: h(x) \mapsto \frac{h(x)-h(0)}{x}$.
    \end{enumerate}
\end{aaa}
\section{Problem Set for 13 March 2025}
\subsection{Exercise 1}
Let $f: U \to V$ be linear map and $\op{dim}U=n,\op{din}V=m$.
\begin{enumerate}
    \item Let $E \subset V$ be an $k$-dimensional subspace. Prove that $$f^{-1}(E)\geq n-m+k$$.
    \item When $f$ is surjective , the equality holds.
\end{enumerate} 
\begin{aaa}
    \begin{enumerate}
        \item Let $g:f^{-1}(E)\to E$ , then $\ker g=\ker f$ , $\op{im}(g) = \op{im}(f)\cap E$.\\$\dim(f^{-1}(E))=\dim(\op{im}(g))+\dim(\ker(g))$.\\$\dim(E\cap \op{im}f)\geq \dim(E)+\dim(\op{im}(f))-\dim(V)$\\$\dim(V)=\dim(\op{im}(f))+\dim(\ker(f))$.\\Q.E.D
        \item $f$ is surjective , $\op{im}(f)=V$ , $E+\op{im}f=V$
    \end{enumerate}
\end{aaa}
\subsection{Exercise 2}
\begin{aaa}
    有限维数的线性映射和矩阵密切联系,在抽象的线性映射和运算中,几乎都能找到具体的矩阵和运算进行类比.矩阵也可以看作一种线性映射
\end{aaa}
\end{document}