\documentclass[11pt]{ctexart}
\usepackage[margin=2cm,a4paper]{geometry}
\usepackage{amsthm, amsfonts, amsmath, amssymb, mathrsfs, newclude, tikz-cd, tikz, ctex, mathtools, stmaryrd, datetime}


%\setmainfont{Caladea}

%% 也可以选用其它字库:
% \setCJKmainfont[%
%   ItalicFont=AR PL KaitiM GB,
%   BoldFont=Noto Sans CJK SC,
% ]{Noto Serif CJK SC}
% \setCJKsansfont{Noto Sans CJK SC}
% \renewcommand{\kaishu}{\CJKfontspec{AR PL KaitiM GB}}



\usepackage[colorlinks = true,
linkcolor = blue,
urlcolor  = blue,
citecolor = blue,
anchorcolor = blue]{hyperref}

% Include the x-color package for color support
\usepackage{xcolor}

% Define a new environment for red comments
\usepackage{verbatim} % Required for the comment environment
\usepackage{environ}

\usepackage{mdframed} % Include mdframed for creating framed environments

\definecolor{pinked}{RGB}{255,231,229} % Define a base color 
% Define a new environment with a background color
\newmdenv[
  backgroundcolor=pinked, % Set the desired background color
  linecolor=white, % Optional: Set the border line color
  linewidth=1pt, % Optional: Set the border line width
  roundcorner=5pt, % Optional: Set rounded corners
  nobreak=true % Optional: Prevent page breaks within the environment
]{pinked}

\theoremstyle{definition}
\newtheorem{qqq}{问题}[section]

\newcommand{\ExternalLink}{%
    \tikz[x=1.2ex, y=1.2ex, baseline=-0.05ex]{% 
        \begin{scope}[x=1ex, y=1ex]
            \clip (-0.1,-0.1) 
                --++ (-0, 1.2) 
                --++ (0.6, 0) 
                --++ (0, -0.6) 
                --++ (0.6, 0) 
                --++ (0, -1);
            \path[draw, 
                line width = 0.5, 
                rounded corners=0.5] 
                (0,0) rectangle (1,1);
        \end{scope}
        \path[draw, line width = 0.5] (0.5, 0.5) 
            -- (1, 1);
        \path[draw, line width = 0.5] (0.6, 1) 
            -- (1, 1) -- (1, 0.6);
        }
    }

\NewEnviron{aaa}{~\\
    \noindent {\textcolor{teal}{\textbf{解答}} \BODY }
}

\NewEnviron{llll}{
    \noindent {~\\$\ExternalLink$ 外部链接 $\,\,\,$ \color{blue}\url{\BODY} }
}

\renewcommand{\proofname}{证明}
\renewcommand\qedsymbol{${\boxed{\substack{\textit{完证}\\\textit{毕明}}}}$}


% Define a custom command for \kuing
\newcommand{\kuing}{\texorpdfstring{$\textstyle{\int_u^c k}=\texttt{kuing}$}{}}



% Change equation numbering to include the section number
\usepackage{cleveref}
\renewcommand{\theequation}{\thesection.\thesubsection.\arabic{equation}}
\numberwithin{equation}{section}


%我可以在这里定义一些简化的命令一图便利
\newcommand{\sspan}{\operatorname{span}}%定义span正体
\newcommand{\op}[1]{\operatorname{#1}}%简化函数表达
\newcommand{\cnm}[2]{\binom{#1}{#2} }%简化组合数命令
\newcommand{\set}[2]{\{ #1 \mid #2\}}%定义集合表示
\newcommand{\FF}{\mathbb{F}}%数域
\newcommand{\RR}{\mathbb{R}}
\newcommand{\CC}{\mathbb{C}}
\newcommand{\QQ}{\mathbb{Q}}


%%注意:  行内公式强制行间形式用"  \limits  ",反之用  "  \nolimits   ";
%%注意:  公式引用编号 :" \label{}  ";   公式显示编号" \tag{}  ":  引用公式"  \eqref{}    "


%截止到这里是我定义的

\usepackage{listings}
% Define listings style
\lstset{
  frame=tb,
  language=TeX,
  aboveskip=3mm,
  belowskip=3mm,
  showstringspaces=false,
  columns=flexible,
  basicstyle={\small\ttfamily},
  numbers=none,
  breaklines=true,
  breakatwhitespace=true,
  tabsize=3
}

\theoremstyle{definition}
\newtheorem*{definition}{定义}
\newtheorem*{proposition}{命题}
\newtheorem*{theorem}{定理}
\newtheorem*{notation}{记号}
\newtheorem*{example}{例子}
\newtheorem*{exercise}{习题}
\theoremstyle{remark}
\newtheorem*{remark}{备注}
\newtheorem*{lemma}{引理}
\newtheorem*{corollary}{推论}



\title{第十周作业}
\author{董仕强}

\setcounter{section}{-1}

\setcounter{page}{0}

\setlength\parindent{0pt}

\begin{document}

\maketitle

\section{说明}

可以将作业中遇到的问题标注在此. 如有, 请补充.

\tableofcontents

\newpage

%%%%%%%%%%%%%%%%%%%%%%%%%%%%%%%%%%%%%%%%%%%%%%
%%%%%%%%%%%%%%%%%%%%%%%%%%%%%%%%%%%%%%%%%%%%%%
%%%%%%%%%%%%%%%%%%%%%%%%%%%%%%%%%%%%%%%%%%%%%%
%%%%%%%%%%%%%%%%%%%%%%%%%%%%%%%%%%%%%%%%%%%%%%
%%%%%%%%%%%%%%%%%%%%%%%%%%%%%%%%%%%%%%%%%%%%%%
%%%%%%%%%%%%% 请从此处开始阅读 %%%%%%%%%%%%%%%%

\section{Exercise}
\begin{qqq}
    1. ($V$ is non-trivial)** Demonstrate that $V$ is not finite-dimensional. 
\end{qqq}\
\begin{aaa}
    考虑$a(x)=\begin{cases}
    e^{1/x^2-1}, \quad |x|<1 \\
    0, \quad |x|\geq1
    \end{cases}$,
    $a_n(x)=a(x-2n)$,显然每个$a_n(x)$都是$V$中的元素,且$a_n(x)$是线性无关的,那么$V$中有无穷多个线性无关的元素,所以$V$不是有限维的.
\end{aaa}
\begin{qqq}
    2. (Differential Operator)** Show that the operator $D: V \to V$, defined by $f \mapsto f'$, is well-defined. Determine the kernel $\ker D$, the image $\operatorname{im} D$, and the cokernel $\operatorname{coker} D$.
\end{qqq}
\begin{aaa}
    因为$V$中的元素都是光滑函数,所以它们的导数也是光滑函数.且大于0的部分其导数也是0,因此$D(f)$也在$V$中,所以$D$是良定义的.\\
    2. $\ker D=0$,因为$D(f)=0$当且仅当$f$是常数函数,而常数函数在$V$中只有0.\\
    3. $\operatorname{im} D={f \in V\mid \exists x_0,\int _{-x_0}^{x_0} f \op{d}x=0 }$\\
    4. $\operatorname{coker} D= C+\op{im} D$, $C\in \RR$
\end{aaa}
\begin{qqq}
    3. (Integrable Functionals)** Let $\varphi: \mathbb{R} \to \mathbb{R}$ be a function that is Riemann integrable over any bounded interval $[a,b]$. Show that the following map defines an element of $V^\ast$:
$$
V \to \mathbb{R}, \quad f \mapsto \int_{-\infty}^{\infty} f(x)\varphi(x)\, \mathrm{d}x.
$$
\end{qqq}
\begin{aaa}
    定义该映射为$\sigma$,那么$\sigma$是线性映射,因为$\int_{-\infty}^{\infty} (af_1+bf_2)\varphi(x)\, \mathrm{d}x=a\int_{-\infty}^{\infty} f_1(x)\varphi(x)\, \mathrm{d}x+b\int_{-\infty}^{\infty} f_2(x)\varphi(x)\, \mathrm{d}x$,所以$\sigma$是线性映射.\\
\end{aaa}
\begin{qqq}
    ($V^\ast$, merging some functions, and introducing some non-functions)**. The preceding exercise defines a mapping
$$
\Phi : \{\text{Locally Riemann-Integrable Functions}\} \to V^\ast, \quad f \mapsto \left[g \mapsto \int_{-\infty}^\infty f(x)g(x)\, \mathrm{d}x\right].
$$
It is known that $\Phi$ is neither injective nor surjective.  
- Let $f$ be the function defined by $f(x) = 0$ for all $x \neq 1$, and $f(1) = 1$. Then $f \in \ker \Phi$.  
- The Dirac delta functional $\delta$, defined informally below, is not in the image of $\Phi$.
\end{qqq}
\begin{aaa}
    1. 令$f(x)=\begin{cases}
    0,\quad x\neq 1\\
    1,\quad x=1
    \end{cases}$,那么$f(x)g(x)=\begin{cases}
        0,\quad x\neq 1\\
        g(1),\quad x=1
        \end{cases}$,由数学分析知识知道$\int_{-\infty}^{\infty}f(x)g(x)\op{d}x=0$, 也就是$\Phi(f)=0$  , 但是 $f\neq 0$ , 因此$\Phi$ 不是单射\\
        ?f不属于$V$,应该定义$f(x)=\lim_{n\to \infty}exp(1-1/(1-(nx^2)))=\lim_{n\to\infty}b(nx)$,别的地方全是0?\\\[{}\]
    2.Dirac delta function $\delta$,不能被写成某两个正常黎曼可积的乘积的无穷积分.?
\end{aaa}
\begin{qqq}
    5. ($V \hookrightarrow V^\ast$)** Show that $\ker \Phi \cap V = \{0\}$, and thus that $V$ can be regarded as a subspace of $V^\ast$.

\end{qqq}
\begin{aaa}
    令$f\in V$且满足对任意$\varphi\in V$都有$\int_{-\infty}^{infty}f\varphi\op{d}x=0$, 令$\varphi =f$ ,再由于$f$ 连续可以得到$f=0$
\end{aaa}
\begin{qqq}
    6. (The Dirac Delta Functional)** The Dirac delta functional $\delta$ is informally described as:
$$
\delta(x) = 
\begin{cases}
0, & x \neq 0, \\
\infty, & x = 0,
\end{cases} \quad \int_{-\infty}^\infty \delta(x)\, \mathrm{d}x = 1.
$$
Provide a formal definition of $\delta$ as an element of $V^\ast$.
\end{qqq}
\begin{aaa}
    $\delta(f)=f(0)$,那么$\delta(af+bg)=(af+bg)(0)=af(0)+bg(0)=a\delta(f)+b\delta(g)$
\end{aaa}
\begin{qqq}
    *7. (Generalised Derivatives)** Define $\varphi$ as a piecewise linear function passing through the points:
$$
(-\infty, 0) \to (-1,0) \to (0,2) \to (1,-1) \to (2,0) \to (+\infty, 0).
$$
Although $\varphi$ is continuous, its derivative $\varphi'$ is not classically defined. Use the identity
$$
\int_{-\infty}^\infty f(x)g'(x)\, \mathrm{d}x = -\int_{-\infty}^\infty f'(x)g(x)\, \mathrm{d}x
$$
to define $\varphi'$ in the distributional sense. Express $\varphi''$ explicitly as a linear combination of shifted Dirac delta functions $\delta(x-a)$.
\end{qqq}
\begin{aaa}
$\varphi'=2\delta(x)-\delta(x-1)$\\
$\varphi''=2\delta^2(x)-\delta^2(x-1)$
\end{aaa}
\begin{qqq}
    8. (Generalised Limits)** For each $\varphi \in V \hookrightarrow V^\ast$, define a sequence of functions for each $n \in \mathbb{N}_+$ by:
$$
(-)_n : V \to V, \quad \varphi \mapsto \left[x \mapsto n \cdot \varphi(nx)\right].
$$
It is evident that:
$$
\int_{-\infty}^\infty \varphi_n(x)\, \mathrm{d}x = \int_{-\infty}^\infty \varphi(x)\, \mathrm{d}x.
$$
The sequence $\{\varphi_n\}$ does not converge uniformly in $V$. However, show that there exists $L \in V^\ast$ such that:
$$
L(f) = \lim_{n \to \infty} \int_{-\infty}^\infty f(x)\varphi_n(x)\, \mathrm{d}x.
$$
Your task is to show that the limit exists for arbitrary $f$, and that $L$ is linear.

\end{qqq}
\begin{aaa}
    \begin{align*}
        \lim_{n \to \infty} \int_{-\infty}^\infty f(x)\varphi_n(x)\, \mathrm{d}x &= \lim_{n \to \infty} \int_{-\infty}^\infty nf(x)\varphi(nx)\, \mathrm{d}x \\
        &= \lim_{n \to \infty} \int_{-\infty}^\infty f(x/n)\varphi(x)\, \mathrm{d}x \\
    \end{align*}
    \begin{align*}
        |\int_{-\infty}^{\infty}(f(x/n)-f(0))\varphi(x)\op{d}x|&=|\int_{-A}^{A}(f(x/n)-f(0))\varphi(x)\op{d}x|\\
        &< |M_n\int_{-A}^{A}\varphi(x)\op{d}x|.\\
    \end{align*}
    其中$M_n=\max_{-A<x<A}|(f(x/n)-f(0))|=\max_{-A/n<x<A/n}|(f(x/n)-f(0))|\to 0,\, n\to \infty$\\
    因此$L(f)=f(0)\int_{-\infty}^{\infty}\varphi(x)\op{d}x$
    他是线性的.
\end{aaa}
\end{document}