\documentclass[11pt]{ctexart}
\usepackage[margin=2cm,a4paper]{geometry}
\usepackage{amsthm, amsfonts, amsmath, amssymb, mathrsfs, newclude, tikz-cd, tikz, ctex, mathtools, stmaryrd, datetime}


%\setmainfont{Caladea}

%% 也可以选用其它字库:
% \setCJKmainfont[%
%   ItalicFont=AR PL KaitiM GB,
%   BoldFont=Noto Sans CJK SC,
% ]{Noto Serif CJK SC}
% \setCJKsansfont{Noto Sans CJK SC}
% \renewcommand{\kaishu}{\CJKfontspec{AR PL KaitiM GB}}



\usepackage[colorlinks = true,
linkcolor = blue,
urlcolor  = blue,
citecolor = blue,
anchorcolor = blue]{hyperref}

% Include the x-color package for color support
\usepackage{xcolor}

% Define a new environment for red comments
\usepackage{verbatim} % Required for the comment environment
\usepackage{environ}

\usepackage{mdframed} % Include mdframed for creating framed environments

\definecolor{pinked}{RGB}{255,231,229} % Define a base color 
% Define a new environment with a background color
\newmdenv[
  backgroundcolor=pinked, % Set the desired background color
  linecolor=white, % Optional: Set the border line color
  linewidth=1pt, % Optional: Set the border line width
  roundcorner=5pt, % Optional: Set rounded corners
  nobreak=true % Optional: Prevent page breaks within the environment
]{pinked}

\theoremstyle{definition}
\newtheorem{qqq}{问题}[section]

\newcommand{\ExternalLink}{%
    \tikz[x=1.2ex, y=1.2ex, baseline=-0.05ex]{% 
        \begin{scope}[x=1ex, y=1ex]
            \clip (-0.1,-0.1) 
                --++ (-0, 1.2) 
                --++ (0.6, 0) 
                --++ (0, -0.6) 
                --++ (0.6, 0) 
                --++ (0, -1);
            \path[draw, 
                line width = 0.5, 
                rounded corners=0.5] 
                (0,0) rectangle (1,1);
        \end{scope}
        \path[draw, line width = 0.5] (0.5, 0.5) 
            -- (1, 1);
        \path[draw, line width = 0.5] (0.6, 1) 
            -- (1, 1) -- (1, 0.6);
        }
    }

\NewEnviron{aaa}{~\\
    \noindent {\textcolor{teal}{\textbf{解答}} \BODY }
}

\NewEnviron{llll}{
    \noindent {~\\$\ExternalLink$ 外部链接 $\,\,\,$ \color{blue}\url{\BODY} }
}

\renewcommand{\proofname}{证明}
\renewcommand\qedsymbol{${\boxed{\substack{\textit{完证}\\\textit{毕明}}}}$}


% Define a custom command for \kuing
\newcommand{\kuing}{\texorpdfstring{$\textstyle{\int_u^c k}=\texttt{kuing}$}{}}



% Change equation numbering to include the section number
\usepackage{cleveref}
\renewcommand{\theequation}{\thesection.\thesubsection.\arabic{equation}}
\numberwithin{equation}{section}


%我可以在这里定义一些简化的命令一图便利
\newcommand{\sspan}{\operatorname{span}}%定义span正体
\newcommand{\op}[1]{\operatorname{#1}}%简化函数表达
\newcommand{\cnm}[2]{\binom{#1}{#2} }%简化组合数命令
\newcommand{\set}[2]{\{ #1 \mid #2\}}%定义集合表示
\newcommand{\FF}{\mathbb{F}}%数域
\newcommand{\RR}{\mathbb{R}}
\newcommand{\CC}{\mathbb{C}}
\newcommand{\QQ}{\mathbb{Q}}
\newcommand{\ZZ}{\mathbb{Z}}


%%注意:  行内公式强制行间形式用"  \limits  ",反之用  "  \nolimits   ";
%%注意:  公式引用编号 :" \label{}  ";   公式显示编号" \tag{}  ":  引用公式"  \eqref{}    "


%截止到这里是我定义的

\usepackage{listings}
% Define listings style
\lstset{
  frame=tb,
  language=TeX,
  aboveskip=3mm,
  belowskip=3mm,
  showstringspaces=false,
  columns=flexible,
  basicstyle={\small\ttfamily},
  numbers=none,
  breaklines=true,
  breakatwhitespace=true,
  tabsize=3
}

\theoremstyle{definition}
\newtheorem*{definition}{定义}
\newtheorem*{proposition}{命题}
\newtheorem*{theorem}{定理}
\newtheorem*{notation}{记号}
\newtheorem*{example}{例子}
\newtheorem*{exercise}{习题}
\theoremstyle{remark}
\newtheorem*{remark}{备注}
\newtheorem*{lemma}{引理}
\newtheorem*{corollary}{推论}



\title{第一周作业}
\author{董仕强}

\setcounter{section}{-1}

\setcounter{page}{0}

\setlength\parindent{0pt}

\begin{document}

\maketitle

\section{说明}

可以将作业中遇到的问题标注在此. 如有, 请补充.

\tableofcontents

\newpage

%%%%%%%%%%%%%%%%%%%%%%%%%%%%%%%%%%%%%%%%%%%%%%
%%%%%%%%%%%%%%%%%%%%%%%%%%%%%%%%%%%%%%%%%%%%%%
%%%%%%%%%%%%%%%%%%%%%%%%%%%%%%%%%%%%%%%%%%%%%%
%%%%%%%%%%%%%%%%%%%%%%%%%%%%%%%%%%%%%%%%%%%%%%
%%%%%%%%%%%%%%%%%%%%%%%%%%%%%%%%%%%%%%%%%%%%%%
%%%%%%%%%%%%% 请从此处开始阅读 %%%%%%%%%%%%%%%%

\section{Problem Set for 17-Feb-2025}
\subsection{Problem 1}
Let $\FF$ be an arbitrary field, and let $\FF[x]$ denote the polynomial ring(algebra) in one indeterminate. For the sake
of convention, assume that $x^0=1$ .
\begin{qqq}
    Demonastrate that $|FF[x]$ forms a vactor space over $\FF$ with the basis $\{x^n\}_{n\geq 0}$.
\end{qqq}
\begin{aaa}
    $\forall f,g,h \in \FF[x]$ , $f+g = g+f $ and $(f+g)+h=f+(g+h)$.\\
    $\forall f \in \FF[x], a.b\in \FF , a(bf)=(ab) f$ and $(a+b)f=af+bf$.\\
    $\forall f,g \in \FF[x],a\in \FF, a(f+g)=af+ag$.\\
    $\forall f \in \FF[x], 1\cdot f=f$.\\
    $\forall f \in \FF[x], 0\cdot f=0$.\\
    $\forall f \in \FF[x], \exists g \in \FF[x]$ such that $f+g=0$.\\
    $\forall f \in \FF[x], \exists a_i \in \FF$ such that $f=\sum_{i=0}^{n}a_i x^i$.\\
\end{aaa}
\begin{qqq}
    Determine  whether  the  set $\{x^n+2\cdot x^{n-1}\}_{n \geq 1}$ constitutes  a  basis  for $\FF[x]$ ,and provide your reasoning.
\end{qqq}
\begin{aaa}
    no.\\
    For example, 1 can not be expressed as a linear combination of $\{x^n+2\cdot x^{n-1}\}_{n \geq 1}$.
\end{aaa}
\begin{qqq}
    Investigate whether the series $\op{e}^x=1+\frac{x}{1!}+\frac{x^2}{2!}+\cdots$ belongs to $\FF[x]$.
\end{qqq}
\begin{aaa}
    no.\\
    For $f \in \FF[x]$,$f=\sum_{i=0}^{\infty}a_ix^i$, only a finite number of $a_i$ are non-zero.
\end{aaa}
\begin{qqq}
    Is it posible to define a linear map $\mathcal{L}: \FF\left\langle x\right\rangle \rightarrow \FF$ such that $\mathcal{L}(f)=l(f)$ for any $f \in \FF[x]$ .
\end{qqq}
\begin{aaa}
    yes.\\
    $$\mathcal{L}(f)=\lim_{n \to infty}\sum_{i=0}^{n}a_i$$
    if $f$ is a finate degree polynomial, higher degree terms are zero.
\end{aaa}



\section{Problem Set for 20-Feb-2025}
\subsection{Problem 1}
Find two linear maps\[\alpha,\beta: \FF[x]\to \FF[x],\]
such that \[\alpha(\beta(f))-\beta(\alpha(f))=f\]
for any $f \in \FF[x]$,\\
is it posible to find such $\alpha,\beta: V \to V$ when $V$ is of finite dimension?
\begin{aaa}
    \begin{enumerate}
        \item Let $\alpha(f)=f'$ , $\beta(f) = xf.$
        \item no.for finite $V$ ,$\alpha: V\to V$ can be represented by a matrix $A$ , $\beta: V\to V$ can be represented by a matrix $B$ , then $\alpha(\beta(f))-\beta(\alpha(f))=f$ is equivalent to $AB-BA=I$ , but $tr(AB-BA)=tr(AB)-tr(BA)=0\neq n$
    \end{enumerate}
\end{aaa}
\subsection{Problem 2}
$\op{Prove}$ the following:
\begin{enumerate}
    \item If $f$ is irreducible in $\ZZ[x]$,then it is also irreducible in $\QQ[x]$;\\
    \item If $f$ is irreducible in $\RR[x]$,then it is also irreducible in $\QQ[x]$.
\end{enumerate}
\begin{aaa}
    \begin{enumerate}
        \item If $f$ is reducible in $\QQ[x]$, then $f=gh$ , $g,h \in \QQ[x]$ , $g,h$ can be written as $g=\frac{a}{b}g_1$ , $h=\frac{c}{d}h_1$ , $a,b,c,d \in \ZZ$ , $g_1,h_1 \in \ZZ[x]$ , then $f=\frac{ac}{bd}g_1h_1$ , $f$ is reducible in $\ZZ[x]$.
        \item If $f$ is reducible in $\QQ[x]$, then $f=gh$ , $g,h \in \QQ[x]$ , $g,h$ can be written as $g=\frac{a}{b}g_1$ , $h=\frac{c}{d}h_1$ , $a,b,c,d \in \RR$ , $g_1,h_1 \in \RR[x]$ , then $f=\frac{ac}{bd}g_1h_1$ , $f$ is reducible in $\RR[x]$.
    \end{enumerate}
\end{aaa}
\subsection{Problem 3}
\begin{enumerate}
    \item Let $f \in \ZZ[x]$ be a monic polynomial of degree n. Denote the zaros of $f$ in $\CC$ by $(z_i)_{i=1}^n$. Show that, if there is existly one $z_i$ such that $|z_i| \geq 1$ and $f(0)\neq 0 $, then $f$ is irreducible in $\QQ[x]$.
    \item Let $f \in \ZZ[x]$ be a polynomial such that $f(0)$ is a prime. DEnote the zeros of $f$ in $\CC$ by $(z_i)_{i=1}^n$. Show that, if $|z_i| >1$ for all $i$ , then $f$ is irreducible.
    \item Let $f(x) = \sum_{k=0}^{n}a_k\cdot x^k\in \ZZ[x]$ be a polynomial with $f(0)$ prime. Suppose that $|a_0|>\sum_{k=1}^{n}|a_k|$. Show that $F$ is irreducible. 
\end{enumerate}
\begin{aaa}
\begin{enumerate}
    \item only prove $f$ is irreducible in $\ZZ[x]$.(Similarly hereinafter)\\
    If $f$ is reducible in $\ZZ[x]$, then $f=gh$ , $g,h \in \ZZ[x]$ .Let the absolute value of all the zeros of $g$ in $\CC$ is less than 1, then $|g(0)|=|g(z_1)g(z_2)\cdots g(z_n)|<1$ , and $g(0) \in \ZZ$ , so $g(0)=0$ ,$f(0)=g(0)h(0)=0$.
    \item If $f$ is reducible in $\ZZ[x]$, then $f=gh$ , $g,h \in \ZZ[x]$ .$|f(0)|=|g(0)||h(0)|$ and $|f(0)|$ is a prime. Let $|g(0)|=1$, so at least one of absolute value of zeros of $g$ is less than 1.
    \item If exists $z\in \CC$ such that $f(z)=0$ and $|z| \leq 1$, then $|a_0|=|\sum_{i=1}^{n}a_iz^i|\leq \sum_{i=1}^n |a_i|$.
\end{enumerate}
\end{aaa}

\subsection{Problem 4}
\begin{enumerate}
    \item Is there any irreducible $f(x)\in \ZZ[x]$ such that $f(f(x))$ is reducible?
    \item Prove that $1+\prod_{k=1}^{2025}(x-k)^2$ is irreducible in $\ZZ[x]$;
    \item Prove that $\prod_{k=1}^n(x-x_k)+1$is either irreducible in $\ZZ[x]$,or a perfect square;
    \item $(f\in \ZZ[x])$Prove that if $f(x)=1$ has $\geq 4$ solutions in $\ZZ$ ,then $f(x)=-1$ has no solutions in $\ZZ$.
    \item Prove that the partial sum $ (\op{e}^x)_{\op{deg}\leq n}$ is always irreducible in $\QQ[x]$.
\end{enumerate}
\begin{aaa}
    \begin{enumerate}
        \item Noticed thst $f(x)=x^2+10x+17$ satisfies the condition.
        \item can't solve
        \item Let $f(x)=\prod_{k=1}^n(x-x_k)+1$ , $(x_1 <x_2 < \cdots < x_n)$. If $f$ is reducible , Let $f=gh$,$g,h \in \ZZ[x]$ and $0 <\op{deg}(g),\op{deg}(h) < n$, then $\forall k$ , $g(x_k)h(x_k)=1$ , so $g(x_k)=h(x_k)=1$ , so $g=h$. That means $f=g^2$.
        \item If $f(x)=(x-a_1)(x-a_2)(x-a_3)(x-a_4)g(x)+1$, $a_1,a_2,a_3,a_4 \in \ZZ$ , $g \in \ZZ[x]$ , then $f(k)=-1 \Leftrightarrow (k-a_1)(k-a_2)(k-a_3)(k-a_4)g(k)=-2$ . so -2 must be divisible by $(k-a_1)(k-a_2)(k-a_3)(k-a_4)$ , but $-2$ is not divisible by 4.
        \item Let $f(x)=1+\sum_{k=1}^{n}\frac{x^k}{k!}$ , $n!\cdot f(x)=x^n + nx^{n-1} + \cdots + \frac{n!}{2!}x^2 + n!x + n!=g(x)$ is irreducible in $\ZZ[x]$.\\
            \href{https://www.zhihu.com/question/41438786?sort=created}{Schur}?
        
    \end{enumerate}
\end{aaa}
\end{document}