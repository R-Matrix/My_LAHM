\documentclass[11pt]{ctexart}
\usepackage[margin=2cm,a4paper]{geometry}
\usepackage{amsthm, amsfonts, amsmath, amssymb, mathrsfs, newclude, tikz-cd, tikz, ctex, mathtools, stmaryrd, datetime}


%\setmainfont{Caladea}

%% 也可以选用其它字库:
% \setCJKmainfont[%
%   ItalicFont=AR PL KaitiM GB,
%   BoldFont=Noto Sans CJK SC,
% ]{Noto Serif CJK SC}
% \setCJKsansfont{Noto Sans CJK SC}
% \renewcommand{\kaishu}{\CJKfontspec{AR PL KaitiM GB}}



\usepackage[colorlinks = true,
linkcolor = blue,
urlcolor  = blue,
citecolor = blue,
anchorcolor = blue]{hyperref}

% Include the x-color package for color support
\usepackage{xcolor}

% Define a new environment for red comments
\usepackage{verbatim} % Required for the comment environment
\usepackage{environ}

\usepackage{mdframed} % Include mdframed for creating framed environments

\definecolor{pinked}{RGB}{255,231,229} % Define a base color 
% Define a new environment with a background color
\newmdenv[
  backgroundcolor=pinked, % Set the desired background color
  linecolor=white, % Optional: Set the border line color
  linewidth=1pt, % Optional: Set the border line width
  roundcorner=5pt, % Optional: Set rounded corners
  nobreak=true % Optional: Prevent page breaks within the environment
]{pinked}

\theoremstyle{definition}
\newtheorem{qqq}{问题}[section]

\newcommand{\ExternalLink}{%
    \tikz[x=1.2ex, y=1.2ex, baseline=-0.05ex]{% 
        \begin{scope}[x=1ex, y=1ex]
            \clip (-0.1,-0.1) 
                --++ (-0, 1.2) 
                --++ (0.6, 0) 
                --++ (0, -0.6) 
                --++ (0.6, 0) 
                --++ (0, -1);
            \path[draw, 
                line width = 0.5, 
                rounded corners=0.5] 
                (0,0) rectangle (1,1);
        \end{scope}
        \path[draw, line width = 0.5] (0.5, 0.5) 
            -- (1, 1);
        \path[draw, line width = 0.5] (0.6, 1) 
            -- (1, 1) -- (1, 0.6);
        }
    }

\NewEnviron{aaa}{~\\
    \noindent {\textcolor{teal}{\textbf{解答}} \BODY }
}

\NewEnviron{llll}{
    \noindent {~\\$\ExternalLink$ 外部链接 $\,\,\,$ \color{blue}\url{\BODY} }
}

\renewcommand{\proofname}{证明}
\renewcommand\qedsymbol{${\boxed{\substack{\textit{完证}\\\textit{毕明}}}}$}


% Define a custom command for \kuing
\newcommand{\kuing}{\texorpdfstring{$\textstyle{\int_u^c k}=\texttt{kuing}$}{}}



% Change equation numbering to include the section number
\usepackage{cleveref}
\renewcommand{\theequation}{\thesection.\thesubsection.\arabic{equation}}
\numberwithin{equation}{section}


%我可以在这里定义一些简化的命令一图便利
\newcommand{\sspan}{\operatorname{span}}%定义span正体
\newcommand{\op}[1]{\operatorname{#1}}%简化函数表达
\newcommand{\cnm}[2]{\binom{#1}{#2} }%简化组合数命令
\newcommand{\set}[2]{\{ #1 \mid #2\}}%定义集合表示
\newcommand{\FF}{\mathbb{F}}%数域
\newcommand{\RR}{\mathbb{R}}
\newcommand{\CC}{\mathbb{C}}
\newcommand{\QQ}{\mathbb{Q}}


%%注意:  行内公式强制行间形式用"  \limits  ",反之用  "  \nolimits   ";
%%注意:  公式引用编号 :" \label{}  ";   公式显示编号" \tag{}  ":  引用公式"  \eqref{}    "


%截止到这里是我定义的

\usepackage{listings}
% Define listings style
\lstset{
  frame=tb,
  language=TeX,
  aboveskip=3mm,
  belowskip=3mm,
  showstringspaces=false,
  columns=flexible,
  basicstyle={\small\ttfamily},
  numbers=none,
  breaklines=true,
  breakatwhitespace=true,
  tabsize=3
}

\theoremstyle{definition}
\newtheorem*{definition}{定义}
\newtheorem*{proposition}{命题}
\newtheorem*{theorem}{定理}
\newtheorem*{notation}{记号}
\newtheorem*{example}{例子}
\newtheorem*{exercise}{习题}
\theoremstyle{remark}
\newtheorem*{remark}{备注}
\newtheorem*{lemma}{引理}
\newtheorem*{corollary}{推论}



\title{第十三周作业}
\author{董仕强}

\setcounter{section}{-1}

\setcounter{page}{0}

\setlength\parindent{0pt}

\begin{document}

\maketitle

\section{说明}

可以将作业中遇到的问题标注在此. 如有, 请补充.

\tableofcontents

\newpage

%%%%%%%%%%%%%%%%%%%%%%%%%%%%%%%%%%%%%%%%%%%%%%
%%%%%%%%%%%%%%%%%%%%%%%%%%%%%%%%%%%%%%%%%%%%%%
%%%%%%%%%%%%%%%%%%%%%%%%%%%%%%%%%%%%%%%%%%%%%%
%%%%%%%%%%%%%%%%%%%%%%%%%%%%%%%%%%%%%%%%%%%%%%
%%%%%%%%%%%%%%%%%%%%%%%%%%%%%%%%%%%%%%%%%%%%%%
%%%%%%%%%%%%% 请从此处开始阅读 %%%%%%%%%%%%%%%%

\subsection{Exercise 1}
Determine whether the following are inner product spaces (简要回答即可).

1. $V$ is the vector space continuous functions over $\mathbb R$, and the pairing is
   $$
   (f,g) \mapsto \int_0^1 f(x)g(x) \operatorname dx.
   $$

2. $V = \mathbb R[x]$, and the pairing is
   $$
   (f,g) \mapsto \int_{-\infty}^\infty f(x)g(x)e^{-x^2} \operatorname dx.
   $$

3. $V = \mathbb R^{n \times n}$, and the pairing is
   $$
   (A,B) \mapsto \det (A^T\cdot B).
   $$
\begin{aaa}
1. 是 \\
2. 是 , $e^{-x^2}$保证了收敛性 \\
3. 不是 , 没有双线性\\

\end{aaa}
\subsection{Exercise 2}
Show that every real matrix is similar to a block diagonal matrix, wherein the candidates of the blocks are

1. the Jordan form of a real eigenvalue;

2. the block of the form
   $$
   \begin{bmatrix}
   H_{r,\theta } & Q &  &  & \\
    & H_{r,\theta } & Q &  & \\
    &  & \ddots  & \ddots  & \\
    &  &  & H_{r,\theta } & Q\\
    &  &  &  & H_{r,\theta }
   \end{bmatrix},
   $$
    where

   1. $H_{r,\theta} = \begin{bmatrix}
      r\cos \theta  & -r\sin \theta \\
      r\sin \theta  & r\cos \theta 
      \end{bmatrix}$, and
   2. $Q = \begin{bmatrix}
      0 & 0 \\ 1 & 0
      \end{bmatrix}$.
\begin{aaa}
    对实矩阵进行Jordan分解, 其特征值为实数的保持不变\\
    对于复数特征值$\lambda=re^{i\theta}$, 可以把复数写成矩阵形式 
\end{aaa}
\subsection{Exercise 2.5}
Show that every real matrix is a product of two real symmetric matrices.
\begin{aaa}
    $A=P^{-1}JP$, 然后把每个块$J_i$乘次对角线全为1的矩阵得到$J_i'$, 那么$A=P^{-1}J'(P^{-1})^T\cdot P^TI'P$
\end{aaa}
\subsection{Exercise 3}
Let $(V, (-,-))$ be a finite dimensional real inner product space. For unit vector $v_0$ (i.e., $(v_0,v_0)=1$), we define the reflection
$$
\varphi : V \to V,\quad x \mapsto x - 2(x,v_0)\cdot v_0.
$$
Show that for isometric transform $\psi$ (i.e., $(\psi(u), \psi (v)) = (u,v)$), $1$ is an eigenvalue for $\psi$ or $\psi \circ \varphi$. Overall, $1 \in \sigma (\psi) \cup \sigma (\psi \circ \varphi)$.
\begin{aaa}
    假设1不是$\psi$的特征值,令$(\psi\circ \phi)(x)= x $ , 即 寻找非0的$x$ 使得 $(\psi-I)(x) = 2(x,v_0)\psi(v_0)$ , 即 $x = 2(x,v_0)(\psi-I)^{-1}\psi(v_0)$,$\Rightarrow$$(x,v_0) = 2(x,v_0)((\psi-I)^{-1}\psi(v_0),v_0)$\\
    所以$(\psi-I)^{-1}\psi(v_0)$非0, x非0
\end{aaa}
\subsection{Exercise 4}
Set $V:=\mathbb R[x]$ and $V_0:=\{f\in \mathbb R[x]\mid f(0)=f(1)\}$.

1. Prove that $V\times V\to \mathbb R,\quad (f,g)\mapsto \int_0^1 f(x)g(x)\operatorname dx$ is an inner product.

2. Set $\mathscr D:V_0\to V,\quad f(x)\mapsto f^\prime (x)$. Find $\dim\ker(\mathscr D)$ and $\dim\operatorname{coker}(\mathscr D)$.

3. Define the inner product restricted on the subspace
   $$
   (\cdot ,\cdot )_0:V_0\times V_0\to \mathbb R,\quad (f,g)\mapsto \int_0^1 f(x)g(x)\operatorname d x.
   $$
   Is there any linear map $\mathscr D^\ast:V\to V_0$ such that for any $h\in V_0$ and $g\in V$,
   $$
    (\mathscr D^\ast g,h)_0=(g,\mathscr Dh)?
   $$
\begin{aaa}
    1. 双线性, 对称, 正定\\
    2. $\dim\ker(\mathscr D) = 1$, $\dim\operatorname{coker}(\mathscr D) = 1$\\
    3. $\mathscr D^\ast=-\mathscr D$
\end{aaa}
\subsection{Exercise 5}
Let $(V, (,))$ be an inner product space, where $V$ is not necessary finite dimensional.

- Say $\varphi ^\star$ is the adjoint of $\varphi \in \mathrm{Hom}_{\mathbb R}(V,V)$, provided $(\varphi^\star (x), y) = (x , \varphi (y))$ for arbitrary $x,y\in V$.

- Let $U$ be a subspace of $V$. Set
  $$
  U^\perp := \{v \in V \mid (u, v) = 0 , \forall u \in U\}  =\bigcap _{u \in U}\ker((u,-))
  $$

Now we assume that $\varphi$ is invertible, and $\varphi^\star$ exists.\\

1. Show that $\varphi^\star$ is injective, and $(\mathrm{im}(\varphi ^\star))^\perp = 0$.\\
2. Show that $\varphi^\star$ is surjective $\implies$ $(\varphi^{-1})^\star$ exists and $(\varphi ^{-1})^\star = (\varphi^\star)^{-1}$.\\
3. Show that $(\varphi^{-1})^\star$ exists $\implies$ $\varphi^\star$ is invertible and $(\varphi ^{-1})^\star = (\varphi^\star)^{-1}$.\\
4. Let $V = \mathbb R[x]$ and $(f,g) := \sum_{n \geq 0}f_i g_i$, where $h = \sum h_i\cdot x^i$. Show that\\
   1. $(V, (,))$ is an inner product space;\\
   2. $L : f \mapsto \frac{f-f(0)}{x}$ is the linear map of moving left. Show that $(\mathrm{id}+L)$ is invertible;\\
   3. Show that $\varphi := (\mathrm{id}+L)^{-1}$ has no adjoint $\varphi^\star$.
\begin{aaa}
    1. 令$\phi^*(x)=0$, 那么 $(\phi^*(x),y) = (x , \phi(y)) =0 $ 对任意的$y$都成立, 所以$x=0$\\令$y\in (\op{im(\phi^*)})^{\perp}$ , 那么对任意$x$ , $(下, \phi(y))=0$, $\phi(y) = 0 $, $y = 0$.\\ 
    2. $\phi^*$ bij, $\exists z, \phi(z)=x$, $(\phi^*((\phi^{-1})^{*}(x)),y)=((\phi^{-1})^{*}(x),\phi(y))=(x,y)$ , 因此 $(\varphi ^{-1})^\star = (\varphi^\star)^{-1}$\\
    3. $(\phi^*((\phi^{-1})^{*}(x)),y)=((\phi^{-1})^{*}(x),\phi(y))=(x,y)$ , 因此 $(\varphi ^{-1})^\star = (\varphi^\star)^{-1}$\\
    4.1. 双线性, 对称, 正定\\
    4.2. $(id+L)(f)=\sum_{i=0}^{n-1} (f_i+f_{i+1})x^i+f_nx^n$, 显然可逆.\\
    4.3. $(id+L)^{-1}(f)=\sum_{i=0}^{n-1}(\sum_{j=i}^n(-1)^{j-1}f_j)x^i+f_nx^n$\\
    假设$\phi=(id+L)^{-1}$存在$\phi^*$, 那么取$f=1$, $(\phi^*(1),g)=(1,\phi(g))=g_0-g_1+g_2-\cdots$,取不同的$g$, 得到的$\phi^*(1)$不同, 矛盾.
\end{aaa}
\end{document}