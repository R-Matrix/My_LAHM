\documentclass[11pt]{ctexart}
\usepackage[margin=2cm,a4paper]{geometry}
\usepackage{amsthm, amsfonts, amsmath, amssymb, mathrsfs, newclude, tikz-cd, tikz, ctex, mathtools, stmaryrd, datetime}


%\setmainfont{Caladea}

%% 也可以选用其它字库:
% \setCJKmainfont[%
%   ItalicFont=AR PL KaitiM GB,
%   BoldFont=Noto Sans CJK SC,
% ]{Noto Serif CJK SC}
% \setCJKsansfont{Noto Sans CJK SC}
% \renewcommand{\kaishu}{\CJKfontspec{AR PL KaitiM GB}}



\usepackage[colorlinks = true,
linkcolor = blue,
urlcolor  = blue,
citecolor = blue,
anchorcolor = blue]{hyperref}

% Include the x-color package for color support
\usepackage{xcolor}

% Define a new environment for red comments
\usepackage{verbatim} % Required for the comment environment
\usepackage{environ}

\usepackage{mdframed} % Include mdframed for creating framed environments

\definecolor{pinked}{RGB}{255,231,229} % Define a base color 
% Define a new environment with a background color
\newmdenv[
  backgroundcolor=pinked, % Set the desired background color
  linecolor=white, % Optional: Set the border line color
  linewidth=1pt, % Optional: Set the border line width
  roundcorner=5pt, % Optional: Set rounded corners
  nobreak=true % Optional: Prevent page breaks within the environment
]{pinked}

\theoremstyle{definition}
\newtheorem{qqq}{问题}[section]

\newcommand{\ExternalLink}{%
    \tikz[x=1.2ex, y=1.2ex, baseline=-0.05ex]{% 
        \begin{scope}[x=1ex, y=1ex]
            \clip (-0.1,-0.1) 
                --++ (-0, 1.2) 
                --++ (0.6, 0) 
                --++ (0, -0.6) 
                --++ (0.6, 0) 
                --++ (0, -1);
            \path[draw, 
                line width = 0.5, 
                rounded corners=0.5] 
                (0,0) rectangle (1,1);
        \end{scope}
        \path[draw, line width = 0.5] (0.5, 0.5) 
            -- (1, 1);
        \path[draw, line width = 0.5] (0.6, 1) 
            -- (1, 1) -- (1, 0.6);
        }
    }

\NewEnviron{aaa}{~\\
    \noindent {\textcolor{teal}{\textbf{解答}} \BODY }
}

\NewEnviron{llll}{
    \noindent {~\\$\ExternalLink$ 外部链接 $\,\,\,$ \color{blue}\url{\BODY} }
}

\renewcommand{\proofname}{证明}
\renewcommand\qedsymbol{${\boxed{\substack{\textit{完证}\\\textit{毕明}}}}$}


% Define a custom command for \kuing
\newcommand{\kuing}{\texorpdfstring{$\textstyle{\int_u^c k}=\texttt{kuing}$}{}}



% Change equation numbering to include the section number
\usepackage{cleveref}
\renewcommand{\theequation}{\thesection.\thesubsection.\arabic{equation}}
\numberwithin{equation}{section}


%我可以在这里定义一些简化的命令一图便利
\newcommand{\sspan}{\operatorname{span}}%定义span正体
\newcommand{\op}[1]{\operatorname{#1}}%简化函数表达
\newcommand{\cnm}[2]{\binom{#1}{#2} }%简化组合数命令
\newcommand{\set}[2]{\{ #1 \mid #2\}}%定义集合表示
\newcommand{\FF}{\mathbb{F}}%数域
\newcommand{\RR}{\mathbb{R}}
\newcommand{\CC}{\mathbb{C}}
\newcommand{\QQ}{\mathbb{Q}}


%%注意:  行内公式强制行间形式用"  \limits  ",反之用  "  \nolimits   ";
%%注意:  公式引用编号 :" \label{}  ";   公式显示编号" \tag{}  ":  引用公式"  \eqref{}    "


%截止到这里是我定义的

\usepackage{listings}
% Define listings style
\lstset{
  frame=tb,
  language=TeX,
  aboveskip=3mm,
  belowskip=3mm,
  showstringspaces=false,
  columns=flexible,
  basicstyle={\small\ttfamily},
  numbers=none,
  breaklines=true,
  breakatwhitespace=true,
  tabsize=3
}

\theoremstyle{definition}
\newtheorem*{definition}{定义}
\newtheorem*{proposition}{命题}
\newtheorem*{theorem}{定理}
\newtheorem*{notation}{记号}
\newtheorem*{example}{例子}
\newtheorem*{exercise}{习题}
\theoremstyle{remark}
\newtheorem*{remark}{备注}
\newtheorem*{lemma}{引理}
\newtheorem*{corollary}{推论}



\title{十五周作业}
\author{董仕强}

\setcounter{section}{-1}

\setcounter{page}{0}

\setlength\parindent{0pt}

\begin{document}

\maketitle

\section{说明}

可以将作业中遇到的问题标注在此. 如有, 请补充.

\tableofcontents

\newpage

%%%%%%%%%%%%%%%%%%%%%%%%%%%%%%%%%%%%%%%%%%%%%%
%%%%%%%%%%%%%%%%%%%%%%%%%%%%%%%%%%%%%%%%%%%%%%
%%%%%%%%%%%%%%%%%%%%%%%%%%%%%%%%%%%%%%%%%%%%%%
%%%%%%%%%%%%%%%%%%%%%%%%%%%%%%%%%%%%%%%%%%%%%%
%%%%%%%%%%%%%%%%%%%%%%%%%%%%%%%%%%%%%%%%%%%%%%
%%%%%%%%%%%%% 请从此处开始阅读 %%%%%%%%%%%%%%%%

\section{Problem Set for 26-May and 29-May}
\subsection{Exercise 0}
Show that $U \otimes V \simeq V \otimes U$ with the following steps.\\

1. Construct $\varphi : U \otimes V \to V \otimes U$ via a 1:1 correspondence (refer to the preceding assignment). Describe $\varphi$ by tracing the image of simple tensors.\\
2. Define $\psi : V \otimes U \to U \otimes V$ analogously.\\
3. Establish via 1:1 correspondence that both $\psi \circ \varphi$ and $\varphi \circ \psi$ are identical mappings.\\
\begin{aaa}
    定义双线性映射$\Phi : U\& V \to V\otimes U,  (u,v)\mapsto v\otimes u$, 这唯一确定了线性映射 $\varphi: U\otimes V \to V\otimes U$ 和 $\psi : V \otimes U \to U \otimes V$, 那么 $\varphi \circ \psi = I$ 且 $\psi \circ \varphi = I$ , 因此 $\varphi, \psi$ 是双射, 于是 $U\otimes V \simeq V\otimes U$.
\end{aaa}

\subsection{Exercise 1}
Show that the following is an isomorphism:
$$
\mathrm{Hom}(U, V^*) \simeq \mathrm{Hom}(V,U^\ast),\quad [u \mapsto f_u] \mapsto [v \mapsto [u \mapsto f_u(v)]].
$$
\begin{aaa}
    $\op{Hom}(U,V^*)\simeq (V\otimes U)^*$ , $\op{Hom}(V,U^*)\simeq (U\otimes V)^*$ , 且 $U\otimes V \simeq V\otimes U$
\end{aaa}

\subsection{Exercise 2}
Let $V$ be a finite-dimensional vector space with basis $\{e_i\}$. Let $\{f_i\}$ denote the dual basis of $V^\ast$. Define the following mappings:\\

- $\Delta : \mathbb{F} \to V \otimes V^\ast, \quad 1 \mapsto \sum_{i=1}^n e_i \otimes f_i$;\\
- $\nabla : V \otimes V^\ast \to \mathbb{F},\quad \sum u_i \otimes \varphi_i \mapsto \sum \varphi_i(u_i)$.\\

For $f, g : V \to V$, determine the composition
$$
\mathbb{F} \xrightarrow{\Delta} V \otimes V^\ast \xrightarrow{(f \otimes g^\ast)} V \otimes V^\ast \xrightarrow{\nabla} \mathbb{F}.
$$
\begin{aaa}
    $1 \mapsto \sum e_i\otimes f_i\mapsto f(e_i)\otimes f_ig\mapsto \sum f_igfe_i=tr(gf)$
\end{aaa}

\subsection{Exercise 3}
Demonstrate that, for finite-dimensional vector spaces, there exists an isomorphism determined via simple tensors:
$$
\Phi: V_1^\ast \otimes V_2^\ast \otimes V_3 \xrightarrow{\sim} \mathrm{Hom}(V_1 \otimes V_2, V_3),\quad f \otimes g \otimes x \mapsto [a \otimes b \mapsto f(a)g(b)x].
$$

Now take $\otimes = \otimes_{\mathbb{R}}$ and $V_i = \mathbb{C}$ (the two-dimensional vector space with basis $\{1, i\}$). We know that the usual multiplication defines a map in $\mathrm{Hom}_{\mathbb{R}}(\mathbb{C} \otimes \mathbb{C}, \mathbb{C})$:
$$
\times : \mathbb{C} \otimes_{\mathbb{R}} \mathbb{C} \to \mathbb{C},\quad w \otimes z \mapsto w \cdot z.
$$

What is $\Phi^{-1}(\times)$?
\begin{aaa}
    考虑基的变换, $(1\otimes 1, 1\otimes i, i\otimes 1, i\otimes i)\mapsto (1,i,i,-1)$, 因此令$1,i$的对偶基为$f,g$,$\Phi^{-1}(\times)=f\otimes f\otimes 1 +f\otimes g\otimes i + g\otimes f \otimes i + g\otimes g\otimes (-1)$
\end{aaa}

\subsection{Exercise 5}
Let $A \in \mathbb{C}^{m \times m}$ and $B \in \mathbb{C}^{n \times n}$ be normal matrices (i.e., $PP^H = P^HP$). Show that $AX = XB$ if and only if $A^H X = X B^H$.
\begin{aaa}
    $\Rightarrow$: $AX=XB$ , 得到 $A^HAXB^H=A^HXBB^H$, 因此 $A(A^HXB^H)=(A^HXB^H)B$, \\
    $X\in \ker(A\otimes I-I\otimes B^T)$, we want to show $X\in \ker(A^H\otimes I - I \otimes \bar{B})$, \\
    $A\otimes I-I\otimes B^T$也是正规矩阵,$(A\otimes I-I\otimes B^T)^H=P^H=A^H\otimes I - I \otimes \bar{B}$\\$\ker (P)=\ker (PP^H) = \ker (P^HP) =\ker (P^H)$
\end{aaa}
\subsection{Exercise 6}
Let $A, B \in \mathbb{C}^{n \times n}$ be Hermitian positive-definite matrices, i.e., $A = A^H$ and $u^H \cdot A \cdot u \geq 0$ with equality if and only if $u = \mathbf{0}$. Define the matrix $C$ via component-wise multiplication (the stupid multiplication):
$$
C = (c_{i,j}),\quad c_{i,j} = a_{i,j} \cdot b_{i,j}.
$$
Show that $C$ is also Hermitian positive-definite 
\begin{aaa}
    $A=P^HP, \; B=Q^HQ$, $A\otimes B=(P^HP)\otimes (Q^HQ)=(P^H\otimes Q^H)(PQ)=(P\otimes Q)^H(P\otimes Q)$.\\
    因此$A\otimes B$ 厄尔米特正定, $C$是$A\otimes B$的一个主子式,因此也正定.
\end{aaa}
\end{document}