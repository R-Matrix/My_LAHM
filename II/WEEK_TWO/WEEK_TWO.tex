\documentclass[11pt]{ctexart}
\usepackage[margin=2cm,a4paper]{geometry}
\usepackage{amsthm, amsfonts, amsmath, amssymb, mathrsfs, newclude, tikz-cd, tikz, ctex, mathtools, stmaryrd, datetime}


%\setmainfont{Caladea}

%% 也可以选用其它字库:
% \setCJKmainfont[%
%   ItalicFont=AR PL KaitiM GB,
%   BoldFont=Noto Sans CJK SC,
% ]{Noto Serif CJK SC}
% \setCJKsansfont{Noto Sans CJK SC}
% \renewcommand{\kaishu}{\CJKfontspec{AR PL KaitiM GB}}



\usepackage[colorlinks = true,
linkcolor = blue,
urlcolor  = blue,
citecolor = blue,
anchorcolor = blue]{hyperref}

% Include the x-color package for color support
\usepackage{xcolor}

% Define a new environment for red comments
\usepackage{verbatim} % Required for the comment environment
\usepackage{environ}

\usepackage{mdframed} % Include mdframed for creating framed environments

\definecolor{pinked}{RGB}{255,231,229} % Define a base color 
% Define a new environment with a background color
\newmdenv[
  backgroundcolor=pinked, % Set the desired background color
  linecolor=white, % Optional: Set the border line color
  linewidth=1pt, % Optional: Set the border line width
  roundcorner=5pt, % Optional: Set rounded corners
  nobreak=true % Optional: Prevent page breaks within the environment
]{pinked}

\theoremstyle{definition}
\newtheorem{qqq}{问题}[section]

\newcommand{\ExternalLink}{%
    \tikz[x=1.2ex, y=1.2ex, baseline=-0.05ex]{% 
        \begin{scope}[x=1ex, y=1ex]
            \clip (-0.1,-0.1) 
                --++ (-0, 1.2) 
                --++ (0.6, 0) 
                --++ (0, -0.6) 
                --++ (0.6, 0) 
                --++ (0, -1);
            \path[draw, 
                line width = 0.5, 
                rounded corners=0.5] 
                (0,0) rectangle (1,1);
        \end{scope}
        \path[draw, line width = 0.5] (0.5, 0.5) 
            -- (1, 1);
        \path[draw, line width = 0.5] (0.6, 1) 
            -- (1, 1) -- (1, 0.6);
        }
    }

\NewEnviron{aaa}{~\\
    \noindent {\textcolor{teal}{\textbf{解答}} \BODY }
}

\NewEnviron{llll}{
    \noindent {~\\$\ExternalLink$ 外部链接 $\,\,\,$ \color{blue}\url{\BODY} }
}

\renewcommand{\proofname}{证明}
\renewcommand\qedsymbol{${\boxed{\substack{\textit{完证}\\\textit{毕明}}}}$}


% Define a custom command for \kuing
\newcommand{\kuing}{\texorpdfstring{$\textstyle{\int_u^c k}=\texttt{kuing}$}{}}



% Change equation numbering to include the section number
\usepackage{cleveref}
\renewcommand{\theequation}{\thesection.\thesubsection.\arabic{equation}}
\numberwithin{equation}{section}


%我可以在这里定义一些简化的命令一图便利
\newcommand{\sspan}{\operatorname{span}}%定义span正体
\newcommand{\op}[1]{\operatorname{#1}}%简化函数表达
\newcommand{\cnm}[2]{\binom{#1}{#2} }%简化组合数命令
\newcommand{\set}[2]{\{ #1 \mid #2\}}%定义集合表示
\newcommand{\FF}{\mathbb{F}}%数域
\newcommand{\RR}{\mathbb{R}}
\newcommand{\CC}{\mathbb{C}}
\newcommand{\QQ}{\mathbb{Q}}
\newcommand{\ZZ}{\mathbb{Z}}


%%注意:  行内公式强制行间形式用"  \limits  ",反之用  "  \nolimits   ";
%%注意:  公式引用编号 :" \label{}  ";   公式显示编号" \tag{}  ":  引用公式"  \eqref{}    "


%截止到这里是我定义的

\usepackage{listings}
% Define listings style
\lstset{
  frame=tb,
  language=TeX,
  aboveskip=3mm,
  belowskip=3mm,
  showstringspaces=false,
  columns=flexible,
  basicstyle={\small\ttfamily},
  numbers=none,
  breaklines=true,
  breakatwhitespace=true,
  tabsize=3
}

\theoremstyle{definition}
\newtheorem*{definition}{定义}
\newtheorem*{proposition}{命题}
\newtheorem*{theorem}{定理}
\newtheorem*{notation}{记号}
\newtheorem*{example}{例子}
\newtheorem*{exercise}{习题}
\theoremstyle{remark}
\newtheorem*{remark}{备注}
\newtheorem*{lemma}{引理}
\newtheorem*{corollary}{推论}



\title{第二周作业}
\author{董仕强}

\setcounter{section}{-1}

\setcounter{page}{0}

\setlength\parindent{0pt}

\begin{document}

\maketitle

\section{说明}

可以将作业中遇到的问题标注在此. 如有, 请补充.

\tableofcontents

\newpage

%%%%%%%%%%%%%%%%%%%%%%%%%%%%%%%%%%%%%%%%%%%%%%
%%%%%%%%%%%%%%%%%%%%%%%%%%%%%%%%%%%%%%%%%%%%%%
%%%%%%%%%%%%%%%%%%%%%%%%%%%%%%%%%%%%%%%%%%%%%%
%%%%%%%%%%%%%%%%%%%%%%%%%%%%%%%%%%%%%%%%%%%%%%
%%%%%%%%%%%%%%%%%%%%%%%%%%%%%%%%%%%%%%%%%%%%%%
%%%%%%%%%%%%% 请从此处开始阅读 %%%%%%%%%%%%%%%%

\section{Problem Set for 24-Feb-2025}
\subsection{Problem 1}
Let $\FF$ be any field , and let $\{u_i\}_{i=1}^n$ and $\{v_j\}_{j=1}^m$  be bases of $U$ and $V$,
respectively . Define $\op{Hom}_{\FF}(U,V)$ as the set of $\FF$ -linear maps from $U$ to $V$ .
\begin{enumerate}
    \item Endow $\op{Hom}_{\FF}(U,V)$ with the structure of a vector space over $\FF$.
    \item Determine the dimension of $\op{Hom}_{\FF}(U,V)$ and provide a basis for it.
\end{enumerate}
\begin{aaa}
    \begin{enumerate}
        \item 由于$\op{Hom}_{\FF}(U,V)$是从$U$到$V$的线性映射的集合, 因此我们可以定义加法和数乘如下:
        \begin{align*}
            (f+g)(u) &= f(u) + g(u)\\
            (\lambda f)(u) &= \lambda f(u)
        \end{align*}
        其中$f,g\in \op{Hom}_{\FF}(U,V)$, $\lambda\in \FF$, $u\in U$.
        \item 由于$\{u_i\}_{i=1}^n$和$\{v_j\}_{j=1}^m$分别是$U$和$V$的基, 因此我们可以定义一个线性映射$f_{ij}$如下:
        \begin{align*}
            f_{ij}(u_k) = v_j\delta_{ik}
        \end{align*}
        显然, $\{f_{ij}\}_{i=1,j=1}^{n,m}$是$\op{Hom}_{\FF}(U,V)$的一个基. 由此我们可以得到$\op{Hom}_{\FF}(U,V)$的维数为$nm$.
    \end{enumerate}
\end{aaa}
\subsection{Problem 2}
(Blank-Filling Question)Throughout , $\FF$ is an arbitrary field, and $\{u_i\}_{i=1}^l$ , $\{b_j\}_{j=1}^m$ , and $\{c_k\}_{k=1}^n$ are bases of $U , V$ and $W$ , respectly.
Carefully endow the following spaces with $\FF$ -linear structures. and write down their dimensions along with the corrseponding distinguished bases.
\begin{enumerate}
    \item $U \oplus V$ as subspace.
    \item $U \times V$ as the usual Cartesian product.
    \item $\op{Hom}_{\FF}(U,V)$ as the set of linear maps.
    \item $\op{Hom}_{\op{Sets}}(U,V)$ as the set of maps.
    \item $\op{Hom}_{\FF}(U\times V, W)$, where $U \times V$ is defined a priori.
    \item $\op{Bil}_{\FF}(U,VlW)$, the set of bilinear maps from $U $ and $V$ to $W$.
    \item $\op{Hom}_{\FF}(U,\op{Hom}_{\FF}(V,W))$,also known as the  currying of bilinear maps.
\end{enumerate} 
\begin{aaa}
    \begin{enumerate}
        \item $U\oplus V$是$U$和$V$的直和,\[(u_1,v_1)+(u_2,v_2)=(u_1+u_2,v_1+v_2),\qquad \lambda(u_1,v_1)=(\lambda u_1,\lambda v_1)\]维数为$l+m$, 基为$\{u_i\}_{i=1}^l\cup\{v_j\}_{j=1}^m$.
        \item $U\times V$是$U$和$V$的笛卡尔积, 维数为$l+m$, 基为$\{(u_i,0)\}_{i=1}^l\cup\{(0,v_j)\}_{j=1}^m$.
        \item $\op{Hom}_{\FF}(U,V)$是从$U$到$V$的线性映射的集合, 维数为$lm$, 基为$\{f_{ij}\}_{i=1,j=1}^{l,m}$.
        \item $\op{Hom}_{\op{Sets}}(V,W)$是从$U$到$V$的映射的集合, 维数为$m^l$, 基为$\{g_{i}\}_{i=1}^{l}$.
        \item $\op{Hom}_{\FF}(U\times V, W)$是从$U\times V$到$W$的线性映射的集合, 维数为$(l+m)n$, 基为$\{h_{ijk}\}_{i=1,k=1}^{l,n} \cup \{h_{ijk}\}_{j=1,k=1}^{m,n}$.
        \item $\op{Bil}_{\FF }(U,V;W)$是从$U$和$V$到$W$的双线性映射的集合, 维数为$lmn$, 基为$\{f_{ijk}\}_{i=1,j=1,k=1}^{l,m,n}$.
        \item $\op{Hom}_{\FF}(U,\op{Hom}_{\FF}(V,W))$是从$U$到$\op{Hom}_{\FF}(V,W)$的线性映射的集合, 维数为$lmn$, 基为$\{g_{ijk}\}_{i=1,j=1,k=1}^{l,m,n}$.
    \end{enumerate}
\end{aaa}
\section{Problem Set for 24-Feb-2025}
\subsection{Exercise 1.1}
Let $P_n:= \{f(x) \in \FF[x] \mid \op{deg} f(x) <n\}$. Pick $a_1,a_2,\ldots ,a_n \in \FF$ such that $a_i \neq a_j$ for any $i\neq j$. Show that \[f_j(x):= \prod_{i\neq j}(x-a_i)(1 \leq i \leq n)\]from a basis of $P_n$.
\begin{aaa}
    Lagrange 插值公式告诉我们, 对于任意的$f(x)\in P_n$, 我们有\[f(x) = \sum_{i=1}^n f(a_i)\prod_{j\neq i}\frac{x-a_j}{a_i-a_j}=\sum_{i=1}^{n}\frac{f(a_i)f_j(x)}{\prod_{j\neq i}(a_i-a_j)}\]
\end{aaa}
\subsection{Exercise 4.1}
Assume $f(x)=x^3 +px+q \in \ZZ[x]$ is irreducible and $\alpha \in \CC $ is a root of $f$.
\begin{enumerate}
    \item Show that $\QQ[\alpha]:=\{g(\alpha)\mid g(x)\in \QQ[x]\}$ is a linear space over $\QQ$ and $1,\alpha,\alpha^2$ from a basis.
    \item Prove that $\phi:\beta \mapsto f'(\alpha)\beta$ gives a linear map on $\QQ[\alpha]$ and find its matrix under $1,\alpha,\alpha^2$.
\end{enumerate}
\begin{aaa}
    \begin{enumerate}
        \item $0 \in \QQ[\alpha],(f+g)(\alpha)=f(\alpha)+g(\alpha),(\lambda f)(\alpha)=\lambda f(\alpha)$\\For $\alpha^n(n \geq 3)$,利用等式$\alpha^3=-p\alpha-q$反复代换可以表示成$1,\alpha,\alpha^2$的线性组合.\\
        \item 由于$\phi(\beta)=\beta f'(\alpha)$, 因此我们有\[\phi(1)=f'(\alpha),\quad \phi(\alpha)=\alpha f'(\alpha),\quad \phi(\alpha^2)=\alpha^2 f'(\alpha)\]因此我们有\[\begin{pmatrix}f'(\alpha)\\ \alpha f'(\alpha)\\ \alpha^2 f'(\alpha)\end{pmatrix}=\begin{pmatrix}0&1&0\\0&0&1\\-q&-p&-3\end{pmatrix}\begin{pmatrix}1\\ \alpha\\ \alpha^2\end{pmatrix}\]
    \end{enumerate}
\end{aaa}
\subsection{Exercise 4.3}
Prove that if $f,g\in \ZZ[x]$ are primitive, then so is $fg$.
\begin{aaa}
    Let $f(x)=\sum_{i=0}^n a_ix^i$, $g(x)=\sum_{i=0}^m b_ix^i$, then we have $fg(x)=\sum_{i=0}^{n+m}c_ix^i$, where $c_i=\sum_{j=0}^i a_jb_{i-j}$. \\
    Suppose $fg$ is not primitive, then there exists a prime $p$ such that $p|c_i$ for all $i$. Let $a_i$ is the smallest index such that $p\nmid a_i$, $b_j$ is the smallest index such that $p\nmid b_j$, so $c_{i+j}$ is not divide by $p$.
\end{aaa}
\end{document}