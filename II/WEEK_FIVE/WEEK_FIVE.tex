\documentclass[11pt]{ctexart}
\usepackage[margin=2cm,a4paper]{geometry}
\usepackage{amsthm, amsfonts, amsmath, amssymb, mathrsfs, newclude, tikz-cd, tikz, ctex, mathtools, stmaryrd, datetime}


%\setmainfont{Caladea}

%% 也可以选用其它字库:
% \setCJKmainfont[%
%   ItalicFont=AR PL KaitiM GB,
%   BoldFont=Noto Sans CJK SC,
% ]{Noto Serif CJK SC}
% \setCJKsansfont{Noto Sans CJK SC}
% \renewcommand{\kaishu}{\CJKfontspec{AR PL KaitiM GB}}



\usepackage[colorlinks = true,
linkcolor = blue,
urlcolor  = blue,
citecolor = blue,
anchorcolor = blue]{hyperref}

% Include the x-color package for color support
\usepackage{xcolor}

% Define a new environment for red comments
\usepackage{verbatim} % Required for the comment environment
\usepackage{environ}

\usepackage{mdframed} % Include mdframed for creating framed environments

\definecolor{pinked}{RGB}{255,231,229} % Define a base color 
% Define a new environment with a background color
\newmdenv[
  backgroundcolor=pinked, % Set the desired background color
  linecolor=white, % Optional: Set the border line color
  linewidth=1pt, % Optional: Set the border line width
  roundcorner=5pt, % Optional: Set rounded corners
  nobreak=true % Optional: Prevent page breaks within the environment
]{pinked}

\theoremstyle{definition}
\newtheorem{qqq}{问题}[section]

\newcommand{\ExternalLink}{%
    \tikz[x=1.2ex, y=1.2ex, baseline=-0.05ex]{% 
        \begin{scope}[x=1ex, y=1ex]
            \clip (-0.1,-0.1) 
                --++ (-0, 1.2) 
                --++ (0.6, 0) 
                --++ (0, -0.6) 
                --++ (0.6, 0) 
                --++ (0, -1);
            \path[draw, 
                line width = 0.5, 
                rounded corners=0.5] 
                (0,0) rectangle (1,1);
        \end{scope}
        \path[draw, line width = 0.5] (0.5, 0.5) 
            -- (1, 1);
        \path[draw, line width = 0.5] (0.6, 1) 
            -- (1, 1) -- (1, 0.6);
        }
    }

\NewEnviron{aaa}{~\\
    \noindent {\textcolor{teal}{\textbf{解答}} \BODY }
}

\NewEnviron{llll}{
    \noindent {~\\$\ExternalLink$ 外部链接 $\,\,\,$ \color{blue}\url{\BODY} }
}

\renewcommand{\proofname}{证明}
\renewcommand\qedsymbol{${\boxed{\substack{\textit{完证}\\\textit{毕明}}}}$}


% Define a custom command for \kuing
\newcommand{\kuing}{\texorpdfstring{$\textstyle{\int_u^c k}=\texttt{kuing}$}{}}



% Change equation numbering to include the section number
\usepackage{cleveref}
\renewcommand{\theequation}{\thesection.\thesubsection.\arabic{equation}}
\numberwithin{equation}{section}


%我可以在这里定义一些简化的命令一图便利
\newcommand{\sspan}{\operatorname{span}}%定义span正体
\newcommand{\op}[1]{\operatorname{#1}}%简化函数表达
\newcommand{\cnm}[2]{\binom{#1}{#2} }%简化组合数命令
\newcommand{\set}[2]{\{ #1 \mid #2\}}%定义集合表示
\newcommand{\FF}{\mathbb{F}}%数域
\newcommand{\RR}{\mathbb{R}}
\newcommand{\CC}{\mathbb{C}}
\newcommand{\QQ}{\mathbb{Q}}


%%注意:  行内公式强制行间形式用"  \limits  ",反之用  "  \nolimits   ";
%%注意:  公式引用编号 :" \label{}  ";   公式显示编号" \tag{}  ":  引用公式"  \eqref{}    "


%截止到这里是我定义的

\usepackage{listings}
% Define listings style
\lstset{
  frame=tb,
  language=TeX,
  aboveskip=3mm,
  belowskip=3mm,
  showstringspaces=false,
  columns=flexible,
  basicstyle={\small\ttfamily},
  numbers=none,
  breaklines=true,
  breakatwhitespace=true,
  tabsize=3
}

\theoremstyle{definition}
\newtheorem*{definition}{定义}
\newtheorem*{proposition}{命题}
\newtheorem*{theorem}{定理}
\newtheorem*{notation}{记号}
\newtheorem*{example}{例子}
\newtheorem*{exercise}{习题}
\theoremstyle{remark}
\newtheorem*{remark}{备注}
\newtheorem*{lemma}{引理}
\newtheorem*{corollary}{推论}



\title{第五周作业}
\author{董仕强}

\setcounter{section}{-1}

\setcounter{page}{0}

\setlength\parindent{0pt}

\begin{document}

\maketitle

\section{说明}

可以将作业中遇到的问题标注在此. 如有, 请补充.

\tableofcontents

\newpage

%%%%%%%%%%%%%%%%%%%%%%%%%%%%%%%%%%%%%%%%%%%%%%
%%%%%%%%%%%%%%%%%%%%%%%%%%%%%%%%%%%%%%%%%%%%%%
%%%%%%%%%%%%%%%%%%%%%%%%%%%%%%%%%%%%%%%%%%%%%%
%%%%%%%%%%%%%%%%%%%%%%%%%%%%%%%%%%%%%%%%%%%%%%
%%%%%%%%%%%%%%%%%%%%%%%%%%%%%%%%%%%%%%%%%%%%%%
%%%%%%%%%%%%% 请从此处开始阅读 %%%%%%%%%%%%%%%%

\section{Problem Set for 17 March 2025}
\subsection{Exercise}
Prove the following(Whenever we write $f_*X$, it is assumed that the linear space X is a subspace of the domain, and similarly for $f^*X$):
\begin{enumerate}
    \item $U\subset f^*f_*U$, when does equality hold for all U?
    \item $f_*f^*V\subset V$, when does equality hold for all V?
    \item $f_*f^*f_*=f_*$.
    \item $f^*f_*f^*=f^*$.
    \item $f_*(U+V)=f_*U+f_*V$.
    \item $f^*(U\cap V)=f^*U\cap f^*V$.
    \item Explain $f_*(U\cap V)\subset f_*U\cap f_*V$.
    \item Explain $f^*(U+V)\supset f^*U+f^*V$.
\end{enumerate}
\begin{aaa}
    \begin{enumerate}
        \item $f$单射
        \item 等号能取到吧
        \item $\forall v$, suppose $f(U)=v$, then $f_*(f^*v)=f_*U=v$.
        \item $\forall v$, suppose $f^*(v)=U$, then $f^*(f_*u)=f^*v=U$.
        \item $f(u+v)=f(u)+f(v)\in f_*U+f_*V$, so $f_*(U+V)\subset f_*U+f_*V$.\\$\forall f(u)\in f_*U$ and $f(v) \in f_*V$, $f(u)+f(v)=f(u+v) \in f_*(U+V)$. So $f_*U+f_*V\subset f_*(U+V)$.
        \item $\forall x \in U\cap V$, $f^*(x)=f^*(x)\in f^*U\cap f^*V$. So $f^*(U\cap V)\subset f^*U\cap f^*V$.\\$\forall x\in f^*U\cap f^*V$,$f(x)\in U\cap V$,$x\in f^*(U\cap V)$.so $f^*U\cap f^*V\subset f^*(U\cap V)$.
        \item $\forall x\in U\cap V$, $f_*(x)=f_*(x)\in f_*U\cap f_*V$. So $f_*(U\cap V)\subset f_*U\cap f_*V$.
        \item $\forall x\in f^*U+f^*V$, $x=f^*(u)+f^*(v)=f^*(u+v)\in f^*(U+V)$. So $f^*U+f^*V\subset f^*(U+V)$.
    \end{enumerate}
\end{aaa}
\subsection{Problem}
\begin{enumerate}
    \item Show that $\op{Hom}_{\FF}(\FF[x],\FF)\cong \FF[[x]]$.
    \item Show that $\op{Hom}_{\FF}(\FF[[x]],\FF)$ has a subspace which is iso to $\FF[x]$.
\end{enumerate}
\begin{aaa}
    \begin{enumerate}
        \item $(\op{Hom}_{\FF}(\FF[x],\FF),+,\cdot)$ is a linear space over $\FF$.we can find a basis $\{\varphi(x^k)\}_{k \geq 1}$\\let $\sigma:\op{Hom}_{\FF}(\FF[x],\FF)\to \FF[[x]],f \mapsto g(x)$ is a linear map.\\let $\sigma(\varphi(x^k))=\sum_{n=1}^\infty \delta_{kn}x^n$\\let $\sigma(f)=\sigma(\sum_{k=0}^{\infty}a_k\varphi(x^k))=\sum_{k=0}^{\infty}a_kx^k=0\Rightarrow a_k=0 \Rightarrow f=0$\\and $\sigma$ is surjective obveriously.\\so $\sigma$ is an isomorphism.
        \item $\op{Hom}_{\FF}(\FF[[x]],\FF)$ is a linear space over $\FF$.\\let $U=\{\sum_{k=0}^{\infty}a_k\varphi(x^k)|a_k=0,k\geq 1\}$\\let $\sigma:U\to \FF[x],f \mapsto g(x)$ is a linear map.\\let $\sigma(\sum_{k=0}^{\infty}a_k\varphi(x^k))=\sum_{k=0}^{\infty}a_kx^k$\\let $\sigma(f)=\sigma(\sum_{k=0}^{\infty}a_k\varphi(x^k))=\sum_{k=0}^{\infty}a_kx^k=0\Rightarrow a_k=0 \Rightarrow f=0$\\and $\sigma$ is surjective obveriously.\\so $\sigma$ is an isomorphism.
    \end{enumerate}
\end{aaa}
\end{document}