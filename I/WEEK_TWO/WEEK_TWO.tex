\documentclass[11pt]{ctexart}
\usepackage[margin=2cm,a4paper]{geometry}
\usepackage{amsthm, amsfonts, amsmath, amssymb, mathrsfs, newclude, tikz-cd, tikz, ctex, mathtools, stmaryrd, datetime}


%\setmainfont{Caladea}

%% 也可以选用其它字库:
% \setCJKmainfont[%
%   ItalicFont=AR PL KaitiM GB,
%   BoldFont=Noto Sans CJK SC,
% ]{Noto Serif CJK SC}
% \setCJKsansfont{Noto Sans CJK SC}
% \renewcommand{\kaishu}{\CJKfontspec{AR PL KaitiM GB}}



\usepackage[colorlinks = true,
linkcolor = blue,
urlcolor  = blue,
citecolor = blue,
anchorcolor = blue]{hyperref}

% Include the x-color package for color support
\usepackage{xcolor}

% Define a new environment for red comments
\usepackage{verbatim} % Required for the comment environment
\usepackage{environ}

\usepackage{mdframed} % Include mdframed for creating framed environments

\definecolor{pinked}{RGB}{255,231,229} % Define a base color 
% Define a new environment with a background color
\newmdenv[
  backgroundcolor=pinked, % Set the desired background color
  linecolor=white, % Optional: Set the border line color
  linewidth=1pt, % Optional: Set the border line width
  roundcorner=5pt, % Optional: Set rounded corners
  nobreak=true % Optional: Prevent page breaks within the environment
]{pinked}

\theoremstyle{definition}
\newtheorem{qqq}{问题}[section]

\newcommand{\ExternalLink}{%
    \tikz[x=1.2ex, y=1.2ex, baseline=-0.05ex]{% 
        \begin{scope}[x=1ex, y=1ex]
            \clip (-0.1,-0.1) 
                --++ (-0, 1.2) 
                --++ (0.6, 0) 
                --++ (0, -0.6) 
                --++ (0.6, 0) 
                --++ (0, -1);
            \path[draw, 
                line width = 0.5, 
                rounded corners=0.5] 
                (0,0) rectangle (1,1);
        \end{scope}
        \path[draw, line width = 0.5] (0.5, 0.5) 
            -- (1, 1);
        \path[draw, line width = 0.5] (0.6, 1) 
            -- (1, 1) -- (1, 0.6);
        }
    }

\NewEnviron{aaa}{~\\
    \noindent {\textcolor{teal}{\textbf{解答}} \BODY }
}

\NewEnviron{llll}{
    \noindent {~\\$\ExternalLink$ 外部链接 $\,\,\,$ \color{blue}\url{\BODY} }
}

\renewcommand{\proofname}{证明}
\renewcommand\qedsymbol{${\boxed{\substack{\textit{完证}\\\textit{毕明}}}}$}


% Define a custom command for \kuing
\newcommand{\kuing}{\texorpdfstring{$\textstyle{\int_u^c k}=\texttt{kuing}$}{}}

% Change equation numbering to include the section number
\usepackage{cleveref}
\renewcommand{\theequation}{\thesection.\thesubsection.\arabic{equation}}
\numberwithin{equation}{section}

\usepackage{listings}
% Define listings style
\lstset{
  frame=tb,
  language=TeX,
  aboveskip=3mm,
  belowskip=3mm,
  showstringspaces=false,
  columns=flexible,
  basicstyle={\small\ttfamily},
  numbers=none,
  breaklines=true,
  breakatwhitespace=true,
  tabsize=3
}

\title{高等代数 (荣誉) I 作业模板}
\author{董仕强}

\setcounter{section}{-1}

\setcounter{page}{0}

\setlength\parindent{0pt}

\begin{document}

\maketitle

\section{说明}

可以将作业中遇到的问题标注在此. 如有, 请补充.

\tableofcontents

\newpage

%%%%%%%%%%%%%%%%%%%%%%%%%%%%%%%%%%%%%%%%%%%%%%
%%%%%%%%%%%%%%%%%%%%%%%%%%%%%%%%%%%%%%%%%%%%%%
%%%%%%%%%%%%%%%%%%%%%%%%%%%%%%%%%%%%%%%%%%%%%%
%%%%%%%%%%%%%%%%%%%%%%%%%%%%%%%%%%%%%%%%%%%%%%
%%%%%%%%%%%%%%%%%%%%%%%%%%%%%%%%%%%%%%%%%%%%%%
%%%%%%%%%%%%% 请从此处开始阅读 %%%%%%%%%%%%%%%%

\section{习题课相关}

\subsection{习题课相关}


\begin{qqq}
    State the definition of a number field,and prove that number fields are $\mathbb{Q}$-linear spaces.
\end{qqq}
\begin{aaa}
    设P是由一些复数组成的集合,其中包括0与1,如果P中任意两个数的和,差,积,商(除数不为0)仍是P中的数,则称P为一个数域.
    \newline
    可以将复数域看作在有理数域$\mathbb{Q}$上的线性空间,其维数为无穷.
    任取$\mathbb{C}$中的元素c,可以表示$c=k_1c_1+k_2c_2+...+k_nc_n+...\quad ,k_i\in \mathbb{Q},c_i \in \mathbb{C}$

\end{aaa}
\begin{qqq}
    Prove that the 3-demensional $\mathbb{Q}$-linear space $V$ is a number field.
    $$ V=\{ a+b \cdot 2^{1/3} +c \cdot 2 ^ {2/3} : a,b,c \in \mathbb{Q} \}$$   
\end{qqq}
\begin{proof}
    首先有零元和单位元;
    \newline
    验证加减封闭:\qquad $\forall a,b,c,d,e,f \in \mathbb{Q}$,
    \begin{align*}
    &(a+b \cdot 2^{1/3} +c \cdot 2 ^ {2/3})\pm c+d \cdot 2^{1/3} +e \cdot 2 ^ {2/3}\\
    =&(a\pm d)+(b\pm e) \cdot 2^{1/3} +(c\pm f) \cdot 2 ^ {2/3}\\
    \in &V.
    \end{align*}
    验证乘法封闭:\qquad $\forall a,b,c,d,e,f \in \mathbb{Q}$,
    \begin{align*}
        &(a+b \cdot 2^{1/3} +c \cdot 2 ^ {2/3})(d+e \cdot 2^{1/3} +f \cdot 2 ^ {2/3})\\
        =&(ad+2bf+2ce)+(bd+ae+2cf) \cdot 2^{1/3} +(cd+af+be) \cdot 2 ^ {2/3}\\
        \in &V.
    \end{align*}
    验证除法封闭:\qquad  $\forall a,b,c,d,e,f \in \mathbb{Q}$且$d,e,f$不全为0(记$d^2-2ef,2f^2-de,e^2-af$分别为$D,E,F$),
    \begin{align*}
        &\frac{(a+b \cdot 2^{1/3} +c \cdot 2 ^ {2/3})}{(d+e \cdot 2^{1/3} +f \cdot 2 ^ {2/3})}\\
        =&\frac{(a+b \cdot 2^{1/3} +c \cdot 2 ^ {2/3})((d^2-2ef)+(2f^2-de) \cdot 2^{1/3} +(e^2-af) \cdot 2 ^ {2/3})}{(d+e \cdot 2^{1/3} +f \cdot 2 ^ {2/3})((d^2-2ef)+(2f^2-de) \cdot 2^{1/3} +(e^2-af) \cdot 2 ^ {2/3})}\\
        =&\frac{(aD+2bF+2cE)+(bD+aE+2cF) \cdot 2^{1/3} +(cD+aF+bE) \cdot 2 ^ {2/3}}{d^3+2e^3+4f^3-6def}\\
        =&\frac{aD+2bF+2cE}{d^3+2e^3+4f^3-6def}+\frac{bD+aE+2cF}{d^3+2e^3+4f^3-6def} \cdot 2^{1/3}+\frac{cD+aF+bE}{d^3+2e^3+4f^3-6def} \cdot 2 ^ {2/3}\\
        \in & V
    \end{align*}
    故$V$ is a number field.
\end{proof}
\begin{qqq}
    Find a field $K$ such that $\mathbb{C}$ is a $\mathbf{proper}$ subfield of $K$.
\end{qqq}
\begin{aaa}
    $K$为有理函数域.
\end{aaa}
\begin{qqq}
    Prove that ($1,e^x,e^{2x},...,e^{2024x}$)are linearly independent real-valued functions.
    \newline
    $\bullet$ Hint:take derivatives,and use the fact $\textbf{Vandermonde matrix is invertible}$ as a shortcut.
\end{qqq}
\begin{proof}
    考虑这样的$2024\times 2024$的矩阵.
    $$
    \begin{bmatrix*}
        1 & e^{x_1} & e^{2x_1} & \cdots & e^{(n-1)x_1} \\
        1 & e^{x_2} & e^{2x_2} & \cdots & e^{(n-1)x_2} \\
        \vdots & \vdots & \vdots & \ddots & \vdots \\
        1 & e^{x_n} & e^{2x_n} & \cdots & e^{(n-1)x_n}\\
    \end{bmatrix*}
    $$
    他是一个范德蒙矩阵.只要$x_i \neq x_j, \forall 1 \leq i < j \leq n$,那么这个矩阵可逆,也即矩阵的列向量线性无关.而$x_i$可取遍全体实数.故可以得到,$(1,e^x,e^{2x},\cdots ,e^{2024x})$是线性无关的.
\end{proof}
\begin{qqq}
    find $n$ such that $(\sin \frac{\pi}{2n},\sin \frac{2\pi}{2n},\cdots ,\sin \frac{(n-1)\pi}{2n})$ are linearly dependent (over $\mathbb{Q}$).
\end{qqq}
\begin{aaa}
    skip
\end{aaa}


\section{线性子空间}
\subsection{线性子空间}
\begin{qqq}
    证明一下两个句子包含了相同的子集.\newline
    1.既包含$U_1$,有包含$U_2$的最小线性子空间.\newline
    2.集合$\{ \sum u_1+u_2 | u_1 \in U_1 ,u_2 \in U_2 \}$\newline
    这一子集是线性空间,记作$U_1+U_2$ 
\end{qqq}
\begin{proof}
    一方面,记$U$是包含$U_1,U_2$的最小线性子空间,那么任取$U_1,U_2$中的元素$u_1,u_2$,$u_1+u_2 \in U$.那么$U \subseteq\{ \sum u_1+u_2|u_1 \in U_1,u_2 \in U_2 \}$;\newline
    另一方面,记$U'=\{ \sum u_1+u_2|  u_1 \in U_1 ,u_2 \in U_2\}$,任取$U'$中的元素$u'$,他能用$U_1,U_2$中的元素线性表示,也就是$\forall u' \in U,u' \in U$,即$U' \subseteq U$;\newline
    故$U=U'$.
\end{proof}
\begin{qqq}
    类似的,请以两种观点定义$U_1\cap U_2$.
\end{qqq}
\begin{aaa}
    1.同时是$U_1,U_2$的线性子空间的最大线性子空间.\newline
    2.集合$\{ u:u \in U_1 \wedge u \in U_2\}$
\end{aaa}
\begin{qqq}

\end{qqq}
\begin{qqq}
    直接写出$\cap $满足的交换律.
\end{qqq}
\begin{aaa}
    交换律.$\quad U_1 \cap U_2 = U_2\cap U_1$ \newline
    结合律.$\quad (U_1 \cap U_2)\cap U_3 = U_1 \cap (U_2 \cap U_3)$
\end{aaa}
\begin{qqq}
    写出分配律的反例.此处应化为什么?
\end{qqq}
\begin{aaa}
    取$U_1,U_2,U_3$是平面上三条过同一个点的直线,则等式左边表示$U_3$代表的直线,右边表示三条直线的交点,左边与右边不等.\newline
    $.\qquad \supset$ 
\end{aaa}
\begin{qqq}

\end{qqq}
\begin{aaa}
    $U_1+(U_2 \cap U_3)\subseteq (U_1 + U_2)\cap (U_1 + U_3)$. 
\end{aaa}
\begin{qqq}
    证明线性子空间的modular lattice结构,具体而言,若$U_-$是$U_+$的子空间,则
    $$(U_-+U_0)\cap U_+=U_-+(U_0 \cap U_+)$$.
\end{qqq}
\begin{aaa}
    $\forall u_1+u_2 \in ((U_-+U_0)\cap U_+)$,其中$u_1\in U_-,u_2 \in U_0,\\$
    则$$u_1+u_2\in U_+$$\newline
    又$$u_1 \in u_- \subseteq U_+$$\newline
    则$$u_2 \in U$$\newline
    于是有$$u_1 \in U_- \cap U_+ ,u_2 \in U_0 \cap U_+$$\newline
    既有$$u_1+u_2\in (U_- \cap U_+)+(U_0 \cap U_+)$$\newline
    所以有$$(U_-+U_0)\cap U_+=U_-+(U_0 \cap U_+)$$
\end{aaa}
\section{课堂思考题}
\subsection{课堂思考题}
\begin{qqq}
    To prove that any subset of n+1 vectors of $\mathbb{F}^n$ is linearly dependent.
\end{qqq}
\begin{proof}
    case1.若前n个向量中有线性相关,则n+1个也线性相关.\newline
    case2.若前n个向量线性无关,则这n个向量可以看作这个向量空间中的基$v_1,\cdots ,v_n$.$\mathbb{F}^n=span(v_1,\cdots ,v_n)$.由于第n+1个向量是$\mathbb{F}^n$中的元素,那么他就能被这n个元素线性表示.\newline
    故命题得证.
\end{proof}
\begin{qqq}
    数域(无限域)上,线性方程组解的个数可能有:0个,1个,无限个.那种情况概率大?
\end{qqq}
\begin{aaa}
    1.$\| A \|<1$,则$E-A$可逆,$$E-A=\sum_{k=0}^{\infty}A^k$$.\newline
    $$\|\sum_{k=0}^{\infty}A^k\|\leq \sum_{k=0}^{\infty}\|A\|<+\infty$$\newline
    $$(E-A)\sum_{k=0}^{\infty}A=E$$\newline
    2.$A$可逆,如果$B\in N_\delta (A).\delta =\frac{1}{\|A^{-1}\|}$,则$B$可逆.\newline
    $$\|E-A^{-1}B\|=\|A^{-1}(A-B)\|\leq \|A^{-1}\|\|A-B\|$$.\newline
    $$B\in N_\delta (A),\|B-A\|\leq \frac{1}{\|A^{-1}\|} $$\newline
    $$\|E-A^{-1}B\|<1.$$\newline
    由1.$E-(E-A^{-1}B)$可逆,得$B$可逆.
\end{aaa}
\begin{qqq}
    用Dedekind分割证明$\mathbb{R}$满足加法交换律.
\end{qqq}
\begin{proof}
    由于有理数满足加法的交换律,在定义了加法运算之后我们就有$A+B=B+A$,
\end{proof}
\end{document}
