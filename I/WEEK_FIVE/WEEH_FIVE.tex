\documentclass[11pt]{ctexart}
\usepackage[margin=2cm,a4paper]{geometry}
\usepackage{amsthm, amsfonts, amsmath, amssymb, mathrsfs, newclude, tikz-cd, tikz, ctex, mathtools, stmaryrd, datetime}


%\setmainfont{Caladea}

%% 也可以选用其它字库:
% \setCJKmainfont[%
%   ItalicFont=AR PL KaitiM GB,
%   BoldFont=Noto Sans CJK SC,
% ]{Noto Serif CJK SC}
% \setCJKsansfont{Noto Sans CJK SC}
% \renewcommand{\kaishu}{\CJKfontspec{AR PL KaitiM GB}}



\usepackage[colorlinks = true,
linkcolor = blue,
urlcolor  = blue,
citecolor = blue,
anchorcolor = blue]{hyperref}

% Include the x-color package for color support
\usepackage{xcolor}

% Define a new environment for red comments
\usepackage{verbatim} % Required for the comment environment
\usepackage{environ}

\usepackage{mdframed} % Include mdframed for creating framed environments

\definecolor{pinked}{RGB}{255,231,229} % Define a base color 
% Define a new environment with a background color
\newmdenv[
  backgroundcolor=pinked, % Set the desired background color
  linecolor=white, % Optional: Set the border line color
  linewidth=1pt, % Optional: Set the border line width
  roundcorner=5pt, % Optional: Set rounded corners
  nobreak=true % Optional: Prevent page breaks within the environment
]{pinked}

\theoremstyle{definition}
\newtheorem{qqq}{问题}[section]

\newcommand{\ExternalLink}{%
    \tikz[x=1.2ex, y=1.2ex, baseline=-0.05ex]{% 
        \begin{scope}[x=1ex, y=1ex]
            \clip (-0.1,-0.1) 
                --++ (-0, 1.2) 
                --++ (0.6, 0) 
                --++ (0, -0.6) 
                --++ (0.6, 0) 
                --++ (0, -1);
            \path[draw, 
                line width = 0.5, 
                rounded corners=0.5] 
                (0,0) rectangle (1,1);
        \end{scope}
        \path[draw, line width = 0.5] (0.5, 0.5) 
            -- (1, 1);
        \path[draw, line width = 0.5] (0.6, 1) 
            -- (1, 1) -- (1, 0.6);
        }
    }

\NewEnviron{aaa}{~\\
    \noindent {\textcolor{teal}{\textbf{解答}} \BODY }
}

\NewEnviron{llll}{
    \noindent {~\\$\ExternalLink$ 外部链接 $\,\,\,$ \color{blue}\url{\BODY} }
}

\renewcommand{\proofname}{证明}
\renewcommand\qedsymbol{${\boxed{\substack{\textit{完证}\\\textit{毕明}}}}$}


% Define a custom command for \kuing
\newcommand{\kuing}{\texorpdfstring{$\textstyle{\int_u^c k}=\texttt{kuing}$}{}}



% Change equation numbering to include the section number
\usepackage{cleveref}
\renewcommand{\theequation}{\thesection.\thesubsection.\arabic{equation}}
\numberwithin{equation}{section}


%我可以在这里定义一些简化的命令一图便利
\newcommand{\sspan}{\operatorname{span}}%定义span正体
\newcommand{\op}[1]{\operatorname{#1}}%简化函数表达
\newcommand{\cnm}[2]{\binom{#1}{#2} }%简化组合数命令
\newcommand{\set}[2]{\{ #1 \mid #2\}}%定义集合表示



%截止到这里是我定义的

\usepackage{listings}
% Define listings style
\lstset{
  frame=tb,
  language=TeX,
  aboveskip=3mm,
  belowskip=3mm,
  showstringspaces=false,
  columns=flexible,
  basicstyle={\small\ttfamily},
  numbers=none,
  breaklines=true,
  breakatwhitespace=true,
  tabsize=3
}



\title{高等代数 (荣誉) I 作业模板}
\author{请输入姓名}

\setcounter{section}{-1}

\setcounter{page}{0}

\setlength\parindent{0pt}

\begin{document}

\maketitle

\section{说明}

可以将作业中遇到的问题标注在此. 如有, 请补充.

\tableofcontents

\newpage

%%%%%%%%%%%%%%%%%%%%%%%%%%%%%%%%%%%%%%%%%%%%%%
%%%%%%%%%%%%%%%%%%%%%%%%%%%%%%%%%%%%%%%%%%%%%%
%%%%%%%%%%%%%%%%%%%%%%%%%%%%%%%%%%%%%%%%%%%%%%
%%%%%%%%%%%%%%%%%%%%%%%%%%%%%%%%%%%%%%%%%%%%%%
%%%%%%%%%%%%%%%%%%%%%%%%%%%%%%%%%%%%%%%%%%%%%%
%%%%%%%%%%%%% 请从此处开始阅读 %%%%%%%%%%%%%%%%

\section{逆矩阵}
\subsection{基础习题}
\subsubsection{习题 1}
\begin{qqq}
    将$M$顺时针旋转$90°$,得$\widetilde{M} $.试描述由$M^{-1}$至$(\widetilde{M})^{-1}$的运动过程.
\end{qqq}
\begin{aaa}
    设$M=\begin{pmatrix}
        \alpha_1^T\\\vdots \\\alpha_n^T
    \end{pmatrix}.\quad M^{-1}=\begin{pmatrix}
        \beta_1&\cdots&\beta_n
    \end{pmatrix}$
    \[ (MM^{-1})_{i,j}=\alpha_I^T\beta_j=\beta_j^T\alpha_i=\delta_{ij}E_{ij}. \]
    \[ \widetilde{M}=\begin{pmatrix}
        \alpha_n&\alpha_{n-1}&\cdots&\alpha_1
    \end{pmatrix}.  \]
    \[\left(\begin{pmatrix}
        \beta^T_n\\\beta^T_{n-1}\\\vdots\\\beta_1^T
    \end{pmatrix}\begin{pmatrix}
        \alpha_n&\alpha_{n-1}&\cdots&\alpha_1
    \end{pmatrix}\right)_{ji}=\beta_j^T\alpha_i=\delta_{ji}E_{ji}=(MM^{-1})_{ij}.\]
    所以由$M^{-1}$逆时针旋转$90°$可以得到$(\widetilde{M} )^{-1}$.

\end{aaa}
\subsubsection{习题 2}
\begin{qqq}
    计算以下矩阵的逆矩阵.
    \begin{equation}
        M=\begin{pmatrix}
            1&-1& & & & \\-1&2&-1& & & \\ & -1&2&\ddots& & \\ & &\ddots&\ddots&-1& \\ & & &-1&2&-1\\ & & & &-1&2
        \end{pmatrix}.
    \end{equation}
    备注.$M$并非高度对称的,其严格表述是
    \begin{equation} M=2I-E_{1,1}-\sum_{1\leq i\neq j\leq n}E_{i,j}\end{equation}
\end{qqq}
\begin{aaa}

\end{aaa}
\subsubsection{习题 3}
\begin{qqq}
    计算以下矩阵的逆矩阵.
    \begin{equation}
        M=\begin{pmatrix}
            \xi^{1\cdot 1}&\xi^{1\cdot 2}&\xi^{1\cdot 3}&\cdots&\xi^{1\cdot (n-1)}&\xi^{1\cdot n}\\
            \xi^{2\cdot 1}&\xi^{2\cdot 2}&\xi^{2\cdot 3}&\cdots&\xi^{2\cdot (n-1)}&\xi^{1\cdot n}\\
            \xi^{3\cdot 1}&\xi^{3\cdot 2}&\xi^{3\cdot 3}&\cdots&\xi^{3\cdot (n-1)}&\xi^{3\cdot n}\\
            \vdots&\vdots&\vdots&\ddots&\vdots\\
            \xi^{(n-1)\cdot 1}&\xi^{(n-1)\cdot 2}&\xi^{(n-1)\cdot 3}&\cdots&\xi^{(n-1)\cdot (n-1)}&\xi^{(n-1)\cdot n}\\
            \xi^{n\cdot 1}&\xi^{n\cdot 2}&\xi^{n\cdot 3}&\cdots&\xi^{n\cdot (n-1)}&\xi^{n\cdot n}\\
        \end{pmatrix}
    \end{equation}
    以上$\xi =e^{2\pi i/n}$是$n$次单位根.$n$即$M$的阶数.
    备注.以上矩阵在 Fourier 分析中常见.若无思路,不妨先计算$M^2$.
\end{qqq}
\begin{aaa}
    单位根满足如下性质.
    \[ \sum_{k=1}^n \xi^{pk}=\begin{cases}
        0,&n \nmid p\\
        n,&n \mid p
    \end{cases}  \]
    \[ (M^2)_{ij}=\sum_{k=1}^n \xi^{(i+j)k}=\begin{cases}
        0,&n \nmid (i+j)\\
        n,&i+j=n\hspace{.2cm}or\hspace{.2cm}i+j=2n 
    \end{cases}  \]
    所以
    \[ M^2=\sum_{i+j=n}E_{ij}+E_{nn}. \]
    \[M^4=I.\]
    因此$M\cdot M^3=I$,即$M^{-1}=M^3.$
\end{aaa}
\subsubsection{习题 4}
\begin{qqq}
    计算以下矩阵的逆矩阵.
        \addtocounter{MaxMatrixCols}{15}
        \begin{equation}
            M=\begin{pmatrix}
                0&1&1&\cdots&1&1&1&a&a&a&\cdots&a&a&a\\
                1&0&1&\cdots&1&1&1&a&a&a&\cdots&a&a&a\\
                1&1&0&\cdots&1&1&1&a&a&a&\cdots&a&a&a\\
                \vdots&\vdots&\vdots&\ddots&\vdots&\vdots&\vdots&\vdots&\vdots&\vdots&\ddots&\vdots&\vdots&\vdots\\
                1&1&1&\cdots&0&1&1&a&a&a&\cdots&a&a&a\\
                1&1&1&\cdots&1&0&1&a&a&a&\cdots&a&a&a\\
                1&1&1&\cdots&1&1&0&a&a&a&\cdots&a&a&a\\
                0&0&0&\cdots&0&0&0&0&1&1&\cdots&1&1&1\\
                0&0&0&\cdots&0&0&0&1&0&1&\cdots&1&1&1\\
                0&0&0&\cdots&0&0&0&1&1&0&\cdots&1&1&1\\
                \vdots&\vdots&\vdots&\ddots&\vdots&\vdots&\vdots&\vdots&\vdots&\vdots&\ddots&\vdots&\vdots&\vdots\\
                0&0&0&\cdots&0&0&0&1&1&1&\cdots&0&1&1\\
                0&0&0&\cdots&0&0&0&1&1&1&\cdots&1&0&1\\
                0&0&0&\cdots&0&0&0&1&1&1&\cdots&1&1&0\\
            \end{pmatrix}.
        \end{equation}
\end{qqq}
\begin{aaa}
    记$u=\begin{pmatrix}
        1&1&\cdots&1
    \end{pmatrix}^T$.
    则\[M=\begin{pmatrix}
        uu^T-I&A\\O&uu^T-I
    \end{pmatrix}.\]
    增广矩阵$(M,I)=\begin{pmatrix}
        uu^T-I&A&I&O\\O&uu^T-I&O&I
    \end{pmatrix}\longrightarrow \begin{pmatrix}
        I&O&(uu^T-I)^{-1}&-(uu^T-I)^{-1}A(uu^T-I)^{-1}\\O&I&O&(uu^T-I)^{-1}
    \end{pmatrix}.$
    由\[(A+\alpha\beta^T)^{-1}=A^{-1}-\frac{1}{1+\beta^TA^{-1}\alpha}A^{-1}\alpha\beta^TA^{-1}.\]
    代入即可.
\end{aaa}

\subsubsection{习题 5}
\begin{qqq}
    计算以下矩阵的逆矩阵
    \begin{equation}
        M=\begin{pmatrix}
            1&2&3&\cdots&n-1&n\\n&1&2&\cdots&n-2&n-1\\n-1&n&1&\cdots&n-3&n-2\\\vdots&\vdots&\vdots&\ddots&\vdots&\vdots\\3&4&5&\cdots&1&2\\2&3&4&\cdots&n&1
        \end{pmatrix}
    \end{equation}
\end{qqq}
\begin{aaa}
    \[M^{-1}=\frac{2}{n^2(n+1)}\cdot \text{循环矩阵}
    \begin{pmatrix}
        1-\frac{n(n+1)}{2}&1+\frac{n(n+1)}{2}&1&1&\cdots&1
    \end{pmatrix}.\]
\end{aaa}

\subsubsection{问题 6}
\begin{qqq}
    计算以下矩阵的逆矩阵
    \begin{equation}(M)_{ij}=\min \{i,j\}.\end{equation}
\end{qqq}
\begin{aaa}
    \[M^{-1}=\begin{pmatrix}
        2&-1&&&&&\\-1&2&-1&&&&\\&-1&2&-1&&&\\&&\ddots&\ddots&\ddots&\\&&&-1&2&-1\\&&&&-1&1
    \end{pmatrix}\]
\end{aaa}


\subsubsection{习题 7}
\begin{qqq}
    计算以下矩阵的逆矩阵
    \begin{equation}   
    M=
    \begin{pmatrix}
        k^0\cdot C^0_0&&&&&&\\k^1\cdot C^0_1 &k^0\cdot C^1_1\\k^2\cdot C^0_2&k^1\cdot C^1_2&k^0\cdot C^2_2\\
        \vdots&\vdots&\vdots&\ddots\\k^{n-2}\cdot C^0_{n-2}&k^{n-3}\cdot C^1_{n-2}&k^{n-4}\cdot C^2_{n-2}&\cdots &k^{0}\cdot C^{n-2}_{n-2}\\
        k^{n-1}\cdot C^0_{n-1}&k^{n-2}\cdot C^1_{n-1}&k^{n-3}\cdot C^2_{n-1}&\cdots &k^{1}\cdot C^{n-2}_{n-1}&k^0\cdot C^{n-1}_{n-1}\\
        k^nC^0_n&k^{n-1}\cdot C^0_{n}&k^{n-2}\cdot C^1_{n}&\cdots &k^{2}\cdot C^{n-2}_{n}&k^{1}\cdot C^{n-1}_{n}&k^0\cdot C^{n}_{n}\\
    \end{pmatrix}^T
    \end{equation}
\end{qqq}
\begin{aaa}
    \[M=\begin{pmatrix}
        k^0\cdot C^0_0&&&&&&\\-k^1\cdot C^0_1 &k^0\cdot C^1_1\\-k^2\cdot C^0_2&-k^1\cdot C^1_2&k^0\cdot C^2_2\\
        \vdots&\vdots&\vdots&\ddots\\-k^{n-2}\cdot C^0_{n-2}&-k^{n-3}\cdot C^1_{n-2}&-k^{n-4}\cdot C^2_{n-2}&\cdots &k^{0}\cdot C^{n-2}_{n-2}\\
        -k^{n-1}\cdot C^0_{n-1}&-k^{n-2}\cdot C^1_{n-1}&-k^{n-3}\cdot C^2_{n-1}&\cdots &-k^{1}\cdot C^{n-2}_{n-1}&k^0\cdot C^{n-1}_{n-1}\\
        -k^nC^0_n&k^{n-1}\cdot C^0_{n}&-k^{n-2}\cdot C^1_{n}&\cdots &-k^{2}\cdot C^{n-2}_{n}&-k^{1}\cdot C^{n-1}_{n}&k^0\cdot C^{n}_{n}\\
    \end{pmatrix}\]
\end{aaa}

\subsection{困难习题}
\subsubsection{习题 8}
\begin{qqq}
    计算以下矩阵的逆矩阵
    \begin{equation}
        M=\begin{pmatrix}
            2\cos x&1 & & & & \\1&2\cos x&1 \\&1&2\cos x&1\\&& 1&\ddots&\ddots\\&&&\ddots&2\cos x&1\\&&&&1&2\cos x
        \end{pmatrix}
    \end{equation}
\end{qqq}
\begin{aaa}
    
\end{aaa}





\newpage
\section{线性空间}
\subsection{通过Sage计算LU分解}
\subsubsection{习题 1}
\begin{qqq}
    (广义LU分解)任意矩阵$A\in \mathbb{F}^{m\times n}$可以分解作$A=LD\widetilde{I}SU$的五元乘积形式.
\end{qqq}
\begin{aaa}
    

\end{aaa}

\subsection{线性空间,基的证明题}
\subsubsection{习题 2}
\begin{qqq}
    假定$V$是任意域上的线性空间.试构造子集$S\subset V$(向量组),其同时满足
    \begin{enumerate}
        \item 集合$S$的大小是2024.
        \item $S$中任意20
    \end{enumerate}
\end{qqq}
\begin{aaa}
    取$V$中2024个线性无关的向量$v_1,v_2,\ldots,v_{2024}$,\\构造$S:=\{v_i\mid 1\leq i \leq 2024\}\cup \{v_1+v_2+\cdots +v_{2024}\}$.其满足条件1.\\下证明满足条件2.
    \\$v_1,v_2,\ldots,v_{2024} $是线性无关的,只需证明$v_2,v_3,\ldots,v_{2024},v_1+v_2+\cdots+v_{2024}$线性无关.\\
    假设他们是线性相关的,那么存在不全部为0的$c_1,c_2,\ldots ,c_{2024}$使得\[c_1v_2+c_2v_3+\cdots+c_{2023}v_{2024}+c_{2024}(v_1+v_2+\cdots+v_{2024})=0.\]
    即\begin{equation}c_{2024}v_1+(c_1+c_{2024})v_2+(c_2+c_{2024})v_3+\cdots +(c_{2023}+c_{2024})v_{2024}=0\tag{$\ast $}\end{equation}
    由于$v_1,v_2,\ldots,v_{2024}$线性无关,若$(\ast)$式推得$c_{2024}=c_1=c_2=\cdots=c_{2023}=0$.\\\
    所以构造的$S$符合要求.
\end{aaa}
\subsubsection{习题 4}
\begin{qqq}
    给定数域上的线性空间$V$.任意给定$V$中有限个真子空间$\{U_i\}^m_{i=1}$,总有
    \begin{equation}
        \left(\bigcup ^m_{i=1}U_i\right)\neq V.
    \end{equation}
    (若$\mathbb{F}$不是数域.试给出$m=3$的反例.)
\end{qqq}
\begin{aaa}
    引理:线性空间$U_1,U_2$的并是线性空间当且仅当$U_1\subset U_2$或$U_2\subset U_2.$
    \begin{proof}
        $U_1+U_2$是包含$U_1$和$U_2$的最小的线性空间,因此$(U_1+U_2)\subset (U_1\cup U_2).$\\
        但是任取$U_1\cup U_2$中的元素,他一定在$U_1$或$U_2$中,也在$U_1+U_2$中,因此$(U_1\cup U_2)\subset (U_1+U_2)$.\\
        得出$U_1+U_2=U_1\cup U_2.$\\
        假设结论不成立,那么存在$U_1,U_2$中的元素$v_1,v_2$分别不在向量空间$U_2,U_1$中,因此$v_1+v_2$在$V_1$中或$V_2$中.\\
        若$v_1+v_2\in V_1$,那么$v_2=(v_1+v_2)-v_1\in V_1$,矛盾!\\
        同理可以得到$v_1+v_2 \notin V_2$,因此假设不成立.
    \end{proof}
    由引理可知习题4成立.\\
    \noindent 反例:取$\mathbb{F}$是一个环,不是域.\\
    特别的,\\
    取$\mathbb{F}$为特征值为2的域.那么取$V$是定义在$F$上的线性空间.$V=\{(0,0),(0,1),(1,0),(1,1)\}$.\\
    取$U_1=\{(0,0),(0,1)\},\quad U_2=\{(0,0),(1,0)\},\quad U_3=\{(0,0),(1,1)\},\quad$这是一个反例.
    
\end{aaa}


\subsubsection{习题 8}
\begin{qqq}
    证明\begin{enumerate}
        \item $((V\cap W)+U)\cap V=\hspace{2 cm} =V\cap (W+(U\cap V)),$
        \item $((V+W)\cap U)+V= \hspace{2 cm} = V+(W\cap (U+V)).$
    \end{enumerate}
\end{qqq}
\begin{aaa}
\begin{enumerate}
    \item \begin{proof}
            \[(U\cap V)\subset V ,\quad (V\cap W)\subset V.\]
            \[((V\cap W)+U)\cap V=((V\cap W)\cap V)+(U\cap V)=(U\cap V)+(V+W).\]
            \[V\cap (W+(U\cap V))=(V\cap W)+(V\cap (U+V))=(V\cap W)+(U\cap V)\]
            线性空间的和有交换律.
            \end{proof}

    \item \begin{proof}
             \[V\subset (V+W),\quad V\subset (U+V).\]
             \[((V+W)\cap U)+V=((V+W)+V)\cap (U\cap V)=(V+W)\cap (U+ V).\]
             \[V+(W\cap(U+V))=(V+W)\cap(V+(V+V))=(V+W)\cap (U+V).\]
            \end{proof}
\end{enumerate}
\end{aaa}



\subsection{(span:子集$\rightarrow$子空间)(dim:子空间$\rightarrow \mathbb{N}$)与(rank=dim$\circ $span)}
\subsubsection{习题 16}
\begin{qqq}
    给定子集$S_1$和$S_2$.
    \begin{enumerate}
        \item 证明$\sspan (S_1\cap S_2)\subset \sspan(S_1)\wedge \sspan(S_2).$
        \item 证明$\sspan(S_1)+\sspan(S_2)=\sspan(S_1\cup S_2).$
    \end{enumerate}
\end{qqq}
\begin{aaa}
    \begin{enumerate}
        \item \begin{proof}
                若$S_1\cap S_2=\emptyset$,易知其成立.\\
                若$S_1\cap S_2\neq \emptyset $,\\
                不妨设\[S_1=\{\alpha_1,\ldots,\alpha_n,\beta_1,\ldots,\beta_k\}\]
                    \[S_2=\{\alpha_1,\ldots,\alpha_n,\gamma_1,\ldots,\gamma_{_l}\}\]
                    \[\sspan(S_1)=\sspan\{\alpha_{n_1},\ldots,\alpha_{n_{p}},\beta_{n_1},\ldots,\beta_{n_q}\}\]
                    \[\sspan(S_2)=\sspan\{\alpha_{n_1},\ldots,\alpha_{n_{p}},\gamma_{n_1},\ldots,\gamma_{n_r}\}\]
                    则
                    \[\sspan(S_1\cap S_2)=\sspan\{{\alpha_{n_1}},\ldots,\alpha_{n_p}\}\subset (\sspan(S_1)\wedge \sspan(S_2))\]
                    同时可以举例\[S_1=\{(0,1),(1,1)\},\quad S_2=\{(0,2),(1,1)\}\]说明是包含关系.

                \end{proof}


        \item \begin{proof}
                同1,可以得到\[\sspan(S_1\cup S_2)=\sspan\{\alpha_{n_1},\ldots,\alpha_{n_{p}},\beta_{n_1},\ldots,\beta_{n_q},\gamma_{n_1},\ldots,\gamma_{n_r}\}\]
                容易验证,这等于$\sspan(S_1)+\sspan(S_2)$
                \end{proof}
    \end{enumerate}
\end{aaa}









\end{document}