\documentclass[11pt]{ctexart}
\usepackage[margin=2cm,a4paper]{geometry}
\usepackage{amsthm, amsfonts, amsmath, amssymb, mathrsfs, newclude, tikz-cd, tikz, ctex, mathtools, stmaryrd, datetime}


%\setmainfont{Caladea}

%% 也可以选用其它字库:
% \setCJKmainfont[%
%   ItalicFont=AR PL KaitiM GB,
%   BoldFont=Noto Sans CJK SC,
% ]{Noto Serif CJK SC}
% \setCJKsansfont{Noto Sans CJK SC}
% \renewcommand{\kaishu}{\CJKfontspec{AR PL KaitiM GB}}



\usepackage[colorlinks = true,
linkcolor = blue,
urlcolor  = blue,
citecolor = blue,
anchorcolor = blue]{hyperref}

% Include the x-color package for color support
\usepackage{xcolor}

% Define a new environment for red comments
\usepackage{verbatim} % Required for the comment environment
\usepackage{environ}

\usepackage{mdframed} % Include mdframed for creating framed environments

\definecolor{pinked}{RGB}{255,231,229} % Define a base color 
% Define a new environment with a background color
\newmdenv[
  backgroundcolor=pinked, % Set the desired background color
  linecolor=white, % Optional: Set the border line color
  linewidth=1pt, % Optional: Set the border line width
  roundcorner=5pt, % Optional: Set rounded corners
  nobreak=true % Optional: Prevent page breaks within the environment
]{pinked}

\definecolor{qing}{RGB}{235,243,250} % Define a base color 
% Define a new environment with a background color
\newmdenv[
  backgroundcolor=qing, % Set the desired background color
  linecolor=white, % Optional: Set the border line color
  linewidth=2pt, % Optional: Set the border line width
  roundcorner=5pt, % Optional: Set rounded corners
  nobreak=true % Optional: Prevent page breaks within the environment
]{qing}
\theoremstyle{definition}
\newtheorem{qqq}{问题}[section]

\newcommand{\ExternalLink}{%
    \tikz[x=1.2ex, y=1.2ex, baseline=-0.05ex]{% 
        \begin{scope}[x=1ex, y=1ex]
            \clip (-0.1,-0.1) 
                --++ (-0, 1.2) 
                --++ (0.6, 0) 
                --++ (0, -0.6) 
                --++ (0.6, 0) 
                --++ (0, -1);
            \path[draw, 
                line width = 0.5, 
                rounded corners=0.5] 
                (0,0) rectangle (1,1);
        \end{scope}
        \path[draw, line width = 0.5] (0.5, 0.5) 
            -- (1, 1);
        \path[draw, line width = 0.5] (0.6, 1) 
            -- (1, 1) -- (1, 0.6);
        }
    }

\NewEnviron{aaa}{~\\
    \noindent {\textcolor{teal}{\textbf{解答}} \BODY }
}

\NewEnviron{llll}{
    \noindent {~\\$\ExternalLink$ 外部链接 $\,\,\,$ \color{blue}\url{\BODY} }
}

\renewcommand{\proofname}{证明}
\renewcommand\qedsymbol{${\boxed{\substack{\textit{完证}\\\textit{毕明}}}}$}


% Define a custom command for \kuing
\newcommand{\kuing}{\texorpdfstring{$\textstyle{\int_u^c k}=\texttt{kuing}$}{}}



% Change equation numbering to include the section number
\usepackage{cleveref}
\renewcommand{\theequation}{\thesection.\thesubsection.\arabic{equation}}
\numberwithin{equation}{section}


%我可以在这里定义一些简化的命令一图便利
\newcommand{\sspan}{\operatorname{span}}%定义span正体
\newcommand{\op}[1]{\operatorname{#1}}%简化函数表达
\newcommand{\cnm}[2]{\binom{#1}{#2} }%简化组合数命令
\newcommand{\set}[2]{\{ #1 \mid #2\}}%定义集合表示

%%注意:  行内公式强制行间形式用"  \limits  ",反之用  "  \nolimits   ";



%截止到这里是我定义的

\usepackage{listings}
% Define listings style
\lstset{
  frame=tb,
  language=TeX,
  aboveskip=3mm,
  belowskip=3mm,
  showstringspaces=false,
  columns=flexible,
  basicstyle={\small\ttfamily},
  numbers=none,
  breaklines=true,
  breakatwhitespace=true,
  tabsize=3
}



\title{第7周作业}
\author{董仕强}

\setcounter{section}{-1}

\setcounter{page}{0}

\setlength\parindent{0pt}

\begin{document}

\maketitle

\section{说明}

可以将作业中遇到的问题标注在此. 如有, 请补充.\\
Ex 3.的推广, Ex 5.的第2题不会

\tableofcontents

\newpage

%%%%%%%%%%%%%%%%%%%%%%%%%%%%%%%%%%%%%%%%%%%%%%
%%%%%%%%%%%%%%%%%%%%%%%%%%%%%%%%%%%%%%%%%%%%%%
%%%%%%%%%%%%%%%%%%%%%%%%%%%%%%%%%%%%%%%%%%%%%%
%%%%%%%%%%%%%%%%%%%%%%%%%%%%%%%%%%%%%%%%%%%%%%
%%%%%%%%%%%%%%%%%%%%%%%%%%%%%%%%%%%%%%%%%%%%%%
%%%%%%%%%%%%% 请从此处开始阅读 %%%%%%%%%%%%%%%%

\section{Ex 1.}
\begin{qqq}
    直接写出以下矩阵的行列式,或简要说明其行列式的求解方式.
\begin{qing}
    $\lambda\in \mathbb{F}$是给定的常数,$A\in \mathbb{F}^{n\times n}$是矩阵.
\end{qing}
\begin{enumerate}
    \item 置换矩阵.
    \item 初等变换矩阵$D_i^j,T_{j,i}^\lambda$以及$S_{i,j}$.
    \item 若$A$是对角矩阵,求$\op{det}A$.
    \item 若$A$是上三角矩阵,求$\op{det}A$.
    \item 若$A=\begin{pmatrix}X&O\\Y&Z \end{pmatrix},$其中$X$与$Z$都是方阵.求$\op{det}A$.
    \item $A^{-1}$(若存在)的行列式.
    \item 方阵乘积的行列式.
    \item 若$\op{rank}(A)<n$,求$\op{det}A$.
    \item $\lambda A$的行列式.
    \item $A^T$的行列式.
    \item 将$A$顺时针旋转$\pi /2$后的行列式.
    \item $f$是$\mathbb{F}$上的多项式,求$\op{det}(f(A))$.
    \item 求$\op{det(e^A)}$.
\end{enumerate}
\end{qqq}
\begin{aaa}
    \begin{enumerate}
        \item 分别记第$i$行中$1$出现的位置为$a_i,$将$(a_1,a_2,\ldots,a_i)$相邻两项两两交换最终得到$(1,2,\ldots,n)$需要的次数为$s_n$,那么这个矩阵的行列式为$(-1)^{s_n}$.
        \item $\op{det}(D^\lambda_i)=\lambda,\quad \op{det}(T^\lambda_{j,i})=1,\quad S_{i,j}=-1.$
        \item 主对角线元素的乘积.
        \item 主对角线元素的乘积.
        \item $\op{det}(A)=\op{det}(X)\op{det}(Y)$.
        \item $\op{det}(A^{-1})=1/{\op{det}(A)}$.
        \item 方阵乘积的行列式等于各方阵行列式的乘积.
        \item $\op{det}(A)=0$.
        \item $\op{det}(\lambda A)=\lambda^n \op{det}(A)$.
        \item $\op{det}(A^T)=\op{det}(A)$.
        \item $(-1)^{n(n-1)/2}\op{det}(A)$.
        \item 记$A$的特征根分别为$\lambda_1,\ldots,\lambda_n$,那么$\op{det}(f(A))=\prod\limits _{i=1}^n f(\lambda_i)$.
        \item $\op{det}(e^A)=e^{\op{tr}(A)}.$
    \end{enumerate}
\end{aaa}
\[{}\]
\[{}\]
\[{}\]
\section{Ex 3}
\begin{qqq}
    使用矩阵的初等变换证明,对任意$A\in \mathbb{F}^{n\times n},B\in \mathbb{F}^{n\times n}$,以及$\lambda \in \mathbb{F}$.总有
    \begin{equation}
        \lambda^n \cdot \op{det}(\lambda I_m-AB)=\lambda^m \cdot \det (\lambda I_n-BA).
    \end{equation}
    \begin{qing}
        推广:对方阵$A,B$与$C$(未必可逆),总有$\det(A+B+ABC)=\det(A+B+BCA)$.
    \end{qing}
\end{qqq}
\begin{aaa}
    \begin{itemize}
        \item [Case 1.]若$\lambda=0$,等式两边均为0.成立.
        \item [Case 2.]若$\lambda \neq 0,$令$A=\lambda M$,则只需证明$\det(I_m-MB)=\det(I_n-BM)$.\\考虑分块矩阵$\begin{pmatrix}
            I&M\\B&I
        \end{pmatrix}$的行列式.
        \[\begin{vmatrix}
            I&M\\B&I
        \end{vmatrix}=\begin{vmatrix}
            I&M\\O&I-BM
        \end{vmatrix}=\det(I)\det(I-BM)=\det(I-BM).\]
        \[\begin{vmatrix}
            I&M\\B&I
        \end{vmatrix}=\begin{vmatrix}
            I-MB&O\\B&I
        \end{vmatrix}=\det(I-MB)\det(I)=\det(I-MB).\]
        证毕!
    \end{itemize}
\end{aaa}


\newpage


\section{Ex 4.}
\begin{qqq}
    求以下矩阵行列式.
    \begin{equation}
        \begin{pmatrix}
            0& & & &a_n\\1&0& & &a_{n-1}\\ & 1&\ddots& &\vdots\\ & &\ddots&0&a_2\\ & & &1&a_1
        \end{pmatrix}
    \end{equation}
\end{qqq}

\begin{aaa}
    分别将第$1$列的$-a_n$倍,第$2$列的$-a_{n-1}$倍,$\ldots$,第$(n-1)$列的$-a_1$倍加到第$n$列,得到\[\begin{pmatrix}
        0& & & &a_n\\1&0& & &0\\ & 1&\ddots& &\vdots\\ & &\ddots&0&0\\ & & &1&0
    \end{pmatrix}.\]
    因此
    \[\begin{vmatrix}
        0& & & &a_n\\1&0& & &a_{n-1}\\ & 1&\ddots& &\vdots\\ & &\ddots&0&a_2\\ & & &1&a_1
    \end{vmatrix}=\begin{vmatrix} 0& & & &a_n\\1&0& & &0\\ & 1&\ddots& &\vdots\\ & &\ddots&0&0\\ & & &1&0\end{vmatrix}=(-1)^{n+1}a_n\begin{vmatrix}1\\&1\\ &&\ddots\\&&&1\end{vmatrix}=(-1)^{n+1}a_n.\]
\end{aaa}

\newpage
\section{Ex 5.}
以下是三对角矩阵的行列式问题.
\begin{qqq}
    求以下三对角矩阵的行列式.
    \begin{equation}
        \begin{pmatrix}
            a&b& & & \\c&a&b& & \\ & c&\ddots&\ddots& \\ & & \ddots &a & b\\ &&&c&a
        \end{pmatrix}.
    \end{equation}
\end{qqq}
\begin{aaa}
    按照最后一行展开得到\[D_{n}=aD_{n-1}-bcD_{n-2}.\]
    其中$D_1=a,\quad D_2=a^2-bc.$\\
    记方程$x^2-ax+bc=0$的根为$\alpha,\beta.$\\
    若$a^2-4bc\neq 0$,$\alpha\neq \beta$$,D_n=A\alpha^n+B\beta^n$,其中$A,B$由带入$D_1,D_2$带入得到.\\
    若$a^2-4bc= 0$,$\alpha= \beta$$,D_n=(A+Bn)\alpha^n$,其中$A,B$由带入$D_1,D_2$带入得到.
\end{aaa}
\[{}\]
\begin{qqq}
    证明
    \begin{equation}
        \det\begin{pmatrix}
            a_1&1\\-1&a_2&1\\ &-1&\ddots&\ddots&\\ &&\ddots&a_{n-1}&1\\&&&-1&a_n
        \end{pmatrix}=
        a_1+\frac{1}{a_2+\frac{1}{\ddots{+\frac{1}{a_{n-1}+\frac{1}{a_n}}}}}
    \end{equation}
\end{qqq}
\begin{aaa}
    按最后一列展开,得到$D_n=a_nD_{n-1}+D_{n-2}$.\quad $D_1=a_1,D_2=a_1a_2+1.$\\
    不会了.为什么$n=2$和$n=3$我算出来只有分子一样,没有分母.
\end{aaa}

\[{}\]
\begin{qqq}
    证明
    \begin{equation}
        \det\begin{pmatrix}
            a_1&b_1\\c_1&a_2&b_2\\ &c_2&\ddots&\ddots&\\ &&\ddots&a_{n-1}&b_{n-1}\\&&&c_{n-1}&a_n
        \end{pmatrix}=\begin{pmatrix}
            a_1&b_1
        \end{pmatrix}\begin{pmatrix}
            a_2&b_2\\-c_1&0
        \end{pmatrix}\cdots\begin{pmatrix}
            a_{n-1}&b_{n-1}\\-c_{n-2}&0
        \end{pmatrix}\begin{pmatrix}
            a_n\\-c_{n-1}
        \end{pmatrix}
    \end{equation}
\end{qqq}
\begin{aaa}
    按最后一列展开,得到$D_n=a_nD_{n-1}-b_{n-1}c_{n-1}D_{n-2}$.\quad $D_1=a_1,D_2=a_1a_2-b_1c_1.$\\
    可以定义$D_0=1$\\
    即\[\begin{pmatrix}
        D_n&b_nD_{n-1}
    \end{pmatrix}=
    \begin{pmatrix}
        D_{n-1}&b_{n-1}D_{n-2}
    \end{pmatrix}
    \begin{pmatrix}
        a_n&b_n\\-c_{n-1}&0
    \end{pmatrix}\]
    因此\[\begin{pmatrix}
        D_{n-1}&bD_{n-2}
    \end{pmatrix}=
    \begin{pmatrix}
        D_1&b_1D_0
    \end{pmatrix}\begin{pmatrix}
        a_2&b_2\\-c_2&0
    \end{pmatrix}\cdots\begin{pmatrix}
        a_{n-1}&b_{n-1}\\-c_{n-2}&0
    \end{pmatrix}.\]
    结合$D_n=a_nD_{n-1}-b_{n-1}c_{n-1}D_{n-2}$及$D_0,D_1$即证!
\end{aaa}

\[{}\]
\[{}\]
\section{Ex 8.}
\begin{qqq}
    取$(a_i)_{i \geq 1}$是周期为$n$的$\mathbb{F}$中的数列,定义$n \times n$矩阵的第$(i,j)$项为$a_{i+j-1}$.计算这一循环矩阵的行列式.
\end{qqq}
\begin{aaa}
    考虑矩阵\[M=\begin{pmatrix}
        &1\\&&1\\&&&\ddots\\&&&&1\\1
    \end{pmatrix}\]
    满足$M^{n}=I$. 定义$M^0=I$.\\
    定义矩阵$A$是题述循环矩阵. 那么$A=\sum \limits _{i=0}^{n-1}a_{i+1}M^{i}.$ 
    记$f(x)=\sum\limits_{i=0}^{n-1}a_{i+1}M^i.$ 则$A=f(M).$\\
    设矩阵$M$的特征值是$\lambda$,  那么$M^n$特征值就是$I$的特征值, 也即$\lambda^n=1$ ,因此矩阵$M$的特征值 为$n$个$n$次单位根.\\
    因此\[\det(A)=\det(f(M))=\prod_{i=0}^{n-1}f(w^i).\]
    其中$w$是$n$次单位根,即$\exp{(\frac{2\pi}{n}\op{i})}.$
\end{aaa}


\section{Ex 9.}
\begin{qqq}
    给定常数$(c_1,c_2,\ldots,c_n)$. 试计算$(c_{\min(i,j)})\in \mathbb{F}^{n\times n}$的行列式. 

\end{qqq}
\begin{aaa}
    将每一行加上前一行得-1倍,即得上三角矩阵,容易求得行列式.
    \[
    \begin{vmatrix}
        c_1&c_1&c_1&\cdots&c_1\\
        c_1&c_2&c_2&\cdots&c_2\\
        c_1&c_2&c_3&\cdots&c_3\\
        \vdots&\vdots&\vdots&\ddots&\vdots\\
        c_1&c_2&c_3&\cdots&c_n
    \end{vmatrix}
    =
    \begin{vmatrix}
        c_1&c_1&c_1&\cdots&c_1\\
        0&c_2-c_1&c_2-c_1&\cdots&c_2-c_1\\
        0&0&c_3-c_2&\cdots&c_3-c_2\\
        \vdots&\vdots&\vdots&\ddots&\vdots\\
        0&0&0&\cdots&c_n-c_{n-1}
    \end{vmatrix}
    =c_1\prod_{i=1}^{n-1}(c_{i+1}-c_i).
    \]
\end{aaa}









\end{document}