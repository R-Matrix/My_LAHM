\documentclass[11pt]{ctexart}
\usepackage[margin=2cm,a4paper]{geometry}
\usepackage{amsthm, amsfonts, amsmath, amssymb, mathrsfs, newclude, tikz-cd, tikz, ctex, mathtools, stmaryrd, datetime}


%\setmainfont{Caladea}

%% 也可以选用其它字库:
% \setCJKmainfont[%
%   ItalicFont=AR PL KaitiM GB,
%   BoldFont=Noto Sans CJK SC,
% ]{Noto Serif CJK SC}
% \setCJKsansfont{Noto Sans CJK SC}
% \renewcommand{\kaishu}{\CJKfontspec{AR PL KaitiM GB}}



\usepackage[colorlinks = true,
linkcolor = blue,
urlcolor  = blue,
citecolor = blue,
anchorcolor = blue]{hyperref}

% Include the x-color package for color support
\usepackage{xcolor}

% Define a new environment for red comments
\usepackage{verbatim} % Required for the comment environment
\usepackage{environ}

\usepackage{mdframed} % Include mdframed for creating framed environments

\definecolor{pinked}{RGB}{255,231,229} % Define a base color 
% Define a new environment with a background color
\newmdenv[
  backgroundcolor=pinked, % Set the desired background color
  linecolor=white, % Optional: Set the border line color
  linewidth=1pt, % Optional: Set the border line width
  roundcorner=5pt, % Optional: Set rounded corners
  nobreak=true % Optional: Prevent page breaks within the environment
]{pinked}

\theoremstyle{definition}
\newtheorem{qqq}{问题}[section]

\newcommand{\ExternalLink}{%
    \tikz[x=1.2ex, y=1.2ex, baseline=-0.05ex]{% 
        \begin{scope}[x=1ex, y=1ex]
            \clip (-0.1,-0.1) 
                --++ (-0, 1.2) 
                --++ (0.6, 0) 
                --++ (0, -0.6) 
                --++ (0.6, 0) 
                --++ (0, -1);
            \path[draw, 
                line width = 0.5, 
                rounded corners=0.5] 
                (0,0) rectangle (1,1);
        \end{scope}
        \path[draw, line width = 0.5] (0.5, 0.5) 
            -- (1, 1);
        \path[draw, line width = 0.5] (0.6, 1) 
            -- (1, 1) -- (1, 0.6);
        }
    }

\NewEnviron{aaa}{~\\
    \noindent {\textcolor{teal}{\textbf{解答}} \BODY }
}

\NewEnviron{llll}{
    \noindent {~\\$\ExternalLink$ 外部链接 $\,\,\,$ \color{blue}\url{\BODY} }
}

\renewcommand{\proofname}{证明}
\renewcommand\qedsymbol{${\boxed{\substack{\textit{完证}\\\textit{毕明}}}}$}


% Define a custom command for \kuing
\newcommand{\kuing}{\texorpdfstring{$\textstyle{\int_u^c k}=\texttt{kuing}$}{}}

% Change equation numbering to include the section number
\usepackage{cleveref}
\renewcommand{\theequation}{\thesection.\thesubsection.\arabic{equation}}
\numberwithin{equation}{section}

\usepackage{listings}
% Define listings style
\lstset{
  frame=tb,
  language=TeX,
  aboveskip=3mm,
  belowskip=3mm,
  showstringspaces=false,
  columns=flexible,
  basicstyle={\small\ttfamily},
  numbers=none,
  breaklines=true,
  breakatwhitespace=true,
  tabsize=3
}

\title{高等代数 (荣誉) I 作业模板}
\author{董仕强}

\setcounter{section}{-1}

\setcounter{page}{0}

\setlength\parindent{0pt}

\begin{document}

\maketitle

\section{说明}

可以将作业中遇到的问题标注在此. 如有, 请补充.\\
Exercise的第三题,不会做。



%%%%%%%%%%%%%%%%%%%%%%%%%%%%%%%%%%%%%%%%%%%%%%
%%%%%%%%%%%%%%%%%%%%%%%%%%%%%%%%%%%%%%%%%%%%%%
%%%%%%%%%%%%%%%%%%%%%%%%%%%%%%%%%%%%%%%%%%%%%%
%%%%%%%%%%%%%%%%%%%%%%%%%%%%%%%%%%%%%%%%%%%%%%
%%%%%%%%%%%%%%%%%%%%%%%%%%%%%%%%%%%%%%%%%%%%%%
%%%%%%%%%%%%% 请从此处开始阅读 %%%%%%%%%%%%%%%%



\begin{proof}
    $\\\S 1.1\\
    \quad 27.\quad 16 \quad 8 \quad 32\\
    \quad 30. \quad c^2+d^2=0\quad or \quad \textbf{v}=\lambda\textbf w,\lambda \in \mathbf{F}\\
    \quad  \textbf{u}=(1,1,1,1)\textbf{v}=(1,1,1,-1)\textbf{w}=(1,1,-1,1)\textbf{z}=(1,-1,1,1)\\
    \quad 31. \quad 2c-d=0 \quad\& \quad-c+2d-e=0\quad \&\quad -d+2e=0\\
    \quad  c=0.75\quad d=0.5 \quad e=0.25\\ \\
    \S 1.2\\ 
    30.\textbf{u}=(2,1)\textbf{v}=(-2,1)\textbf{w}=(1,-3)\quad 4\\$
    proof:平面中,$\textbf{u}\cdot \textbf{v}<0 \Leftrightarrow \arg(\textbf{u},\textbf{v})>90°\\
    $反证法,假设平面中有四个向量两两内积均为负,也即两两夹角为钝角,这显然不可能,因为这样一个周角就大于360°了。所以平面里四个向量是不可能的。
    $\\ 31.(x+y+z)^2=0,\quad s.t.\quad x^2+y^2+z^2=-2(xy+yz+zx)\\ \frac{\textbf{u}\cdot \textbf{v}}{\|\textbf{u} \| \|\textbf{v}\|}=\frac{xz+yx+zy}{x^2+y^2+z^2}=-\frac12$$\\
    \\
    \\\S1.3\\
    $
    $3.\textbf{A} = \textbf{S}^{-1} \textbf{C}=$
    $\begin{pmatrix}
        1 & 0 & 0 \\ -1 & 1 & 0 \\ 0 &-1 & 1
    \end{pmatrix}
    \begin{pmatrix}
        c_1 \\ c_2 \\ c_3
    \end{pmatrix}\\ independent.\\ $
    $\\ 5.y=
    \begin{pmatrix}
        1\\-2\\1
    \end{pmatrix}
    or
    \begin{pmatrix}
        100\\-200\\100
    \end{pmatrix}
    \\
    \\
     6.\quad 3 \quad -1 \quad 0   \\
        \\$
    
    $
     \textbf{Exercise 1.}\\
    1.\begin{pmatrix}
        a_{11}b{11}+a_{12}b_{21} & a_{11}b_{212}+a_{12}b_{22} \\
        a_{21}b_{11}+a_{22}b_{21} & a_{21}b_{12}+a_{22}b_{22} \\
    \end{pmatrix}
    \\
    2.\quad 8\quad 4\\
    3. 
    $ 
    计算得,
    $$S_2+S_3-S_6-S_7=a_{11}b_{11}+a_{12}b_{21}$$
    $$S_4+S_6= a_{11}b_{212}+a_{12}b_{22}$$
    $$ S_5+S_7=a_{21}b_{11}+a_{22}b_{21}$$
    $$S_1-S_3-S_4-S_5=a_{21}b_{12}+a_{22}b_{22}$$
    $\\
    4.\quad 12 \quad 24 \\
    \\$
    \newline
    $ \textbf{Exercise 2.}\\$
    1.
    $$ A\cdot B=
    \begin{pmatrix}
        ad-bc & 0 \\
        0 & -bc+ad \\
    \end{pmatrix}  $$  
    $$ B\cdot A=
    \begin{pmatrix}
        da-bc & 0 \\
        0 & -cb+ad \\
    \end{pmatrix}  $$ 
    $2.$
    $$T=
    \begin{pmatrix}
        \frac{1}{\sqrt 5}&-\frac{1+\sqrt 5}{2\sqrt 5}\\
        -\frac{1}{\sqrt 5}&\frac{1-\sqrt 5}{2\sqrt 5}\\
    \end{pmatrix}
    $$
    $3.$
    $$\lambda_1=\frac{\sqrt 5-1}{2}$$
    $$\lambda_2=\frac{-\sqrt 5 -1}{2}$$
    $4.$
    $$\begin{pmatrix}
        1&1\\1&0\\
    \end{pmatrix}
    \cdot 
    \begin{pmatrix}
        a_{n+1}\\a_n
    \end{pmatrix}
    =
    \begin{pmatrix}
        a_{n+1}+a_n\\a_{n+1}\\
    \end{pmatrix}
    =
    \begin{pmatrix}
        a_{n+2}\\a_{n+1}
    \end{pmatrix}
    $$
    $5.$
    $$
    \begin{pmatrix}
        a_{n+1}\\a_n
    \end{pmatrix}
    =
    \begin{pmatrix}
        1&1\\1&0\\
    \end{pmatrix}
    ^{n+1}
    \cdot
    \begin{pmatrix}
        1\\1\\
    \end{pmatrix}
    $$
    $6.$利用矩阵,得到的通项表示更加简洁。与特征根法相比,不用求方程的根。而实际上,特征根法中的特征方程与这个矩阵的特征方程相同,特征根法中的特征根就是这个矩阵的两个特征值。$\\$
    \newline
    \newline
    $\textbf{Exercise 3.}\\$
    $1.\text{  加法 . 乘法.  共轭  .模平方  . 实部?}\\
    2.A=\begin{pmatrix}
        \cos \frac{2k\pi}{2023}&\sin \frac{2k\pi}{2023}\\
        -\sin \frac{2k\pi}{2023}&\cos \frac{2k\pi}{2023}\\
    \end{pmatrix}\quad , k \in \mathbb{Z}\\
    3.??????????????????????????????????????????????\\
    \text{矩阵级数的第n+1项}\frac{A}{n!},\text{他的范数} \| \frac{A^n}{n!} \|=\frac{\|A^n\|}{n!}\leq\frac{\|A\|^n}{n!}\\
    \text{这个级数的收敛性可以通过比较它与标量级数}\sum \frac{\|A\| ^n}{n!}  \text{来判断.由于标量级数是指数函数} e ^{\|A\|}\text{的展开,肯定收敛。}\\\text{所以矩阵级数也收敛}\\
    4.\text{可以。}j=
    \begin{pmatrix}
        i&0\\0&-i\\
    \end{pmatrix},k=
    \begin{pmatrix}
        0&i\\i&0\\
    \end{pmatrix}   
    $
    
    
$\int_{0}^{+\infty}e^{-\sqrt {x}} = 2 = \infty =$
\end{proof}
    
\end{document}