\documentclass[11pt]{ctexart}
\usepackage[margin=2cm,a4paper]{geometry}
\usepackage{amsthm, amsfonts, amsmath, amssymb, mathrsfs, newclude, tikz-cd, tikz, ctex, mathtools, stmaryrd, datetime}

%\setmainfont{Caladea}

%% 也可以选用其它字库:
% \setCJKmainfont[%
%   ItalicFont=AR PL KaitiM GB,
%   BoldFont=Noto Sans CJK SC,
% ]{Noto Serif CJK SC}
% \setCJKsansfont{Noto Sans CJK SC}
% \renewcommand{\kaishu}{\CJKfontspec{AR PL KaitiM GB}}



\usepackage[colorlinks = true,
linkcolor = blue,
urlcolor  = blue,
citecolor = blue,
anchorcolor = blue]{hyperref}

% Include the x-color package for color support
\usepackage{xcolor}

% Define a new environment for red comments
\usepackage{verbatim} % Required for the comment environment
\usepackage{environ}

\usepackage{mdframed} % Include mdframed for creating framed environments

\definecolor{pinked}{RGB}{255,231,229} % Define a base color 
% Define a new environment with a background color
\newmdenv[
  backgroundcolor=pinked, % Set the desired background color
  linecolor=white, % Optional: Set the border line color
  linewidth=1pt, % Optional: Set the border line width
  roundcorner=5pt, % Optional: Set rounded corners
  nobreak=true % Optional: Prevent page breaks within the environment
]{pinked}

\theoremstyle{definition}
\newtheorem{qqq}{问题}[section]

\newcommand{\ExternalLink}{%
    \tikz[x=1.2ex, y=1.2ex, baseline=-0.05ex]{% 
        \begin{scope}[x=1ex, y=1ex]
            \clip (-0.1,-0.1) 
                --++ (-0, 1.2) 
                --++ (0.6, 0) 
                --++ (0, -0.6) 
                --++ (0.6, 0) 
                --++ (0, -1);
            \path[draw, 
                line width = 0.5, 
                rounded corners=0.5] 
                (0,0) rectangle (1,1);
        \end{scope}
        \path[draw, line width = 0.5] (0.5, 0.5) 
            -- (1, 1);
        \path[draw, line width = 0.5] (0.6, 1) 
            -- (1, 1) -- (1, 0.6);
        }
    }

\NewEnviron{aaa}{~\\
    \noindent {\textcolor{teal}{\textbf{解答}} \BODY }
}

\NewEnviron{llll}{
    \noindent {~\\$\ExternalLink$ 外部链接 $\,\,\,$ \color{blue}\url{\BODY} }
}

\renewcommand{\proofname}{证明}
\renewcommand\qedsymbol{${\boxed{\substack{\textit{完证}\\\textit{毕明}}}}$}


% Define a custom command for \kuing
\newcommand{\kuing}{\texorpdfstring{$\textstyle{\int_u^c k}=\texttt{kuing}$}{}}



% Change equation numbering to include the section number
\usepackage{cleveref}
\renewcommand{\theequation}{\thesection.\thesubsection.\arabic{equation}}
\numberwithin{equation}{section}


%我可以在这里定义一些简化的命令一图便利
\newcommand{\sspan}{\operatorname{span}}%定义span正体
\newcommand{\op}[1]{\operatorname{#1}}%简化函数表达
\newcommand{\cnm}[2]{\binom{#1}{#2} }%简化组合数命令
\newcommand{\set}[2]{\{ #1 \mid #2\}}%定义集合表示



%截止到这里是我定义的

\usepackage{listings}
% Define listings style
\lstset{
  frame=tb,
  language=TeX,
  aboveskip=3mm,
  belowskip=3mm,
  showstringspaces=false,
  columns=flexible,
  basicstyle={\small\ttfamily},
  numbers=none,
  breaklines=true,
  breakatwhitespace=true,
  tabsize=3
}



\title{第三周作业}
\author{董仕强}

\setcounter{section}{-1}

\setcounter{page}{0}

\setlength\parindent{0pt}

\begin{document}

\maketitle
{涉及矩阵的初等变换与阶梯型 (échelon form)}
\section{说明}

可以将作业中遇到的问题标注在此. 如有, 请补充.

\tableofcontents

\newpage

%%%%%%%%%%%%%%%%%%%%%%%%%%%%%%%%%%%%%%%%%%%%%%
%%%%%%%%%%%%%%%%%%%%%%%%%%%%%%%%%%%%%%%%%%%%%%
%%%%%%%%%%%%%%%%%%%%%%%%%%%%%%%%%%%%%%%%%%%%%%
%%%%%%%%%%%%%%%%%%%%%%%%%%%%%%%%%%%%%%%%%%%%%%
%%%%%%%%%%%%%%%%%%%%%%%%%%%%%%%%%%%%%%%%%%%%%%
%%%%%%%%%%%%% 请从此处开始阅读 %%%%%%%%%%%%%%%%

\section{第三周习题}

\subsection{Problem 1}给定n行矩阵A.
\begin{qqq}
    \begin{pinked}
    交换$A$的$[i,j]$两行,等价于左乘一个矩阵$S_{i,j}$.写出该矩阵.
    \end{pinked}
\end{qqq}
\begin{aaa}
    $$S_{i,j}=I-E_{i,i}-E_{j,j}+E_{i,j}+E_{j,i}\quad.$$
\end{aaa}
\begin{qqq}
    \begin{pinked}
        将$A$第$k$行的各项同时乘上一个非零常数$\lambda$,等价于左乘一个矩阵$D^{\lambda}_k$,写出该矩阵.
    \end{pinked}
    
\end{qqq}
\begin{aaa}
    $$D^{\lambda}_k=I+(\lambda-1)E_{k,k}\quad.$$
\end{aaa}
\begin{qqq}
    \begin{pinked}
        向$A$的第$j$行加上第$i$行的$\lambda$倍(这一过程仅改变第$j$行,其他行不变),等价于左乘一个矩阵$T^{\lambda}_{i,j}.$写出该矩阵.
    \end{pinked}
    
\end{qqq}
\begin{aaa}
    $$T^{\lambda}_{i,j}=I+\lambda E_{j,i}\quad .$$
\end{aaa}
\begin{qqq}
    \begin{pinked}
        求逆变换(逆矩阵)$S^{-1}_{i,j},(D^\lambda _{k})^{-1},$以及$(T^\lambda _{i,j})^{-1}\quad.$
    \end{pinked}
   
\end{qqq}
\begin{aaa}
    $$S^{-1}_{i,j}=S_{i,j}\quad , \quad
    (D^\lambda _{k})^{-1}=(D^{1/\lambda} _{k})\quad , \quad
    (T^\lambda _{i,j})^{-1}=(T^{-\lambda} _{i,j})^{-1} \quad .$$
\end{aaa}
\begin{qqq}
    \begin{pinked}
        使用自然语言描述这三类逆变换.
    \end{pinked}
    
\end{qqq}
\begin{aaa}
    将置换后的两行再次置换;\newline
    将第$k$行的各项同时乘上一个非零常数$1/\lambda$;\newline
    向$A$的第$j$行加上第$i$行的$-\lambda$倍.
\end{aaa}
\begin{qqq}
    \begin{pinked}
        求$S_{i,j}S_{k,l}=S_{k,l}S_{i,j}$的充要条件.
    \end{pinked}
        
\end{qqq}
\begin{aaa}
    $\{i,j\}=\{k,l\}$或$\{i,j\}\cap \{k,l\}=\varnothing $;\newline
    充分性:结合$S_{i,j}$变换的意义容易验证.\newline
    必要性:我们结合$S_{i,j}$变换的意义容易验证有$S_{i,j}=S_{j,i}$.\newline
    不妨设$i<j,k<l$,若$i,j,k,l$互不相等,那么先交换$i,j$再交换$k,l$和先交换$k,l$再交换$i,j$效果相同.\newline
    若有一个相同,不妨设$j=k$,那么三列顺序本来是$xyz$,先$i,j$后$j,l$变成$yzx$,先$j,l$后$i,j$变成$zxy$,不相等,不成立.\newline
    若有两个相同,那么由$S_{i,j}=S_{j,i}$知,两边相同.
\end{aaa}
\begin{qqq}
    \begin{pinked}
        求$T^\lambda _{i,j}T^\mu _{k,l}=T^\mu _{k,l}T^\lambda _{i,j}$的充要条件.
    \end{pinked}
    
\end{qqq}
\begin{aaa}
    $i=j=k=l$或$(i-k)(j-l)\neq 0$.
    $$T^\lambda _{i,j}T^\mu _{k,l}=(I+\lambda E_{j,i})(I+\mu E_{k,l})=I+\lambda E_{j,i}+\mu E_{k,l}+\lambda\mu E_{j,i}E_{k,l}$$
    $$T^\mu _{k,l}T^\lambda _{i,j}=(I+\mu E_{k,l})(I+\lambda E_{j,i})=I+\lambda E_{j,i}+\mu E_{k,l}+\lambda\mu E_{k,l}E_{j,i}$$
    故只需考虑$E_{j,i}E_{k,l}=E_{k,l}E_{j,i}$的充要条件.\newline
    若$i \neq k,\quad E_{j,i}E_{k,l}=O,$那么在该条件下等号成立的充要条件是$E_{k,l}E_{j,i}=O$,也即$l \neq j$;\newline
    若$i=k,E_{j,i}E_{k,l}=E_{j,l}$那么在该条件下等号成立的充要条件是$E_{k,l}E_{j,i}=E_{j,l}$,也即$k=j,l=j,i=l$,也即$i=j=k=l$;\newline
    综上,原命题的充要条件是$i=j=k=l$或$(i-k)(j-l)\neq 0$
\end{aaa}
\begin{qqq}
    \begin{pinked}
        能否通过某两类矩阵得到第三类?
    \end{pinked}
    
\end{qqq}
\begin{aaa}
    $S\& D\to T:\\$
    \begin{align*}
        &S_{i,j}D^\lambda _i+S_{i,j}S_{i,j}-S_{i,j}\\
        =&(I-E_{i,i}-E_{j,j}+E_{i,j}+E_{j,i})(I+(\lambda-1)E_{i,i})+I-(I-E_{i,i}-E_{j,j}+E_{i,j}+E_{j,i})\\
        =&(I-E_{i,i}-E_{j,j}+E_{i,j}+\lambda E_{j,i})-(E_{i,i}-E_{j,j}+E_{i,j}+E_{j,i})\\
        =&I+(\lambda-1)E_{j,i}\\
        =&T^{\lambda-1}_{i,j}
    \end{align*}
    $S \& T \to D$:
    \begin{align*}
        &S_{i,j}T^\lambda _{i,j}+S_{i,j}S_{i,j}-S_{i,j}\\
        =&(I-E_{i,i}-E_{j,j}+E_{i,j}+E_{j,i})(I+\lambda E_{j,i})+I-(I-E_{i,i}-E_{j,j}+E_{i,j}+E_{j,i})\\
        =&(I+(\lambda-1)E_{i,i}-E_{j,j}+E_{i,j}+E_{j,i})+(E_{i,i}+E_{j,j}-E_{i,j}-E{j,i})\\
        =&I+\lambda E_{i,i}\\
        =&D^{\lambda+1}_i
    \end{align*}
    $D \& T \to S$:
    \begin{align*}
        &-D^2_i-D^2_j+T^1_{i,j}+T^1_{j,i}+D^1_i\\
        =&-(I+E_{i,i})-(I-E_{j,j})+(I+E_{j,i})+(I+E_{i,j})+I\\
        =&I-E_{i,i}-E_{j,j}+E_{i,j}+E{j,i}\\
        =&S_{i,j}
    \end{align*}
\end{aaa}
\begin{qqq}
    \begin{pinked}
        假定$A$是方阵,将以上三个矩阵乘在$A$右侧,效果如何.
    \end{pinked}
    
\end{qqq}
\begin{aaa}
    将$A$的第$i$列与第$j$列交换;\newline
    将$A$的第$k$列的各项同时乘上一个非零常数;\newline
    向矩阵$A$的第$j$列加上第$i$行的$\lambda$倍.
\end{aaa}

\subsection{Problem 2}矩阵的最简行阶梯形
\begin{qqq}
    \begin{pinked}
        给定矩阵$A$,其最简行阶梯形$R$为何唯一?
    \end{pinked}
    
\end{qqq}
\begin{proof}
    仅考虑非零矩阵$A=[\alpha_1,\ldots,\alpha_n]$,不妨设$n\geq m$,$\alpha_1\neq 0$,若不然考虑将最小下标非零向量$\alpha_k$列做类似的操作.\newline
    不妨设$a_{11}=0$,若不然可以利用置换矩阵$S$来左乘$A$使得非零元素位于第一行.然后利用倍乘矩阵使得$a_{11}=1$.\newline
    再左乘消元矩阵使得$a_{k1}=0,\quad(k=2,3,\ldots,n)$.此时$A$可以表示为
    $$\begin{pmatrix}
        1&B^{(1)}_{1\times (n-1)}\\O_{(m-1)\times }1&A^{(1)}_{(m-1)\times(n-1)}
    \end{pmatrix}.$$
    再对$A^{(1)}$进行上述的操作,得到的矩阵再操作,最后会得到一个行阶梯型矩阵$A'$,其递推类似于
    \begin{align*}
        A^{(k-1)}=
        \begin{pmatrix}
            1&B^{(k-1)}\\
            O&A^{(k)}
        \end{pmatrix}\quad or \quad
        \begin{pmatrix}
            O& 1&B^{(k-1)}\\
            O&O&A^{(k)}
        \end{pmatrix}.
    \end{align*}
    再进行化为最简操作.只需考虑$A'$,从最后一列开始,进行消元操作,仅保留该列最后一行元素,该列其他行元素都能变成0.进行有限次操作后就能得到最简行阶梯形.\newline
    在这样的操作下肯定得到的矩阵是唯一的.
        
\end{proof}
\begin{qqq}
    \begin{pinked}
    尝试给出一个无字证明.
    \end{pinked}
    
\end{qqq}
\begin{aaa}
    一个矩阵$A$可以对应一个线性方程组$Ax=0$.矩阵化为行最简即视为解出这个方程组.\newline
    当这个方程组有非零解时矩阵$A$的各列是线性相关的,否则是线性无关的.
    由于对行进行初等变换不改变各列的线性相关性,也即不改变方程的解,这些变换是同解变化.\newline
    而一个确定的方程他的解也是确定的,所以行最简矩阵唯一.
\end{aaa}
\begin{qqq}
    转置矩阵$R^T$的最简行阶梯形是什么?
\end{qqq}
\begin{aaa}
    矩阵$R$的列最简矩阵.
\end{aaa}
\begin{qqq}
    \begin{pinked}
    证明相抵标准型的存在性:对任意矩阵$A$,存在可逆矩阵$P$和$Q$使得
    $$A=P
    \begin{pmatrix}
        I_r & O\\O & O
    \end{pmatrix}
    Q.$$
    以上,$I_r$是$r$阶单位矩阵,$O$表示数字0出现的位置.
    \end{pinked}
    
\end{qqq}
\begin{proof}
    由问题1.10的解答可以知道,\newline
    对任意矩阵$A$,其行最简存在,可以由一系列初等变换得到.对他的行最简矩阵再进行置换操作可以得到一个新的矩阵$A''$,而这些变换都可以通过对$A$左乘初等矩阵$P_i$得到.记$P'=P_1P_2\cdots P_s.\\$
    $$A''=P'A=
    \begin{pmatrix}
        I_r&F\\O&O
    \end{pmatrix}.$$
    
    $$(A'')^T=\begin{pmatrix}
        I_r&O\\F^T&O\\
    \end{pmatrix}$$
    对转置后的矩阵进行行最简操作,由于$I_r$与$F^T$列数相同且这$r$列均非0,因此会得到矩阵$A'''$,左乘的矩阵为$Q_i$,记$Q'=(Q_1Q_2\cdots Q_t)^T$
    $$A'''=(Q')^T(A'')^T=\begin{pmatrix}
        I_r&O\\O&O
    \end{pmatrix}$$
    得到了,
    $$\begin{pmatrix}
        I_r&O\\O&O
    \end{pmatrix}
    =
    \begin{pmatrix}
        I_r&O\\O&O
    \end{pmatrix}^T
    =
    ((Q')^T(A'')^T)^T
    =
    A''Q'
    =
    P'AQ'$$
    $P',Q'$显然可逆,因为他们都可以表示成初等矩阵的乘积,且可逆矩阵转置后仍然可逆.\newline
    两边同时左乘右乘逆,得到
    $$A=(P')^{-1}\begin{pmatrix}
        I_r&O\\O&O
    \end{pmatrix}(Q')^{-1}$$
    取$P=(P')^{-1},Q=(Q')^{-1}$,即可得
    $$A=P
    \begin{pmatrix}
        I_r & O\\O & O
    \end{pmatrix}
    Q.$$

    
\end{proof}    
\begin{qqq}
    \begin{pinked}
        证明以上的$r$由$A$唯一决定.作为推论,矩阵的行秩等于列秩.往后统一称作秩.
    \end{pinked}  
\end{qqq}
\begin{proof}
    由问题1.13知道,$A$经过初等变换后得到的$\begin{pmatrix}
        I_r&F\\O&O
    \end{pmatrix}$与$\begin{pmatrix}
        I_r & O\\O & O
    \end{pmatrix}$的$r$相同.\newline
    而矩阵$\begin{pmatrix}
        I_r&F\\O&O
    \end{pmatrix}$中的$r$表示方程$Ax=0$的主元的个数.这是由方程系数矩阵$A$唯一决定的.
\end{proof}
\newpage
\section{Challenging Problem 1}
\begin{qqq} 
    \begin{pinked}
    若整数矩阵$A=
    \begin{pmatrix}
        a&b\\c&d
    \end{pmatrix}$满足$ad-bc=1$,则$A$是以下几类矩阵的有限乘积.$$
    S=\begin{pmatrix}
        0 &1\\-1&0
    \end{pmatrix},\quad
    T=\begin{pmatrix}
        1&1\\0&1
    \end{pmatrix},\quad
    T^{-1}=\begin{pmatrix}
        1&-1\\0&1
    \end{pmatrix}.
    $$
    \end{pinked}   
\end{qqq}
\begin{proof}
    反过来考虑对$A$进行操作,对$A$右乘$S^{-1}$,$T$或$T^{-1}$,目的是得到单位矩阵.
    我们有两种操作.
    \begin{align}
        AS^{-1}=
        \begin{pmatrix}
            b&-a\\d&-c
        \end{pmatrix}
    \end{align}
    \begin{align*}
        AS^{-1}S^{-1}=
        \begin{pmatrix}
            -a&-b\\-c&-d
        \end{pmatrix}\text{(由(2.1)可以推得)}
    \end{align*}
    \begin{align}
        AT=
        \begin{pmatrix}
            a+b&b\\c+d&d
        \end{pmatrix}
    \end{align}
    \begin{align}
        AT^{-1}=
        \begin{pmatrix}
            a-b&b\\c-d&d
        \end{pmatrix}
    \end{align}
    (2.1)操作视为两列互换位置,并且新的第二列乘以-1.\newline
    (2,2)和(2.3)操作视为将第二列加或减到第一列.\newline
    如果将矩阵$A$任选一种操作,经过变换后的矩阵为$A'=\begin{pmatrix}
        a'&b'\\c'&d'
    \end{pmatrix},$计算可知$\boldmath{a'd'-b'c'=1.}$\newline
    此外,我们注意到$ad-bc=1$.\newline
    若$bc=0,ad=1$.若$a,d$均为1那么就是单位矩阵.若均为$-1$则操作两次(2.1)即可.\newline
    若$ad<0$,那么对矩阵$A$操作两次(2.1)即可变得均为正.\newline
    若$bc>0$,可以知道,$\op{gcd}(ad,bc)=1$,有$a,b$互质.\newline
    对其进行$\textbf{辗转相除法}$的有限次操作,最终会得到一个矩阵,使得其左上角的元素为$1$.\newline
    得到新矩阵$A''=\begin{pmatrix}
        1&x\\y&z
    \end{pmatrix}$,且$z-xy=1$.\newline
    $$A^{(3)}=A''S^{-1}=\begin{pmatrix}
        x&-1\\z&-y
    \end{pmatrix}$$
    $$A^{(4)}=A^{(3)}T^{x}=\begin{pmatrix}
        0&-1\\z-xy&-y
    \end{pmatrix}=\begin{pmatrix}
        0&-1\\1&-y
    \end{pmatrix}$$
    $$A^{(5)}=A^{(4)}S^{-1}=\begin{pmatrix}
        -1&0\\-y&-1
    \end{pmatrix}$$
    $$A^{(6)}=A^{(5)}(T^{-1})^y=\begin{pmatrix}
        -1&0\\0&-1
    \end{pmatrix}$$
    $$A^{(7)}=A^{(6)}(S^{-1})^2=\begin{pmatrix}
        1&0\\0&1
    \end{pmatrix}$$
\end{proof}
\end{document}