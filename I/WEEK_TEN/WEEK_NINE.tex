\documentclass[11pt]{ctexart}
\usepackage[margin=2cm,a4paper]{geometry}
\usepackage{amsthm, amsfonts, amsmath, amssymb, mathrsfs, newclude, tikz-cd, tikz, ctex, mathtools, stmaryrd, datetime}


%\setmainfont{Caladea}

%% 也可以选用其它字库:
% \setCJKmainfont[%
%   ItalicFont=AR PL KaitiM GB,
%   BoldFont=Noto Sans CJK SC,
% ]{Noto Serif CJK SC}
% \setCJKsansfont{Noto Sans CJK SC}
% \renewcommand{\kaishu}{\CJKfontspec{AR PL KaitiM GB}}



\usepackage[colorlinks = true,
linkcolor = blue,
urlcolor  = blue,
citecolor = blue,
anchorcolor = blue]{hyperref}

% Include the x-color package for color support
\usepackage{xcolor}

% Define a new environment for red comments
\usepackage{verbatim} % Required for the comment environment
\usepackage{environ}

\usepackage{mdframed} % Include mdframed for creating framed environments

\definecolor{pinked}{RGB}{255,231,229} % Define a base color 
% Define a new environment with a background color
\newmdenv[
  backgroundcolor=pinked, % Set the desired background color
  linecolor=white, % Optional: Set the border line color
  linewidth=1pt, % Optional: Set the border line width
  roundcorner=5pt, % Optional: Set rounded corners
  nobreak=true % Optional: Prevent page breaks within the environment
]{pinked}

\definecolor{qing}{RGB}{235,243,250} % Define a base color 
% Define a new environment with a background color
\newmdenv[
  backgroundcolor=qing, % Set the desired background color
  linecolor=white, % Optional: Set the border line color
  linewidth=2pt, % Optional: Set the border line width
  roundcorner=5pt, % Optional: Set rounded corners
  nobreak=true % Optional: Prevent page breaks within the environment
]{qing}
\theoremstyle{definition}
\newtheorem{qqq}{问题}[section]

\newcommand{\ExternalLink}{%
    \tikz[x=1.2ex, y=1.2ex, baseline=-0.05ex]{% 
        \begin{scope}[x=1ex, y=1ex]
            \clip (-0.1,-0.1) 
                --++ (-0, 1.2) 
                --++ (0.6, 0) 
                --++ (0, -0.6) 
                --++ (0.6, 0) 
                --++ (0, -1);
            \path[draw, 
                line width = 0.5, 
                rounded corners=0.5] 
                (0,0) rectangle (1,1);
        \end{scope}
        \path[draw, line width = 0.5] (0.5, 0.5) 
            -- (1, 1);
        \path[draw, line width = 0.5] (0.6, 1) 
            -- (1, 1) -- (1, 0.6);
        }
    }

\NewEnviron{aaa}{~\\
    \noindent {\textcolor{teal}{\textbf{解答}} \BODY }
}

\NewEnviron{llll}{
    \noindent {~\\$\ExternalLink$ 外部链接 $\,\,\,$ \color{blue}\url{\BODY} }
}

\renewcommand{\proofname}{证明}
\renewcommand\qedsymbol{${\boxed{\substack{\textit{完证}\\\textit{毕明}}}}$}


% Define a custom command for \kuing
\newcommand{\kuing}{\texorpdfstring{$\textstyle{\int_u^c k}=\texttt{kuing}$}{}}



% Change equation numbering to include the section number
\usepackage{cleveref}
\renewcommand{\theequation}{\thesection.\thesubsection.\arabic{equation}}
\numberwithin{equation}{section}


%我可以在这里定义一些简化的命令一图便利
\newcommand{\sspan}{\operatorname{span}}%定义span正体
\newcommand{\op}[1]{\operatorname{#1}}%简化函数表达
\newcommand{\cnm}[2]{\binom{#1}{#2} }%简化组合数命令
\newcommand{\set}[2]{\{ #1 \mid #2\}}%定义集合表示
\newcommand{\FF}{\mathbb{F}}%数域
\newcommand{\RR}{\mathbb{R}}
\newcommand{\CC}{\mathbb{C}}
\newcommand{\QQ}{\mathbb{Q}}

%%注意:  行内公式强制行间形式用"  \limits  ",反之用  "  \nolimits   ";
%%注意:  公式引用编号 :" \label{}  ";   公式显示编号" \tag{}  ":  引用公式"  \eqref{}    "


%截止到这里是我定义的

\usepackage{listings}
% Define listings style
\lstset{
  frame=tb,
  language=TeX,
  aboveskip=3mm,
  belowskip=3mm,
  showstringspaces=false,
  columns=flexible,
  basicstyle={\small\ttfamily},
  numbers=none,
  breaklines=true,
  breakatwhitespace=true,
  tabsize=3
}



\title{第10周作业}
\author{董仕强}

\setcounter{section}{-1}

\setcounter{page}{0}

\setlength\parindent{0pt}

\begin{document}

\maketitle

\section{说明}

可以将作业中遇到的问题标注在此. 如有, 请补充.\\


\tableofcontents

\newpage

%%%%%%%%%%%%%%%%%%%%%%%%%%%%%%%%%%%%%%%%%%%%%%
%%%%%%%%%%%%%%%%%%%%%%%%%%%%%%%%%%%%%%%%%%%%%%
%%%%%%%%%%%%%%%%%%%%%%%%%%%%%%%%%%%%%%%%%%%%%%
%%%%%%%%%%%%%%%%%%%%%%%%%%%%%%%%%%%%%%%%%%%%%%
%%%%%%%%%%%%%%%%%%%%%%%%%%%%%%%%%%%%%%%%%%%%%%
%%%%%%%%%%%%% 请从此处开始阅读 %%%%%%%%%%%%%%%%


%\section{Problem 1(零空间的增长)}记$A\in \FF^{n\times n}$是任意域上的方阵.约定$A^0=I$是单位矩阵,以及$A^{k+1}=A\cdot A^K$.
%\begin{qqq}
%    证明有子空间的包含列\[
%    0=N(A^0)\subset N(A^1) \subset N(A^2) \subset \cdots
%    \]
%    特别地,若$N(A^N)=N(A^{N+1})$,则$N(A^{N+1})=N(A^{N+2})=\cdots$.
%\end{qqq}
%\begin{proof}
    
%\end{proof}

\section{二次型}
\subsection{Ex 1.消歧义问题}
假定$U$是$\FF$上的有限维线性空间.
\begin{qqq}
    称$f:U\times U \rightarrow \FF $是双线性的,当且仅当对任意向量与常数,
    \[f(au+v,bx+y)=abf(u,x)+af(u,y)+f(v,x)+f(v,y),\]
    试证明:$\{f\mid f:U\times U\rightarrow \FF\text{是双线性映射}\}$是一个$\FF$-线性空间,其对象是一些二元函数.求其维度与基.
\end{qqq}
\begin{proof}
    $f\equiv 0$符合定义,因此集合非空.\\
    定义加法$(f+g)(x,y)=f(x,y)+g(x,y)$,数乘$(kf)(x,y)=kf(x,y)$,
    由定义$$(f+g)(au+v,bx+y)=ab(f+g)(u,x)+a(f+g)(u,y)+(f+g)(v,x)+(f+g)(v,y)$$
    $$(kf)(au+v,bx+y)=f(kau+kv,bx+y)$$
    维数是$\dim U\times \dim U$,基为${f(u_i,u_j)}$,其中$u_i$是$U$的一组基.
\end{proof}
\[{}\]
\begin{qqq}
    依照集合的Cartesian积,定义新的集合$U\times U=\{(u_1,u_2)\mid u_1,u_2 \in U\}.$试证明$U\times U$也是线性空间,并求其维数与基.
\end{qqq}
\begin{proof}
     定义加法$(u_1,u_2)+(u_1'+u_2')=(u_1+u_1',u_2+u_2')\in U\times U$.\\
     定义数乘$k(u_1,u_2)=(ku_1,ku_2)\in U\times U$.\\
     维数是$dim U\times \dim U$.基是$(v_i,v_j)$,其中$v_i$是$U$的一组基.
\end{proof}
\[{}\]
\begin{qqq}
    试证明:$\{f\mid f: U\times U\rightarrow \FF \text{是线性映射}\}$是一个$\FF$-线性空间,其对象是一些一元函数.求其维数与基.
\end{qqq}
\begin{proof}
    显然集合非空\\
    定义加法$(f+g)(u_1,u_2)=f(u_1,u_2)+g(u_1,u_2)$和数乘$(kf)(u_1,u_2)=kf(u_1,u_2)$.\\
    那么由定义可以知道这个集合对加法和数乘都封闭.且满足其他性质.\\
    维数是$\dim U\times \dim U$,基为${f(u_i,u_j)}$,其中$u_i$是$U$的一组基.
\end{proof}
\newpage
\subsection{Ex 2.二次型的最值问题}
\begin{qqq}
     记$A$是实对称矩阵,证明$A$的最大特征值是$\sup_{x\neq 0}\frac{x^TAx}{x^Tx}$,并考虑取达最大值的充要条件. 同时, 这也说明$\sup$可以改成$\max$.
\end{qqq}
\begin{proof}
    作换元处理,令$x=Py$,其中$P$为正交矩阵且$P^{-1}AP=D$,$D$为对角矩阵.\\那么$$\sup_{x\neq 0} \frac{x^TAx}{x^Tx}=\sup_{y\neq 0} \frac{y^TDy}{y^Ty}$$.\\
    设$y=\begin{pmatrix}
        y_1&y_2&\cdots&y_n
    \end{pmatrix}^T,D=\op{diag}(\lambda_1,\lambda_2,\ldots,\lambda_n)$\\
    因此求$$\sup _{\sum\limits_{k=1}^n y_k^2\neq 0}\frac{\sum_{k=1}^n \lambda_ky_k^2}{\sum_{k=1}^n y_k^2}$$
    考虑$\max\limits_{1\leq k \leq n}(\lambda_k)=\lambda_r$,那么$$\frac{\sum \lambda_ky_k^2}{\sum y_k^2}\leq \frac{\sum \lambda_ry_k^2}{\sum y_k^2}=\lambda_r$$
    取等当且仅当$y_i=y_r\cdot \delta_{ir},\forall 1\leq i\leq n$,其中$r$满足$\lambda_r=\max\limits_{1\leq i \leq n}(\lambda_i)$.
\end{proof}
\[{}\]
\begin{qqq}
    记$A$是实对称矩阵,记最大特征值为$\lambda_1$的重数为1,相应的特征向量为$Av=\lambda_1v$.证明$A$的第二特征值是$\sup _{x \bot x_1,x\neq 0}\frac{x^TAx}{x^Tx}$.此处$x_1$是使得上一问取达最大值的任一向量.
\end{qqq}
\begin{proof}
    若$x\bot x_1$,也即$xx_1^T=0$,也即$Py\cdot(Py_1)^T=-0$,即$yy_1^T=0$.其中$y_1=\begin{pmatrix}t&0&\cdots&0\end{pmatrix}^T$, $t$ 为任意非0实数.\\
    因此令$y'=\begin{pmatrix}
        y_2&\cdots&y_n
    \end{pmatrix}^T$,$D=\op{diag}(\lambda_2,\ldots,\lambda_n)$\\
    $$\sup _{x \bot x_1,x\neq 0}\frac{x^TAx}{x^Tx}=\sup_{y'\neq 0}\frac{y'^TD'y}{y'^Ty}$$
    下同 \textbf{问题1.4.} 即可.
\end{proof}
\[{}\]
\begin{qqq}
    假定$A$是实对称正定矩阵,证明$\inf _{x\neq 0}\frac{x^TA^{-1}x}{x^Tx}$和$\sup _{x\neq 0}\frac{x^TAx}{x^Tx}$互为倒数.
\end{qqq}
\begin{proof}
    由于$A$是正定矩阵,知道$A^{-1}$也正定,即$A$的所有特征值都大于0.而$A^{-1}$的特征值为$A$的特征值的倒数.\\
    类似\textbf{问题1.4.}我们可以知道,$A$的最小特征值是$\inf_{x\neq 0}\frac{x^TAx}{x^Tx}$.因此若$A$的最大特征值为$\lambda_1$,那么$A^{-1}$的最小特征值是$1/\lambda_1$.
\end{proof}
\[{}\]
\begin{qqq}
    记$\{x_i\}^n_{i=1}$是实数,满足$x_1^2+x_2^2+\cdots+x_n^2=1$与$x_1+x_2+\cdots+x_n=0$.求\[x_1x_2+x_2x_3+\cdots+x_{n-1}x_n+x_nx_1\]
    的最大值.
\end{qqq}
\begin{aaa}
    记$x=\begin{pmatrix}
        0&x_1&x_2&\cdots&x_n
    \end{pmatrix}^T$, $x^T\cdot\textbf{1}=0.$\\
    对称矩阵$A=\begin{pmatrix}
        0&1&0&0&\cdots&1\\
        1&0&1&0&\cdots&0\\
        0&1&0&1&\ddots&0\\
        \vdots&\ddots&\ddots&\ddots&\ddots&\vdots\\
        0&\cdots&0&1&0&1\\
        1&0&\cdots&0&1&0
    \end{pmatrix}$,第二大特征值为$\lambda_2$,那么$\max \sum \limits_{cyc}x_1x_2=\frac{\lambda_2}{2}$
\end{aaa}
\newpage
\subsection{Ex 5.极分解}以下仅讨论对称半正定矩阵.
\begin{qqq}
    若$A$是对称半正定矩阵,则存在唯一的对称半正定矩阵$\sqrt{A}$使得$\sqrt{A}^2=A$.
\end{qqq}
\begin{proof}
    $A=U^T\Lambda U$,那么取$\sqrt{A}=U^T\sqrt{\Lambda}U$.\\
    假设存在对称半正定矩阵$B^2=C^2=A$,那么$B^2=C^2$,设$B=U^*\Lambda U,C=V^*\Lambda V.$ 因此$U^*\Lambda^2U=V^*\Lambda^2V$.故而$\Lambda^2UV^*=UV^*\Lambda^2,$,因此$UV^*$与$\Lambda^2$可交换,因此$UV^*$是对角矩阵因此$UV^*$与$\Lambda$可交换    .故$UV^*\Lambda=\Lambda UV^*$,即$V^*\Lambda V=U^*\Lambda U$. 
\end{proof}
\[{}\]
\begin{qqq}
    任意矩阵$A$都是对称半正定矩阵与正交矩阵的乘积.若$A$对称正定,则这一分解唯一.
\end{qqq}
\begin{proof}
    $A=U^*\Sigma V=U^*\Sigma U\cdot U^*V=SQ$.\\
    假设$A=SQ=S_1Q_1$,因此$AA^*=SS^*=S_1S_1^*$,故$S^2=S_1^2$,利用上一问知道$S=S_1$.
\end{proof}\[{}\]
\begin{qqq}
    假设$S$实正定,$Q$正交.若$\det(xI-SQ)=\det(xI-S)$,则$S=SQ$.
\end{qqq}
\begin{proof}
    由题目知$S \sim SQ$,存在正交矩阵$U,USU^*=D$.\\
    不会了.
\end{proof}
\newpage
\section{Jordan标准型}
\subsection{Problem 3.}
假定$A$与$B$是\textbf{一般域}上的方阵,下面研究乘法式$AX-XB$.
\begin{qqq}
    (第三问)证明:$AX-XB=O$只有零解,当且仅当$A$与$B$的特征多项式互素.
\end{qqq}
\begin{proof}
    不失一般性设$X=\begin{pmatrix}
        I_r&O\\O&O
    \end{pmatrix}$,其中$r=rank(X)$,$A=\begin{pmatrix}
        A_{11}&A_{12}\\A_{21}&A_{22}
    \end{pmatrix},B=\begin{pmatrix}
        B_{11}&B_{12}\\B_{21}&B_{22}
    \end{pmatrix}$\\
    \[AX=\begin{pmatrix}
        A_{11}&O\\A_{21}&O
    \end{pmatrix},\quad XB=\begin{pmatrix}
        B_{11}&B_{12}\\O&O
    \end{pmatrix}.\]
    $AX=XB$,那么$A_{11}=B_{11},A_{21}=O,B_{12}=O$
    \[\det(\lambda I-A)=\det \begin{pmatrix}
        \lambda I-A_{11}&A_{12}\\O&\lambda I-A_{22}
    \end{pmatrix}\]
    \[\det(\lambda I-B)=\det \begin{pmatrix}
        \lambda I-B_{11}&O\\B_{21}&\lambda I-B_{22}
    \end{pmatrix}\]
    所以$\lambda I-A$与$\lambda I-B$的特征值由$r$个相同.
    所以$X=O$等价于$A$与$B$的特征多项式互素.
\end{proof}
\[{}\]
\begin{qqq}
    给定矩阵$A\in \FF^{n\times n}$和$B\in \FF^{m\times m}$.证明一下命题等价.
    \begin{enumerate}
        \item 对未知量$X\in \FF^{n\times m}$,方程$AX-XB=O$只有零解.
        \item 任意给定矩阵$C\in \FF^{n\times m}$,对未知量$X\in \FF^{n\times m}$,方程$AX-XB=C$总有解.
        \item 任意给定矩阵$C\in \FF^{n\times m}$,对未知量$X\in \FF^{n\times m}$,方程$AX-XB=C$有且仅有唯一解.
        \item 对任意相似矩阵$C$,总有相似矩阵\[\begin{pmatrix}A&C\\O&B\end{pmatrix}\sim\begin{pmatrix}A&O\\O&B\end{pmatrix}\]
        \item $A$与$B$的特征多项式互素.
    \end{enumerate}
\end{qqq}
\begin{proof}
    \[{}\]
    \begin{enumerate}
        \item [$(5)\Leftrightarrow(1)$]上一问已经证明.
        \item [$(5)\Rightarrow(3)$] 作以下矩阵变换结果不变\[B\rightarrow P^{-1}BP,C\rightarrow CP,X\rightarrow XP.\]于是假定$B$是Jordan标准型.\\设$X=(\alpha_1,\ldots,\alpha_n),C=(\beta_1,\ldots,\beta_n)$.那么$AX-XB=C$可以拆分为k个独立方程,如果$B$有n个快\\$(A-\lambda_i I)\alpha_i=\beta_i.$由于互素因此$\lambda_i$不是$A$的特征值.$A-\lambda_iI$可逆,因此每个方程都有解,所以矩阵方程有解且唯一.
        \item [$(3)\Rightarrow(5)$] 假设$A,B$有相同特征值$\lambda$.\\若这$k$个方程有一个无解,那么这个矩阵方程无解.\\若都有解,由于$\lambda$是$A$的特征值,那么$(A-\lambda I)x=0$有无穷多组解.若$(\alpha_1,\ldots,\alpha_n)$是上述方程的一组解,那么对$(A-\lambda I)x=0$的任意一个解$\alpha_0$,$(\alpha_1,\ldots,\alpha_n+\alpha_0)$也是解,因此$AX-XB=C$有无穷多组解.矛盾
        \item [$(1)\Leftrightarrow(4)$] 显然成立.
        \item [$(2)\Rightarrow(3)$] 不太会
    \end{enumerate}
\end{proof}














\end{document}