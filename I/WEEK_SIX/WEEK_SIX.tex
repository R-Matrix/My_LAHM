\documentclass[11pt]{ctexart}
\usepackage[margin=2cm,a4paper]{geometry}
\usepackage{amsthm, amsfonts, amsmath, amssymb, mathrsfs, newclude, tikz-cd, tikz, ctex, mathtools, stmaryrd, datetime}


%\setmainfont{Caladea}

%% 也可以选用其它字库:
% \setCJKmainfont[%
%   ItalicFont=AR PL KaitiM GB,
%   BoldFont=Noto Sans CJK SC,
% ]{Noto Serif CJK SC}
% \setCJKsansfont{Noto Sans CJK SC}
% \renewcommand{\kaishu}{\CJKfontspec{AR PL KaitiM GB}}



\usepackage[colorlinks = true,
linkcolor = blue,
urlcolor  = blue,
citecolor = blue,
anchorcolor = blue]{hyperref}

% Include the x-color package for color support
\usepackage{xcolor}

% Define a new environment for red comments
\usepackage{verbatim} % Required for the comment environment
\usepackage{environ}

\usepackage{mdframed} % Include mdframed for creating framed environments

\definecolor{pinked}{RGB}{255,231,229} % Define a base color 
% Define a new environment with a background color
\newmdenv[
  backgroundcolor=pinked, % Set the desired background color
  linecolor=white, % Optional: Set the border line color
  linewidth=1pt, % Optional: Set the border line width
  roundcorner=5pt, % Optional: Set rounded corners
  nobreak=true % Optional: Prevent page breaks within the environment
]{pinked}

\theoremstyle{definition}
\newtheorem{qqq}{问题}[section]

\newcommand{\ExternalLink}{%
    \tikz[x=1.2ex, y=1.2ex, baseline=-0.05ex]{% 
        \begin{scope}[x=1ex, y=1ex]
            \clip (-0.1,-0.1) 
                --++ (-0, 1.2) 
                --++ (0.6, 0) 
                --++ (0, -0.6) 
                --++ (0.6, 0) 
                --++ (0, -1);
            \path[draw, 
                line width = 0.5, 
                rounded corners=0.5] 
                (0,0) rectangle (1,1);
        \end{scope}
        \path[draw, line width = 0.5] (0.5, 0.5) 
            -- (1, 1);
        \path[draw, line width = 0.5] (0.6, 1) 
            -- (1, 1) -- (1, 0.6);
        }
    }

\NewEnviron{aaa}{~\\
    \noindent {\textcolor{teal}{\textbf{解答}} \BODY }
}

\NewEnviron{llll}{
    \noindent {~\\$\ExternalLink$ 外部链接 $\,\,\,$ \color{blue}\url{\BODY} }
}

\renewcommand{\proofname}{证明}
\renewcommand\qedsymbol{${\boxed{\substack{\textit{完证}\\\textit{毕明}}}}$}


% Define a custom command for \kuing
\newcommand{\kuing}{\texorpdfstring{$\textstyle{\int_u^c k}=\texttt{kuing}$}{}}



% Change equation numbering to include the section number
\usepackage{cleveref}
\renewcommand{\theequation}{\thesection.\thesubsection.\arabic{equation}}
\numberwithin{equation}{section}


%我可以在这里定义一些简化的命令一图便利
\newcommand{\sspan}{\operatorname{span}}%定义span正体
\newcommand{\op}[1]{\operatorname{#1}}%简化函数表达
\newcommand{\cnm}[2]{\binom{#1}{#2} }%简化组合数命令
\newcommand{\set}[2]{\{ #1 \mid #2\}}%定义集合表示



%截止到这里是我定义的

\usepackage{listings}
% Define listings style
\lstset{
  frame=tb,
  language=TeX,
  aboveskip=3mm,
  belowskip=3mm,
  showstringspaces=false,
  columns=flexible,
  basicstyle={\small\ttfamily},
  numbers=none,
  breaklines=true,
  breakatwhitespace=true,
  tabsize=3
}



\title{第六周作业}
\author{董仕强}

\setcounter{section}{-1}

\setcounter{page}{0}

\setlength\parindent{0pt}

\begin{document}

\maketitle

\section{说明}

可以将作业中遇到的问题标注在此. 如有, 请补充.

\tableofcontents

\newpage

%%%%%%%%%%%%%%%%%%%%%%%%%%%%%%%%%%%%%%%%%%%%%%
%%%%%%%%%%%%%%%%%%%%%%%%%%%%%%%%%%%%%%%%%%%%%%
%%%%%%%%%%%%%%%%%%%%%%%%%%%%%%%%%%%%%%%%%%%%%%
%%%%%%%%%%%%%%%%%%%%%%%%%%%%%%%%%%%%%%%%%%%%%%
%%%%%%%%%%%%%%%%%%%%%%%%%%%%%%%%%%%%%%%%%%%%%%
%%%%%%%%%%%%% 请从此处开始阅读 %%%%%%%%%%%%%%%%

\section{投影,正交性,最小二乘法}
\subsection{problem 1}
\begin{qqq}
    以下是一些关于$\mathbb{R}^n$中向量内积的小题.
    \begin{enumerate}
        \item 给定$\|u\|\leq 1$与$\|v\|\leq 1$,证明:$\sqrt{1-\|u\|^2}\cdot \sqrt{1-\|v\|^2}\leq 1-<u,v>.$
        \item 证明:$\|u+v\|\cdot\|u-v\|\leq\|u\|^2+\|v\|^2.$
        \item 证明:$\left\langle u,v\right\rangle =0,$当且仅当$\|u\|\leq\|u+c\cdot v\|$对一切$c\in\mathbb{R}$成立.
        \item 证明:任意给定$u\neq 0,$则对一切$\|v\|=1$均有$\|u-(\|u\|^{-1}\cdot u)\|\leq \|u-v\|.$
        \item 表述极化恒等式.
        \item 表述平行四边形恒等式.
    \end{enumerate}
\end{qqq}
\begin{aaa}
    \begin{enumerate}
        \item \begin{proof}
                由柯西不等式,只需证\[\sqrt{1-\|u\|^2}\cdot \sqrt{1-\|v\|^2}\leq 1-\|u\|\|v\|.\]
                两边同时平方,即证\[(1-\|u\|^2)(1-\|v\|^2)\leq 1-2\|u\|\|v\|+(\|u\|\|v\|)^2.\]
                即\[(\|u\|-\|v\|)^2\geq 0.\]
                \end{proof}
        \item \begin{proof}
                记$a=(u+v)/2,b=(u-v)/2$.\\
                只需证\[4\|a\|\|b\|\leq \|a+b\|^2+\|a-b\|^2.\]
                即证\[2\|a\|\|b\|\leq \|a\|^2+\|b\|^2.\]
                \end{proof}
        \item \begin{proof}
                充分性:两边平方即可.\\
                必要性:两边平方后得到\[-2cu^Tv\leq c^2v^Tv.\]
                不妨$\|v\|\neq 0$,否则显然成立.\\
                当$c>0$时,\[ \frac{-2u^Tv}{v^Tv}\leq c.\]
                两边取$c$趋近于0的极限,得到$u^Tv\geq 0.$\\
                同理可以得到$u^Tv\leq 0.$\\
                因此$\left\langle u,v\right\rangle =0.$
                \end{proof}
        \item \begin{proof}
                由三角不等式$\|u-v\|\geq \mid \|u\|-\|v\|\mid $,取等当且仅当$u=\lambda v,\lambda\geq0$.
                因此\[\|u-v\|\geq \|u\|-\|v\|=\|u-(\|u\|^{-1}\cdot u)\|.\]
                \end{proof}
        \item $u^Tv=\frac14(\|u+v\|^2-\|u-v\|^2).$
        \item $\|u+v\|^2+\|u-v\|^2=2(\|u\|^2+\|v\|^2).$
    \end{enumerate}
\end{aaa}


\subsection{Problem 2}
\begin{qqq}
    使用最小平方法找到一条抛物线$y=a+bx+cx^2$,使得该抛物线可以尽可能地拟合以下所有点
    \[\{(-2,4),(-1,2),(0,1),(3,1)\}.\]
\end{qqq}
\begin{aaa}
    考虑方程\[\begin{pmatrix}
        1 &-2 &4\\1&-1&1\\1&0&0\\1&2&4\\1&3&9
    \end{pmatrix}\cdot\begin{pmatrix}
        a\\b\\c
    \end{pmatrix}=\begin{pmatrix}
        4\\2\\1\\1\\1
    \end{pmatrix}.\]
    这个方程无解,考虑两边同时左乘$\begin{pmatrix}
        1 &-2 &4\\1&-1&1\\1&0&0\\1&2&4\\1&3&9
    \end{pmatrix}^T$,得到
    \[\begin{pmatrix}
        5&2&18\\2&18&26\\18&26&114
    \end{pmatrix}\cdot\begin{pmatrix}
        a\\b\\c
    \end{pmatrix}=\begin{pmatrix}
        9\\-5\\31
    \end{pmatrix}.\]
    利用高斯消元法可以得到\[\begin{pmatrix}
        a&b&c
    \end{pmatrix}=\begin{pmatrix}
        \frac{86}{77}&-\frac{62}{77}&\frac{43}{154}
    \end{pmatrix}\]
    \newline
    定义点到抛物线的距离$d_i=\mid y_i-(a+bx_i+cx_i^2)\mid$,最小二乘法得到的最佳拟合函数就是使得$\sum d_i^2$最小,即$\| Y-AX\|^2$最小.
\end{aaa}








































\end{document}